
\section{Successioni di funzioni}
%Definizione 4.1. 
\begin{boxdef}[Convergenza puntiale]
    Sia $f_{n}: X_{n} \subset \mathbb{R} \rightarrow \mathbb{R}(n=1,2, \ldots)$ una successione di funzioni. Allora l'insieme
    \[D:=\left\{x \in \mathbb{R} \mid x \in X_{n}(\text { def }) \text { ed esiste finito } \lim _{n \rightarrow+\infty} f_{n}(x)\right\}\]
    è detto "insieme di convergenza puntuale" di $\left\{f_{n}\right\}$. La funzione
    \[f: D \rightarrow \mathbb{R}, \quad x \mapsto \lim _{n \rightarrow+\infty} f_{n}(x)\]
    è detta "(funzione) limite puntuale di $\left\{f_{n}\right\}$ "e si scrive: $f_{n} \rightarrow f$ in $D$.
\end{boxdef}

%Definizione 4.2. 
\begin{boxdef}
    Sia $f_{n}: X_{n} \subset \mathbb{R} \rightarrow \mathbb{R}(n=1,2, \ldots)$ una successione di funzioni. Allora si dice che " $\left\{f_{n}\right\}$ converge uniformemente in $E \subset \mathbb{R}$ a una funzione $f$ " se:
    \begin{enumerate}[i]
        \item $E \subset X_{n}($ def $)$;
        \item $f$ è definita nei punti di $E$;
        \item Si ha
        \[\sup _{x \in E}\left|f_{n}(x)-f(x)\right| \rightarrow 0, \text { quando } n \rightarrow+\infty\]
    \end{enumerate}
\end{boxdef}
%Esempio 4.1. 
\begin{ex}
    Si consideri la successione di funzioni $(n=1,2, \ldots)$
    \[f_{n}: X_{n}:=\mathbb{R} \backslash\{-n\} \rightarrow \mathbb{R}, \quad x \mapsto f_{n}(x):=\frac{1}{x+n}\]
    Allora $\left\{f_{n}\right\}$ converge uniformemente in $E:=[-10,+\infty)$ alla funzione
    \[f: \mathbb{R} \rightarrow \mathbb{R}, \quad x \mapsto f(x):=0\]
\end{ex}
%OSSERVAZIONE 4.1. 
\begin{oss}
    Se $f_{n}: X_{n} \subset \mathbb{R} \rightarrow \mathbb{R}(n=1,2, \ldots)$ converge uniformemente a $f$ in $E \subset \mathbb{R}$ allora $f_{n}$ converge puntualmente a $f$ in $E$, cioè $\lim _{n \rightarrow+\infty} f_{n}(x)=f(x)$ per ogni $x \in E$.
\end{oss}

%ESEMPIO 4.2. 
\begin{ex}\label{ex: 4.2}
    La successione
    \[f_{n}: \mathbb{R} \rightarrow \mathbb{R}, \quad x \mapsto x^{n}\]
    ha come insieme di convergenza $D:=(-1,1]$ e come funzione limite
    \[f(x):= \begin{cases}0 & \text { se } x \in(-1,1) \\ 1 & \text { se } x=1\end{cases}\]
    Inoltre $\left\{f_{n}\right\}$ converge uniformemente in ogni intervallo del tipo $[-a, a]$ con $0<a<1$, ma non converge uniformemente in $(-1,1)$ e quindi nemmeno in $(-1,1]$.
\end{ex}
%OsSERVazione 4.2.
\begin{oss}
    
    Consideriamo una funzione $g: X \subset \mathbb{R} \rightarrow \mathbb{R}$, un insieme $E \subset X \mathrm{e}$ $\bar{x} \in E$. Allora le seguenti due affermazioni non sono equivalenti:
    \begin{enumerate}[i]
        \item $g$ è continua in $\bar{x}$;
        \item $\left.g\right|_{E}$ è continua in $\bar{x}$.
    \end{enumerate}
    Infatti è ovvio che ($i$) implica ($ii$), ma in generale il viceversa non vale. Questo si vede molto chiaramente, per esempio, se $g:=\chi_{\mathbb{Q}}$ e $E:=\mathbb{Q}$ (con $\bar{x} \in \mathbb{Q}$ scelto arbitrariamente). Lo stesso vale per le sottostanti affermazioni ($iii$) e ($iv$): ovviamente ($iii$) implica ($iv$), ma in generale il viceversa è falso (come mostra lo stesso esempio).
    \begin{enumerate}[i, resume]
        \item $g$ è continua in $E$
        \item $\left.g\right|_{E}$ è continua.
    \end{enumerate}
    Un caso speciale in cui ($i$) e ($ii$) (risp. ($iii$) e ($iv$)) si equivalgono si ha per $E=X$ (i.e., se $g$ è continua).
\end{oss}

%TeORema 4.1 (**). 
\begin{shadedTheorem}[$**$\,|\,Trasmissione di continuità]\label{thm: 4.1 trasmissione continuità}
    Sia $f_{n}: X \subset \mathbb{R} \rightarrow \mathbb{R}(n=1,2, \ldots)$ una successione di funzioni che converge uniformemente in $E \subset X$ a una funzione $f$. Se (definitivamente - per $n\geq N$) le funzioni $\left.f_{n}\right|_{E}$ sono continue in $\bar{x} \in E$, allora anche $\left.f\right|_{E}$ è continua in $\bar{x}$. In particolare:
    \begin{enumerate}
        \item Se le $\left.f_{n}\right|_{E}$ sono continue (definitivamente), allora anche $\left.f\right|_{E}$ è continua.
        \item Se $E=X$ e le $f_{n}$ sono continue in $\bar{x}$ (definitivamente), allora anche $\left.f\right|_{X}$ è continua in $\bar{x}$.
    \end{enumerate}
\end{shadedTheorem}
\begin{proof}
    Sia $\{x_k\}_{k\in \N}\subset E$ tale che $x_k\to \bar x_k$. Proviamo che $\lim_{k\to +\infty}f(x_k)=f(x)$. Per $n\geq N$, si ha
    \[|f(x_k)-f(\bar x)|\overset{\text{DT}}{\leq} \underbrace{|f(x_k)-f_n(x_k)|}_{\leq \sup_E|f-f_n|}+|f_n(x_k)-f_n(\bar x)| + \underbrace{|f_n(\bar x)-f(\bar x)|}_{\leq \sup_E|f-f_n|} \leq 2 \sup_E|f_n-f| + |f_n(x_k)-f_n(\bar x)|.\]
    Inoltre, 
    \[0\leq \limsup_{k\to +\infty}|f(x_k)-f(\bar x)|\leq 2\sup_E|f_n-f|\xrightarrow[n\to +\infty]{}0\]
    Di conseguenza $\lim_{k\to +\infty}|f(x_k)-f(\bar x)|=0$, quindi $f$ è continua per successioni. Poiché $\R$ è primo-numerabile, la continuità sequenziale è equivalente alla continuità.
\end{proof}
% OSSERVAZIONE 4.3.
\begin{oss}
    Il fatto che la successione in Esempio 4.2 non converga uniformemente in $(-1,1]$ segue anche da Teorema \ref{thm: 4.1 trasmissione continuità}.
\end{oss}

% OSSERVAZIONE 4.4.
\begin{oss}
    Esempio \ref{ex: 4.2} mostra che l'ipotesi di convergenza uniforme assunta in Teorema \ref{thm: 4.1 trasmissione continuità} non può essere sostituita da quella di convergenza puntuale. Sotto questa ipotesi più semplice la continuità non passa al limite nemmeno quando $D=\mathbb{R}$. Per esempio, se
    \[f_{n}(x):= \begin{cases}0 & \text { se } x \in(-\infty, 0] \\ n x & \text { se } x \in(0,1 / n) \\ 1 & \text { se } x \in[1 / n,+\infty)\end{cases}\]
    allora $\left\{f_{n}\right\}$ è una successione di funzioni continue che converge puntualmente alla funzione caratteristica di $(0,+\infty)$ in $\mathbb{R}$.
\end{oss}

% Proposizione 4.1 (*). 
\begin{proposition}[$*$]\label{prop: 4.1}
    Sia $E \subset \mathbb{R}$. Allora l'insieme
    \[L^{\infty}(E):=\left\{f: E \rightarrow \mathbb{R}\ \left|\ \sup _{x \in E}| f(x) \mid<+\infty\right.\right\}\]
    con le ordinarie operazioni di somma e di moltiplicazione per scalare delle funzioni è uno spazio vettoriale. Inoltre la mappa
    \[L^{\infty}(E) \rightarrow[0,+\infty), \quad f \mapsto \sup _{x \in E}|f(x)|\]
    è una norma, indicata con $\|\cdot\|_{\infty, E}$.    
\end{proposition}
\begin{proof}
    Poiché $\R^E:=\{f: E\to \R\}$ è uno spazio vettoriale, verifichiamo semplicemente che $L^{\infty}(E)$ è un sottospazio vettoriale di $\R^E$, ovvero che è chiuso rispetto alle operazioni di somma e prodotto per scalari. Siano $f,g\in L^{\infty}(E)$. Allora per ogni $x\in E$
    \[|(f+g)(x)| = |f(x)+g(x)|\leq |f(x)|+|g(x)|\leq \sup_E|f|+\sup_E|g|\]
    quindi 
    \[\sup_E|f+g|\leq \sup_E|f| + \sup_E|g|<+\infty\]
    da cui $f+g\in L^{\infty}(E)$. Inoltre, l'espressione precedente può essere riscritta come
    \[\|f+g\|_{\infty, E}\leq \|f\|_{\infty, E} + \|g\|_{\infty, E}\]
    per cui abbiamo verificato che $\|\cdot\|_{\infty, E}$ soddisfa la disuguaglianza triangolare.

    Analogamente, per $f\in L^{\infty}(E)$ e $\lambda\in \R$ e per ogni $x \in X$, si ha
    \[|\lambda f(x)| = |\lambda||f(x)|\leq |\lambda|\sup_E|f|\]
    quindi
    \[\sup_E|\lambda f|\leq |\lambda|\sup_E|f|<+\infty\]
    da cui $\lambda f\in L^{\infty}(E)$. Inoltre, l'espressione precedente può essere riscritta come \[\|\lambda f\|_{\infty, E}=|\lambda|\|f\|_{\infty, E},\] per cui abbiamo verificato che $\|\cdot\|_{\infty, E}$ soddisfa la condizione di omogeneità.

    Rimane pertanto da verificare solo il primo assioma di norma. Per definizione segue banalmente che $\|f\|_{\infty,E}\geq 0$ per ogni $f\in L^{\infty}(E)$. Inoltre, se $\|f\|_{\infty,E}=0$, allora $0\leq \sup_E|f|=0$, cioè $|f(x)|=0$ per ogni $x\in E$, da cui $f=0$.
\end{proof}

% Proposizione $4.2(*)$.
\begin{proposition}[$*$]\label{prop: 4.2}
    Sia $E \subset \mathbb{R}$ e sia $f_{n}: X_{n} \subset \mathbb{R} \rightarrow \mathbb{R}(n=1,2, \ldots)$ una successione di funzioni. Allora:
    \begin{enumerate}
        \item Se $\left\{f_{n}\right\}$ converge uniformemente in $E$ (con $E\subset X_n$ definitivamente) a $f \in L^{\infty}(E)$, allora $\left.f_{n}\right|_{E} \in L^{\infty}(E)$ definitivamente e $\left.f_{n}\right|_{E} \rightarrow f$ in $L^{\infty}(E)$.
        \item $\left.\operatorname{Sia} f_{n}\right|_{E} \in L^{\infty}(E)$ definitivamente e sia $f \in L^{\infty}(E)$. Se $\left.f_{n}\right|_{E} \rightarrow f$ in $L^{\infty}(E)$, allora $f_{n}$ converge uniformemente in $E$ a $f$.
    \end{enumerate}
\end{proposition}
\begin{proof}
    \begin{enumerate}
        \item Sia $\{f_n\}_n$ convergente uniformemente a $f\in L^{\infty}(E)$. Proviamo che $\sup_E|f_n|<+\infty$. Per ogni $x\in E$, si hanno
        \[|f_n(x)|\leq |f_n(x)-f(x)|+|f(x)|\leq \sup_E|f_n-f|+\|f\|_{\infty,E}.\]
        Ma $\{f_n\}_n$ converge uniformemente a $f$, quindi per ogni $\epsilon >0$, esiste $N_\epsilon \in \N$ tale che per ogni $n\geq N_\epsilon$, si ha $\sup_E|f_n-f|<\epsilon$. In particolare, se $n\geq N_1$, $|f_n(x)|\leq 1+\|f\|_{\infty,E}$. Di conseguenza $\sup_E|f_n|\leq 1+\|f\|_{\infty,E}<+\infty$ per ogni $n\geq N_1$, cioè $\{f_n\}_n$ è definitivamente in $L^{\infty}(E)$. Per concludere, rimane da provare che 
        \[\bigl\|f_n|_E-f\bigr\|_{\infty, E} = \sup_E|f_n-f|\xrightarrow[n\to +\infty]{}0,\]
        il che segue direttamente dalla definizione di convergenza uniforme.
        \item Sia $f_n|_E\in L^{\infty}(E)$ definitivamente e sia $f\in L^{\infty}(E)$. Supponiamo che $f_n|_E\to f$ in $L^{\infty}(E)$. Allora esiste un $N\in \N$ per cui per ogni $n\geq N$ vale
        \[ \sup_E|f_n-f| = \bigl\|f_n|_E-f\bigr\|_{\infty, E}\xrightarrow[n\to +\infty]{}0,\]
        ovvero la successione converge uniformemente a $f$ in $E$.
    \end{enumerate}
\end{proof}

% OSSERVAzione 4.5. 
\begin{oss}
    La convergenza uniforme in $E$ è più generale di quella in $L^{\infty}(E)$. Per esempio, consideriamo la successione $f_{n}:(0,1) \rightarrow \mathbb{R}(n=1,2, \ldots)$ definita come segue
    \[f_{n}(x):=\frac{1}{x}+\frac{1}{n}, \quad x \in(0,1)\]
    e sia $f:(0,1) \rightarrow \mathbb{R}$ definita da
    \[f(x):=\frac{1}{x}, \quad x \in(0,1)\]
    Allora $f_{n}$ converge uniformemente a $f$ in $(0,1)$, ma $f, f_{n} \notin L^{\infty}((0,1))$.    
\end{oss}

% Proposizione 4.3 (*). 
\begin{proposition}[$*$]\label{prop: 4.3}
    Sia $W$ un sottospazio vettoriale chiuso di uno spazio di Banach $(V,\|\cdot\|)$. Allora $(W,\|\cdot\|)$ è uno spazio di Banach.
\end{proposition}
\begin{proof}
    Sia $\{w_n\}_{n\in \N}$ una successione di Cauchy in $\left(W, \|\cdot \|\big|_W\right)$. Allora essa è anche una successione di Cauchy in $(V, \|\cdot\|)$, quindi esiste un $v\in V$ tale che $\|w_n-v\|\xrightarrow[n\to+\infty]{} 0$. Tuttavia, poiché $W$ è chiuso in $(V,\|\cdot\|)$, $\overline {W}^{seq}\subset \overline W = W$, quindi $w\in W$. Segue quindi che $\|w_n-v\|\big|_W\xrightarrow[n\to+\infty]{} 0$, quindi la successione converge in $W$.
\end{proof}
% Teorema $4.2(* *)$. 
\begin{shadedTheorem}[$**$]\label{thm: 4.2}
    Sia $E \subset \mathbb{R}$. Allora:
    \begin{enumerate}
        \item $\left(L^{\infty}(E),\|\cdot\|_{\infty, E}\right)$ è uno spazio di Banach;
        \item $C_{b}(E):=C(E) \cap L^{\infty}(E)$ è un sottospazio vettoriale chiuso di $\left(L^{\infty}(E),\|\cdot\|_{\infty, E}\right) e$ quindi $\left(C_{b}(E),\|\cdot\|_{\infty, E}\right)$ è uno spazio di Banach. In particolare se $E$ è compatto, dato che in tal caso $C_{b}(E)=C(E)$, si ha che $\left(C(E),\|\cdot\|_{\infty, E}\right)$ è uno spazio di Banach.
    \end{enumerate}
\end{shadedTheorem}
\begin{proof}
    \begin{enumerate}
        \item Sia $\{f_n\}_n$ una successione di Cauchy in $(L^\infty(E), \|\cdot\|_{\infty,E})$ (semplificheremo la notazione con $\|\cdot\|:=\|\cdot\|_{\infty, E}$). Poiché $\forall x \in E$, si ha $|f_{n+k}(x)-f_n(x)|\leq \sup_E|f_{n+k}-f_n| = \|f_{n+k}-f_n\|$, la successione $\{f_n(x)\}_{n\in \N}\subset \R$ è una successione di Cauchy. Poiché $\R$ è uno spazio di Banach, $\{f_n(x)\}_{n\in \N}$ converge, quindi $\{f_n\}_{n\in\N}$ converge puntualmente: indichiamo con $f$ il limite puntuale.
        \begin{itemize}
            \item Proviamo inizialmente che $f\in L^\infty(E)$. Sia $x\in E$, allora per ogni $n\in \N$ e $k\in \N^*$, vale
            \begin{align*}
                |f(x)|&\leq |f(x)-f_n(x)|+|f_n(x)|\leq |f(x)-f_{n+k}(x)| + |f_{n+k}(x)-f_n(x)| + |f_n(x)| \leq \\ &\leq |f(x)-f_{n+k}(x)| + \|f_{n+k}-f_n\| + \|f_n\|
            \end{align*}
            Ora, per definizione di successione di Cauchy, per ogni $\epsilon >0$ esiste un $N_\epsilon\in \N$ tale che 
            \[\|f_n+k -f_n\|\leq \epsilon \qquad \forall n > N_\epsilon, \forall k\in \N^*.\]
            In particolare, con $\epsilon = 1$, $n = N_1$ e $k$ arbitrario, vale
            \[|f(x)|\leq \underbrace{|f(x)-f_{N_1+k}(x)|}_{\xrightarrow[k\to +\infty]{}0} + 1 + \|f_{N_1}\|.\]
            Per $k\to +\infty$ otteniamo quindi 
            \[|f(x)|\leq 1 + \|f_{N_1}\|\qquad \implies \qquad \sup_E|f| \leq 1 + \|f_{N_1}\|<+\infty, \qquad \text{ovvero} f\in L^\infty(E).\]
            \item Verifichiamo la convergenza in $L^\infty$. Sia $x\in E$, allora per ogni $n\in \N$ e $k\in \N^*$, 
            \[|f_n(x)-f(x)|\leq |f_n(x)|-|f_{n+k}(x)| + |f_{n+k}(x)-f(x)| \leq \underbrace{\|f_n-f_{n+k}\|}_{\leq \varepsilon \text{ se }n\geq N_\epsilon} + |f_{n+k}(x)-f(x)|.\]
            Sia ora $\epsilon >0$ qualsiasi, allora 
            \[|f_n(x)-f(x)|\leq \epsilon + \underbrace{|f_{n+k}(x)-f(x)|}_
            {\xrightarrow[k\to +\infty]{}0}\]
            quindi per $k\to +\infty$ otteniamo $|f_n(x)-f(x)|\leq \epsilon$. In particolare, poiché ciò vale per ogni $x\in E$ 
            \[\|f_n-f\| = \sup_E |f_n-f|\leq \varepsilon\qquad i.e.\ \|f_n-f\|\xrightarrow[n\to +\infty]{}0.\] 
        \end{itemize}
        \item È sufficiente provare che $C_b(E)$ è un sottospazio vettoriale chiuso di $(L^\infty(E), \|\cdot\|_{\infty,E})$. Poiché $L^\infty(E)$ è normato, è metrico, quindi primo numerabile. Di conseguenza la chiusura è equivalente alla chiusura sequenziale. Sia quindi $\{f_n\}_{n\in \N}$ una successione di funzioni in $C_b(E)$ e $f\in L^\infty(E)$ il suo limite in $L^\infty$. Allora poiché la convergenza $L^\infty$ implica la convergenza uniforme, per trasimissione di continuità, $f\in C(E)\cap L^\infty(E) = C_b(E)$.
    \end{enumerate}
\end{proof}
% OSSERVAZIONE 4.6.
\begin{oss}
    Esistono spazi vettoriali normati non completi (osserviamo che tali spazi non possono avere dimensione finita, in quanto ogni spazio vettoriale normato di dimensione finita è completo). Per esempio, consideriamo l'insieme dei polinomi a cofficienti reali in $(0,1 / 2)$, cioè
    \[\mc{P}:=\left\{\left.f\right|_{(0,1 / 2)} \mid f \in \mathbb{R}[x]\right\}\]
    Allora $\mc{P}$ è un sottospazio vettoriale di $L^{\infty}((0,1 / 2))$. Consideriamo la successione $\left\{f_{n}\right\} \subset$ $\mc{P}$ con
    \[f_{n}(x):=1+x+x^{2}+\ldots+x^{n}, \quad x \in(0,1 / 2)\]
    Osserviamo che (per ogni $n, k$ con $n, k>0$ ):
    \[0<f_{n+k}(x)-f_{n}(x)=x^{n+1}\left(1+x+\ldots+x^{k-1}\right)<x^{n+1} \sum_{i=0}^{+\infty} x^{i}=\frac{x^{n+1}}{1-x}\]
    da cui si ottiene
    \[\left\|f_{n+k}-f_{n}\right\|_{\infty,(0,1 / 2)} \leq \frac{1 / 2^{n+1}}{1 / 2}=\frac{1}{2^{n}}\]
    Ne segue che $\left\{f_{n}\right\}$ è una successione di Cauchy in $\left(\mc{P},\|\cdot\|_{\infty,(0,1 / 2)}\right)$, quindi anche in $\left(L^{\infty}((0,1 / 2)),\|\cdot\|_{\infty,(0,1 / 2)}\right)$. Per Teorema 4.2 esiste $f \in L^{\infty}((0,1 / 2))$ tale che
    \[\lim _{n \rightarrow+\infty}\left\|f_{n}-f\right\|_{\infty,(0,1 / 2)}=0\]
    Ma tale $f$ può essere calcolata esplicitamente. Infatti essa deve essere anche il limite puntuale delle $f_{n}$, per cui e in effetti sappiamo che
    \[f(x)=\lim _{n \rightarrow+\infty} f_{n}(x)=\lim _{n \rightarrow+\infty}\left(1+x+\ldots+x^{n}\right)=\frac{1}{1-x}\]
    per ogni $x \in(0,1 / 2)$. Si vede così che $f \notin \mc{P}$.
    
\end{oss}

Dalla formula di Taylor con resto in forma integrale, otteniamo il seguente risultato.

% Proposizione 4.4 (*).
\begin{proposition}[$*$]\label{prop: 4.4}
    Si considerino $f \in C^{\infty}(]a, b[)$ e $x_{0} \in\ ]a, b[$, dove $]a, b[$ è un qualsiasi intervallo limitato di $\mathbb{R}$. Si supponga inoltre che esista una costante $C \geq 0$ tale che
    \[\sup _{x \in]a, b[}\left|f^{(n)}(x)\right| \leq C^{n}\]
    per $n>N$, con $N\in \N$ sufficientemente grande. Allora $f \in L^{\infty}(]a, b[)$ e \[\lim _{n \to \infty}\left\|T_{x_{0}, n}-f\right\|_{\infty,(a, b)}=0\], dove $T_{x_{0}, n}$ indica il polinomio di Taylor di grado $n$ con centro in $x_{0}$ (relativo alla funzione $f$), cioè
    \[T_{x_{0}, n}(x):=f\left(x_{0}\right)+f^{(1)}\left(x_{0}\right)\left(x-x_{0}\right)+\frac{f^{(2)}\left(x_{0}\right)}{2}\left(x-x_{0}\right)^{2}+\ldots+\frac{f^{(n)}\left(x_{0}\right)}{n!}\left(x-x_{0}\right)^{n}.\]
\end{proposition}
Osserviamo innanzitutto che questa proposizione affera un tipo di convergenza più forte di quella vista in Analisi A.
\begin{proof}
    Ricordando la formula di Taylor con resto di Lagrange, per ogni $n\in \N^*$, per ogni $x,x_0\in\ ]a,b[$ esiste uno $\xi(x_0,x,n)\in\ ]x_0,x[$ tale che 
    \[f(x) = T_{x_0,n}(x)+\frac{f^{(n+1)}(\xi(x_0,x,n))}{(n+1)!}(x-x_0)^{n+1}.\] 
    Proviamo che $f\in L^\infty$: sviluppando fino all'$N$-esimo termine, per ogni $x\in ]a,b[$ si ha
    \[|f(x)|\leq \underbrace{|T_{x_0,N}(x)|}_{\leq\|T_{x_0,N}(x)\|_{\infty,]a,b[}}+\frac{|f^{(n+1)}(\xi(x_0,x,n))|}{(n+1)!}|x-x_0|^{n+1} \leq \|T_{x_0,N}(x)\|_{\infty,]a,b[}+\frac{C^{n+1}}{(n+1)!}(b-a)^{n+1}<+\infty.\] 
    In particolare, questa proprietà vale per ogni $x\in\ ]a,b[$, quindi 
    \[\sup_{]a,b[}|f| \leq \|T_{x_0,N}(x)\|_{\infty,]a,b[}+\frac{C^{n+1}}{(n+1)!}(b-a)^{n+1}<+\infty\]
    quindi $f\in L^{\infty}(]a,b[)$. Rimane ora da provare la relazione di convergenza: Applicando ancora il teorema di Taylor-Lagrange otteniamo 
    \[|f(x)-T_{x_0,n}(x)|= \frac{|f^{(n+1)}(\xi(x_0,x,n))|}{(n+1)!}|x-x_0|^{n+1} \leq\frac{C^{n+1}}{(n+1)!}(b-a)^{n+1}\]
    e, poiché ciò vale per ogni $x\in \ ]a,b[$,
    \[\sup_{]a,b[}\leq\frac{C^{n+1}}{(n+1)!}(b-a)^{n+1}\xrightarrow[n\to +\infty ]{}0.\]
    In quanto esso è l'addendo generale della serie di potenze di $e^{x}$, che ha raggio di convergenza infinito, calcolata in $C(b-a)$.
\end{proof}
Vale il seguente teorema di passaggio al limite nell'integrale.

% Proposizione $4.5(*)$.
\begin{proposition}[$*$]\label{prop: 4.5}
Sia $\left\{f_{n}\right\}$ una successione convergente a $f$ in $\left(C([a, b]),\|\cdot\|_{\infty,[a, b]}\right)$. Allora
\[\lim _{n \rightarrow+\infty} \int_{a}^{b} f_{n}=\int_{a}^{b} f\]
\end{proposition}
\begin{proof}
    È sufficiente applicare la linearità dell'integrale e stimare come segue per ottenere 
    \[\left|\int_a^bf_n-\int_a^bf\right| = \left|\int_a^b(f_n-f)\right|\leq \int_a^b|f_n-f| \leq \int_a^b\|f_n-f\|_{\infty,[a,b]} = \|f_n-f\|_{\infty,[a,b]}(b-a)\xrightarrow[n\to +\infty ]{}0.\]
\end{proof}

% OSSERVAZIONE 4.7.
\begin{oss}
    Proposizione \ref{prop: 4.5} segue banalmente dal teorema dalla convergenza dominata di Lebesgue (cfr. Teorema 2.7 e Teorema 2.4), osservando prima che, per $n$ sufficientemente grande, si ha
    \[\left|f_{n}(x)\right| \leq\left|f_{n}(x)-f(x)\right|+|f(x)| \leq\left\|f_{n}-f\right\|_{\infty,[a, b]}+\|f\|_{\infty,[a, b]} \leq 1+\|f\|_{\infty,[a, b]}\]
    per ogni $x \in[a, b]$
\end{oss} 

Come corollario otteniamo anche questo risultato sul passaggio al limite della derivata.

% Corollario $4.1(*)^{*}$. 
\begin{corollary}[$*$]\label{cor: 4.1}
    Sia $[a, b]$ un intervatto compatto di $\mathbb{R}$ e consideriamo una successione $\left\{f_{n}\right\} \subset C^{1}([a, b])$ tale che:
    \begin{enumerate}[i]
        \item $\left\{f_{n}\right\}$ converge puntualmente in $[a, b]$. Sia $f$ il limite;
        \item la successione delle derivate $\left\{f_{n}^{\prime}\right\}$ converge in $\left(C([a, b]),\|\cdot\|_{\infty,[a, b]}\right)$. Sia $g$ il limite.
    \end{enumerate}
    Allora $f \in C^{1}([a, b])$ e più precisamente si ha $f^{\prime}=g$.
\end{corollary}
\begin{proof}
    Per il Teorema Fondamentale del Calcolo applicato a ${f_n}$, 
    \[f_n(x) = f_n(a)+\int_a^xf_n'\qquad \forall x \in [a,b].\]
    Passando al limite per $n\to +\infty$ otteniamo (possiamo passare al limite dentro al segno di integrale per la Proposizione precedente)
    \[f(x) = f(a)+\int_a^xg\qquad \forall x \in [a,b].\]
    Applicando ulteriormente il TFC segue la tesi.
\end{proof}
Esempi.

\section{Serie di funzioni generiche}
% DEFINIZIONE 4.3. 
\begin{boxdef}
    In uno spazio vettoriale normato $(V,\|\cdot\|)$, consideriamo una successione $\left\{v_{j}\right\} \subset V$. Diremo allora che "la serie dei $v_{j}$ converge totalmente (in $V$ )" se la serie dei $\left\|v_{j}\right\|$ converge (in $\mathbb{R}$ ), cioè se $\lim _{N \rightarrow+\infty} \sum_{j=1}^{N}\left\|v_{j}\right\|<+\infty$. Se una serie converge in senso usuale in $(V,\|\cdot\|)$, d'ora in poi diremo che essa "converge semplicemente (in $V)$ "(per enfatizzare la distinzione fra le due nozioni di convergenza).
\end{boxdef}

% Proposizione $4.6\left(^{*}\right)$. 
\begin{proposition}[$*$]\label{prop: 4.6}
Consideriamo uno spazio di Banach $(V,\|\cdot\|)$ e una successione $\left\{v_{j}\right\} \subset V$ tale che la serie dei $v_{j}$ converge totalmente in $V$. Allora tale serie converge anche semplicemente in $V$.
\end{proposition}
\begin{proof}
    Poniamo $s_n=\sum_{j=1}^Nv_j$ e $\sigma_n = \sum_{j=1}^n\|v_j\|$. Allora $\forall n,.k\in\N$ possiamo stimare
    \[\|s_{n+k}-s_n\| = \left\|\sum_{j=n+1}^{n+k}v_j\right\|\underset{\text{(D.T.)}}{\leq} \sum_{j=n+1}^{n+k}\left\|v_j\right\| = \sigma_{n+k}-\sigma_n.\]
    Ricordando che in uno spazio di Banach la convergenza è equivalente all'essere successione di Cauchy, abbiamo che, poiché $\{\sigma_n\}_n$ è di Cauchy, anche $\{s_n\}_n$ è di Cauchy, per cui $\sum_jv_j$ converge.
\end{proof}

% OSSERVAZIONE 4.8. 
\begin{oss}
    In uno spazio vettoriale normato non completo può capitare che una serie converga totalmente ma non semplicemente. Per esempio consideriamo lo spazio vettoriale definito in Osservazione 4.6, cioè $\left(\mc{P},\|\cdot\|_{\infty,(0,1 / 2)}\right)$ con
    \[\mc{P}:=\left\{\left.f\right|_{(0,1 / 2)} \mid f \in \mathbb{R}[x]\right\}\]
    e sia $\left\{v_{j}\right\} \subset \mc{P}$ la successione definita da
    \[v_{j}(x):=x^{j}, \quad x \in(0,1 / 2)\]
    Allora la serie delle $v_{j}$ converge totalmente ma non semplicemente in $\left(\mc{P},\|\cdot\|_{\infty,(0,1 / 2)}\right)$.
\end{oss}

Da Teorema 4.2 e da Proposizione 4.6 seguono subito i seguenti risultati.

% Corollario $4.2\left(^{\circ}\right)$. 
\begin{corollary}[$\circ$]
    Sia $E \subset \mathbb{R}$ e sia $\left\{f_{j}\right\}$ una successione in $L^{\infty}(E)$ tale che la serie delle $f_{j}$ converge totalmente in $\left(L^{\infty}(E),\|\cdot\|_{\infty, E}\right)$. Allora tale serie converge anche semplicemente in $\left(L^{\infty}(E),\|\cdot\|_{\infty, E}\right)$.
\end{corollary}
    
% Corollario $4.3\left({ }^{\circ}\right)$.
\begin{corollary}[$\circ$]
    Sia $E \subset \mathbb{R}$ e sia $\left\{f_{j}\right\}$ una successione in $C_{b}(E)$ tale che la serie delle $f_{j}$ converge totalmente in $\left(C_{b}(E),\|\cdot\|_{\infty, E}\right)$. Allora tale serie converge anche semplicemente in $\left(C_{b}(E),\|\cdot\|_{\infty, E}\right)$.
\end{corollary}
% Corollario $4.4\left(^{\circ}\right)$.
\begin{corollary}[$\circ$]
    Sia $E \subset \mathbb{R}$ compatto e sia $\left\{f_{j}\right\}$ una successione in $C(E)$ tale che la serie delle $f_{j}$ converge totalmente in $\left(C(E),\|\cdot\|_{\infty, E}\right)$. Allora tale serie converge anche semplicemente in $\left(C(E),\|\cdot\|_{\infty, E}\right)$.
\end{corollary}

Da Proposizione 4.5 si ottiene subito il seguente teorema di integrazione per serie.

% Corollario $4.5\left(^{\circ}\right)$. 
\begin{corollary}[$\circ$]
    Sia $[a, b]$ un sottoinsieme compatto di $\mathbb{R}$ e consideriamo una successione $\left\{f_{j}\right\} \subset C([a, b])$ tale che la serie delle $f_{j}$ converga semplicemente in $(C([a, b]), \| \cdot$ $\left.\|_{\infty,[a, b]}\right) \cdot$ Allora
    \[\int_{a}^{b} \sum_{j=1}^{+\infty} f_{j}=\sum_{j=1}^{+\infty} \int_{a}^{b} f_{j}\]
\end{corollary}
\begin{proof}
    Sia $s_N = \sum_{j=1}^N f_j$ allora esiste $s\in C([a,b])$ tale che $\|s_n-s\|_{\infty,[a,b]}\xrightarrow[N\to+\infty]{}0$. Ora, per il Teorema di passaggio al limite sotto al segno di integrale, vale 
    \[\int_a^b s = \lim_{N\to +\infty}\int_a^bs_N\]
    o, a meno di cambiare notazione, 
    \[\int_a^b \sum_j f_j = \sum_j\int_a^b f_j.\qedhere\]
\end{proof}

Da Corollario 4.1 segue poi immediatamente il seguente risultato di derivazione per serie. 

% Corollario $4.6\left(^{\circ}\right)$. 
\begin{corollary}[$\circ$]\label{cor: 4.6}
    Sia $[a, b]$ un sottoinsieme compatto di $\mathbb{R}$ e consideriamo una successione $\left\{f_{j}\right\} \subset C^{1}([a, b])$ tale che:
    \begin{enumerate}[i]
        \item La serie delle $f_{j}$ converge puntualmente in $[a, b]$. Sia $F$ il limite;
        \item La serie delle derivate $f_{j}^{\prime}$ converge semplicemente in $\left(C([a, b]),\|\cdot\|_{\infty,[a, b]}\right)$.
    \end{enumerate}

    Allora $F \in C^{1}([a, b])$ e più precisamente si ha
    \[F^{\prime}=\sum_{j=1}^{+\infty} f_{j}^{\prime}\]
\end{corollary}
\begin{proof}
    Sia $s_n = \sum_{j=1}^nf_j$ per ogni $n\in \N^*$, allora $\{s_n\}_n\subset C^1([a,b])$. Ora, per $(i)$
    \[s_N(x) = \sum_{j=1}^Nf_j(x) \xrightarrow[N\to +\infty]{}F(x)\in \R\qquad \forall x \in [a,b].\]
    Inoltre, per ipotesi $(ii)$, esiste $G\in C([a,b])$ tale che 
    \[\left\|\sum_{j=1}^Nf_j'-G\right\| = \|s_N'-G\|_{\infty, [a,b]}\xrightarrow[N\to +\infty]{}0.\]
    Applicando quindi il teorem,a di passaggio al limite sotto il segno di integrale per funzioni, otteniamo che $F\in C^1([a,b])$ e 
    \[F' = \lim_{N\to +\infty} s_N' = \lim_{N\to +\infty}\sum_{j=1}^Nf'_j = \sum_{j=1}^{+\infty}f_j'.\]
\end{proof}

Osserviamo che vale il seguente risultato (cfr. Proposizione 4.4).
% Corollario $4.7\left(^{\circ}\right)$.
\begin{corollary}[$\circ$]
    Si considerino $f \in C^{\infty}(a, b)$ e $x_{0} \in(a, b)$, dove $(a, b)$ è un qualsiasi intervallo limitato di $\mathbb{R}$. Si supponga inoltre che esista una costante $C \geq 0$ tale che
    \[\sup _{x \in(a, b)}\left|f^{(n)}(x)\right| \leq C^{n}\]
    per $n$ sufficientemente grande. Allora la serie di potenze
    \[\sum_{n=0}^{+\infty} \frac{f^{(n)}\left(x_{0}\right)}{n!}\left(x-x_{0}\right)^{n}\]
    converge totalmente in $\left(C_{b}(a, b),\|\cdot\|_{\infty,(a, b)}\right)$ e quindi anche semplicemente in $\left(C_{b}(a, b), \| \cdot\|_{\infty,(a, b)}\right)$ (per Proposizione 4.6).
\end{corollary}
Il teorema generalizza ad un qualsiasi insieme $E$ compatto, per il quale $C_b(E)$ coincide con $C(E)$.
\begin{proof}
    Per $x\in E$ poniamo $v_n(x) = \frac{f^n(x)}{n!}(x-x_0)^n$. Otteniamo quindi la successione $\{v_n\}_{n\in \N}\subset C(E)$, la cui serie $\sum_nv_n$ converge totalmente in $(C(E),\|\cdot\|_{\infty, E})$. Infatti, fissato $R>0$ tale che $x\in [-R,R]$ e $E\subset [-R,R]$, per ogni $x\in E$, $|x-x_0|<2R$ e vale la stima
    \[\|v_n\|_{\infty, E} = \sup_E \left|\frac{f^{(n)}(x_0)}{n!}(x-x_0)^n\right| = \frac{|f^{(n)}|}{n!}\sup_{x\in E}|x-x_0|^n \leq \frac{C^N(2R)^N}{N!} \text{ (definitivamente).}\]
    La convergenza segue quindi per confronto. 
\end{proof}
    
\section{Serie di potenze}
% Proposizione $4.7(*)$. 
\begin{proposition}[$*$]\label{prop: 4.7}
    Data una serie di potenze
    \begin{equation}
    a_{0}+\sum_{j=1}^{+\infty} a_{j} x^{j} \quad\left(a_{j} \in \mathbb{R}\right) \label{eq: 4.3.1}
    \end{equation}
    indichiamo con $D$ il suo insieme di convergenza puntuale, i.e.,
    \[D:=\left\{x \in \mathbb{R}\ \left| \text { esiste finito } \lim _{n \rightarrow+\infty} a_{0}+\sum_{j=1}^{n} a_{j} x^{j}\right.\right\}\]
    Posto
        \[R:=\sup \{|y|: y \in D\},\]
    valgono le seguenti proprietà:
    \begin{enumerate}
        \item \label{prop: 4.7.1} $0\in D$ e $D\neq \{0\}\iff R>0$
        \item \label{prop: 4.7.2} Se $y \in D \backslash\{0\}$, allora \eqref{eq: 4.3.1} converge totalmente in $\left(C([-r, r]),\|\cdot\|_{\infty,[-r, r]}\right)$, per ogni $r<|y|$
        \item \label{prop: 4.7.3} Vale la catena di inclusioni
        \[]-R, R[ \subset D \subset[-R, R]\]
        Il numero $R$ è detto \say{raggio di convergenza della serie di potenze \eqref{eq: 4.3.1}};
        \item \label{prop: 4.7.4} Se $R>0$, la funzione
    \[x \mapsto a_{0}+\sum_{j=1}^{+\infty} a_{j} x^{j}, \quad x \in(-R, R)\]
    è continua.
    \end{enumerate}
\end{proposition}
\begin{proof}
    \begin{enumerate}
        \item Banale per definizione di $D$ e $R$.
        \item Supponiamo $R>0$ e sia $0<r<R$. Allora per definizione di estremo superiore esiste un $y\in D$ tale che $r<|y|\leq R$ e \eqref{eq: 4.3.1} converge in $y$. In particolare definitivamente la successione $\{|a_jy^j|\}_j$ è minore o uguale di 1. Posto quindi $v_j(x):=a_jx^j$ per $x\in [-r,r]$, si ha $v_j\in C([-r,r])$. Inoltre \[\|v_j\|_{\infty, [-r,r]} = \sup_{x\in[-r,r]}|a_jx^j| = |a_j|\left(\sup_{x\in [-r,r]}|x|\right)^j = |a_j|r^j\qquad \forall j\in \N.\]
        Possiamo quindi formulare la seguente stima:
        \[\|v_j\|_{\infty, [-r,r]} = |a_j|r^kj = \underbrace{|a_jy^j|}_{\leq 1 \text{ def.}}\left(\frac{r}{|y|}\right)^j \leq \left(\frac{r}{|y|}\right)^j\]
        abbiamo quindi che la serie di potenze nel punto arbitrario $x$ è maggiorata da una serie geometrica convergente, per cui converge a sua volta. In particolare abbiamo quindi la convergenza totale in $(C([-r,r]), \|\cdot\|_{\infty, [-r,r]})$.
        \item Per definizione di estremo superiore si ha banalmente $D \subset[-R, R]$. Rimane da provare l'altra inclusione. Sia $x\in ]-R,R[$ (possiamo supporre senza perdita di generalità $x\neq 0$ in quanto sussiste $(1)$), o meglio $0<|x|<R$. Vogliamo provare che $x\in D$.
        
        Per $(2)$ con $r=|x|$, \eqref{eq: 4.3.1} converge totalmente in $(C([-|x|,|x|]), \|\cdot\|_{\infty, [-|x|,|x|]})$. In particolare, poiché la convergenza totale implica la convergenza semplice, \eqref{eq: 4.3.1} converge semplicemente in $(C([-|x|,|x|]), \|\cdot\|_{\infty, [-|x|,|x|]})$. Da questo segue quindi la convergenza uniforme di \eqref{eq: 4.3.1} in $[-|x|,|x|]$, da cui discende a sua volta la convergenza puntuale in $[-|x|,|x|]$. In particolare, poiché $x\in [-|x|,|x|]$, la serie converge in $x$.
        \item Sia $s(x) = a_0+\sum_{i=1}^{+\infty}a_ix^i$ con $x \in ]-R,R[\subset D$. Siano inoltre $\bar x\in ]-R,R[$ e $r\in ]0,R[$ con $r>|\bar x|$. 
        \begin{itemize}[label=$\implies$]
            \item \eqref{eq: 4.3.1} converge totalmente in $(C([-r,r]), \|\cdot\|_{\infty, [-r,r]})$;
            \item \eqref{eq: 4.3.1} converge semplicemente in $(C([-r,r]), \|\cdot\|_{\infty, [-r,r]})$;
            \item \eqref{eq: 4.3.1} converge uniformemente in $[-r,r]$;
            \item \eqref{eq: 4.3.1} converge puntualmente;
        \end{itemize}
        e, poiché per unicità del limite, il limite di convergenza uniforme coincide con quello di convergenza puntuale, la serie converge uniformemente a $f$. In particolare per il Teorema di Trasmissione di Continuità (Teorema \ref{thm: 4.1 trasmissione continuità}) $s|_{[-r,r]}$ è continua in $\bar x$. In particolare anche $s$ è continua in $\bar x$.
    \end{enumerate}
\end{proof}

% Proposizione $4.8(* *)$.
\begin{proposition}[$**$]\label{prop: 4.8}
    Sia data una serie di potenze
    \[\sum_{n=0}^{+\infty} a_{n} x^{n} \quad\left(a_{n} \in \mathbb{R}\right)\]
    con raggio di convergenza $R$ e poniamo
    \[\rho:=\limsup _{n \rightarrow+\infty}\left|a_{n}\right|^{\sfrac{1}{n}}\]
    Allora
    \[R= \begin{cases}0 & \text { se } \rho=+\infty \\ \sfrac{1}{\rho} & \text { se } \rho \in(0,+\infty) \\ +\infty & \text { se } \rho=0\end{cases}\]
    Inoltre, se per $n$ sufficientemente grande si ha $a_{n} \neq 0$ ed esiste il limite
    \[\lim _{n \rightarrow+\infty}\left|\frac{a_{n+1}}{a_{n}}\right|\]
    allora tale limite coincide con $\rho$.
\end{proposition}
\begin{proof}Verifichiamo prima l'uguaglianza tra $R$ e il reciproco formale di $\rho$.
    \begin{itemize}[leftmargin = 50pt]
        \item[\boxed{\rho = +\infty}] Proviamo che $R=0$, ovvero che $D = \{0\}$. Sia $x\neq 0$, allora per definizione di limsup $\sfrac{1}{|x|}<|a_n|^{\sfrac{1}{n}}$ per infiniti $n$, ovvero $|a_nx^n|<1$ per infiniti $n$. Di conseguenza la serie di potenze non può convergere (il termine generico dovrebbe essere infinitesimo). Abbiamo quindi che $x\notin D$.
        \item[\boxed{\rho = 0}] Proviamo che $R=+\infty$, ovvero che $D=\R$. Sia $x\in \R$ qualsiasi, se $x=0$ la tesi è banale, altrimenti per definizione di limsup, $|a_n|^{\sfrac{1}{n}}<\sfrac{1}{2|x|}$ definitivamente, ovvero $a_n x^n<\sfrac{1}{2^n}$, quindi la serie converge per confronto con la serie geometrica di ragione $\sfrac{1}{2}<1$. Segue pertanto che $x\in D$; dall'arbitrarietà di $x\in \R$ segue la tesi.
        \item[\boxed{\rho \in \R_{>0}}] Sia $x\in \ ]0,\sfrac{1}{\rho}[$, allora esiste un $0<\sigma<1$ tale che $|x|<\sfrac{\sigma}{\rho}<\frac{1}{\rho}$. In particolare vale (definitivamente, per definizione di limsup)
        \[\rho < \frac{\sigma}{|x|} \quad \implies \quad |a_n|^{\sfrac{1}{n}}<\frac{\sigma}{|x|}\quad \implies \quad|a_nx^n|<\sigma^n.\]
        Per confronto con la serie geometrica di ragione $0<\sigma<1$ ottenmiamo che la serie converge in $x$ e, per l'arbitrarietà di $x \in\ ]0,\sfrac{1}{\rho}[$, $\frac{1}{\rho}\leq R$.

        D'altra parte, sia $x>\sfrac{1}{\rho}$, allroa $x>0$ e $\rho >\sfrac{1}{|x|}$. In particolare per infiniti indici vale 
        \[|a_nx^n|>1\]
        per cui la serie non converge in quanto il termine generico non è infinitesimo. Per l'arbitrarietà di $x$, vale $\sfrac{1}{\rho} \geq R$, per cui per quanto detto prima possiamo concludere.
    \end{itemize}
    Supponiamo ora che esista il limite $\ell = \lim\limits_{n \rightarrow+\infty}\left|\frac{a_{n+1}}{a_{n}}\right|$ e verifichiamo che vale $\ell = \rho$.
    \begin{itemize}[leftmargin = 50pt]
        \item[\boxed{\ell = 0}] Fissato $\epsilon >0$ aribtrariamente, esiste un $N\in \N$ tale che per ogni $n>N$ 
        \[\left|\frac{a_{n+1}}{a_n}\right|\leq \epsilon\qquad \text{o, equivalentemente}\qquad |a_{n+1}|=\epsilon|a_n|.\]
        Iterando questo ragionamento, otteniamo che per ogni $k>0$
        \[|a_{N+k}| \leq \epsilon ^k |a_N|\qquad \implies \qquad |a_{N+k}|^{\sfrac{1}{N+k}}\leq \varepsilon ^{\sfrac{k}{N+k}} |a_N|^{\frac{1}{N+k}}\xrightarrow[k\to+\infty]{} \epsilon\]
        quindi, siccome il membro a sinistra non dipende da $\epsilon$,
        \[\limsup_{k\to +\infty} |a_{N+k}|^{\sfrac{1}{N+k}}\leq \epsilon\quad \forall \varepsilon >0\qquad \implies \qquad \limsup_{k\to +\infty}|a_{N+k}|^{\sfrac{1}{N+k}} = 0.\]
        \item[\boxed{\ell = +\infty}] Analogamente, sia $M \in \ ]0,+\infty[$ fissato arbitrariamente, allora esiste un $N\in\N$ tale che per ogni $n>N$ si ha $|a_{N+k}|\geq M^k|a_n|$ per ogni $n\geq 1$. Segue pertanto che 
        \[|a_{N+k}|^{\sfrac{1}{N+k}} \geq M^{\sfrac{k}{N+k}}|a_N|^{\sfrac{K}{N+k}} \xrightarrow[k\to+\infty]{} M\qquad \implies \qquad \limsup_{k\to +\infty}|a_N+k|^{\sfrac{1}{N+k}}\geq M.\]
        Poiché il membro a sinistra non presenta dipendenza da $M$, per l'arbitrarietà di $M$, $\rho = \ell = +\infty$.
        \item[\boxed{\ell \in \R_{>0}}] Sia $\epsilon >0$ arbitrariio tale che $\ell -\epsilon >0$. Allora esiste un $N\in \N$ tale che
        \[\ell-\epsilon <\left|\frac{a_{n+1}}{a_n}\right|\quad \forall n>N.\]
        Procedendo come nei due casi precedenti otteniamo che 
        \[(\ell-\epsilon)^k|a_N| \leq |a_{N+k}| \leq (\ell +\varepsilon)^k |a_N|\quad \forall k\geq 1\]
        \[\implies (\ell-\epsilon)^{\sfrac{k}{N+k}}|a_N|^{\sfrac{1}{N+k}} \leq |a_{N+k}|^{\sfrac{1}{N+k}} \leq (\ell +\varepsilon)^{\sfrac{k}{N+k}} |a_N|^{\sfrac{1}{N+k}}\quad \forall k\geq 1\]
        e, passando al limite per $k\to +\infty$
        \[L-\varepsilon \leq \limsup_{k\to +\infty}|a_{N+k}|^{\sfrac{1}{N+k}}\leq L+\varepsilon.\]
        Per l'arbitrarietà di $\varepsilon$ e poiché $\rho = \limsup_{k\to +\infty}|a_{N+k}|^{\sfrac{1}{N+k}}$ non dipende da $\varepsilon$, $\rho = L$.\qedhere
    \end{itemize}
\end{proof}

% OSSERVAZIONE 4.9.
\begin{oss}
    In generale (se $a_{n} \neq 0$ definitivamente), non è detto che valga una delle uguaglianze
    \[\limsup _{n \rightarrow+\infty}\left|\frac{a_{n+1}}{a_{n}}\right|=\rho, \quad \liminf _{n \rightarrow+\infty}\left|\frac{a_{n+1}}{a_{n}}\right|=\rho .\]
    Un esempio è dato dalla serie
    \[x+2 x^{2}+x^{3}+4 x^{4}+x^{5}+6 x^{6}+\ldots\]
    per la quale si ha $\rho=1 \mathrm{e}$
    \[\limsup _{n \rightarrow+\infty}\left|\frac{a_{n+1}}{a_{n}}\right|=+\infty, \quad \liminf _{n \rightarrow+\infty}\left|\frac{a_{n+1}}{a_{n}}\right|=0\]
\end{oss}
% Osservazione 4.10.
\begin{oss}
    Stando a (2) di Proposizione 4.7, si possono presentare al più i seguenti casi: $D=(-R, R), D=(-R, R], D=[-R, R), D=[-R, R]$. I seguenti esempi mostrano che questi casi si possono effettivamente presentare tutti e quattro:
    \begin{itemize}
        \item La serie $\sum_{j=1}^{+\infty} x^{j}$ ha raggio di convergenza 1 e insieme di convergenza $(-1,1)$;
        \item La serie $\sum_{j=1}^{+\infty} \frac{1}{j} x^{j}$ ha raggio di convergenza 1 e insieme di convergenza $[-1,1)$;
        \item La serie $\sum_{j=1}^{+\infty} \frac{(-1)^{j}}{j} x^{j}$ ha raggio di convergenza 1 e insieme di convergenza $(-1,1]$;
        \item La serie $\sum_{j=1}^{+\infty} \frac{1}{j^{2}} x^{j}$ ha raggio di convergenza 1 e insieme di convergenza $[-1,1]$. 
    \end{itemize}
    
\end{oss}
%Proposizione $4.9(*)$. 
\begin{proposition}[$*$]\label{prop: 4.9}
    Le due serie di potenze
    \[(1)\ \ \sum_{n=0}^{+\infty} a_{n} x^{n}, \qquad\qquad (2)\ \ \sum_{n=1}^{+\infty} n a_{n} x^{n-1}\]
    hanno lo stesso raggio di convergenza.
\end{proposition}
\begin{proof}
    Indichiamo con $\{A_n\}_n$ le somme partiali di $(1)$, con $\{B_n\}_n$ le somme parziali di
    \[\sum_{n=0}^{+\infty}na_nx^n\tag{3}.\]
    Allora vale $B_n = xA_n$. Di conseguenza, raccogliendo $x$ otteniamo che $D_3\setminus \{0\} = D_2\setminus \{0\}$ (dobbiamo escludere lo 0 perché in quel caso non si può procedere al raccoglimento), e ricordando che le serie convergono sempre in $x=0$, $D_3 = D_2$. Per definizione segue quindi $R_3 = R_2$ e $\rho_3 = \rho_2$. Infine, 
    \[\rho_3 = \limsup_{n\to +\infty}|na_n|^{\sfrac{1}{n}} = \limsup_{n\to +\infty}\cancel{|n|^{\sfrac{1}{n}}}|a_n|^{\sfrac{1}{n}} = \limsup_{n\to +\infty}|a_n|^{\sfrac{1}{n}} = \rho_1,\qquad (\sqrt[n]{a_n}\xrightarrow[n\to +\infty]{}1)\]
    ovvero $R_3 = R_2 = R_1$.
\end{proof}


% Osservazione 4.11. 
\begin{oss}    
    Le due serie di potenze considerate in Proposizione 4.9 possono comportarsi in modo differente negli estremi $\pm R$ dell'insieme di convergenza. Per esempio gli insiemi di convergenza di
    \[\sum_{n=1}^{+\infty} \frac{1}{n} x^{n}, \quad \sum_{n=1}^{+\infty} x^{n-1}\]
    sono $[-1,1)$ e $(-1,1)$, rispettivamente.
\end{oss}

Da Corollario 4.6, Proposizione 4.7 e Proposizione 4.9 segue subito il seguente risultato sulla derivazione delle serie di potenze.
% Proposizione 4.10 (**). 
\begin{proposition}[$**$] \label{prop: 4.10}
    Sia data una serie di potenze
    \[\sum_{n=0}^{+\infty} a_{n} x^{n}\]
    con raggio di convergenza $R>0$. Allora la funzione
    \[x \mapsto S(x):=\sum_{n=0}^{+\infty} a_{n} x^{n}, \quad x \in(-R, R)\]
    appartiene a $C^{\infty}((-R, R))$ e (per ogni $\left.k \geq 1\right)$ si ha
    \[S^{(k)}(x)=\sum_{n=k}^{+\infty} n(n-1) \cdots(n-k+1) a_{n} x^{n-k}, \quad x \in(-R, R)\]
\end{proposition}
\begin{proof}~
    \textbf{Notazione.} Indichiamo con $\Sigma_0$ la scrittura formale $\sum_{n=0}^{+\infty}a_nx^n$ e con $S_0(x) = S(x)$ la sua somma. Analogamente indichiamo con $\Sigma_n$ la scrittura formale $\sum_{n=k}^{+\infty} n(n-1) \cdots(n-k+1) a_{n} x^{n-k}$ e con $S_k(x)$ la sua somma. 
    
    Applicando iterativamente Proposizione \ref{prop: 4.7}\ref{prop: 4.7.4}, otteniamo che tutte le $S_k$ sono continue. Per ottenere la tesi è sufficiente provare che per ogni $r\in \ ]0,R[$, $S\big|_{[-r,r]}\in C^{\infty}([-r,r])$ e 
    \[S^{(k)}(x)=\sum_{n=k}^{+\infty} n(n-1) \cdots(n-k+1) a_{n} x^{n-k}, \quad \forall x \in[-r,r].\]
    Per ogni $k$, fissato arbitrariamente $r\in ]0,R[$, per Proposizione \ref{prop: 4.7}\ref{prop: 4.7.2} si ha che $\Sigma_k$ converge totalmente in $(C([-r,r]), \|\cdot\|_{\infty, [-r,r]})$ a $S\big|_{[-r,r]}$. In particolare converge semplicemente nel medesimo spazio, e quindi uniformemente e puntualmente. Di conseguenza $\Sigma_0$ converge puntualmente in $[-r,r]$ e $\Sigma_1$ converge semplicemente in $(C([-r,r]), \|\cdot\|_{\infty,[-r,r]})$. Possiamo quindi applicare il Corollario \ref{cor: 4.6} e ottenere che
    \[S_0\big|_{[-r,r]} \in \mc C^1([-r,r]) \qquad \text{ e }\qquad \left( S\big|_{[-r,r]} \right)' = \left( S_0\big|_{[-r,r]} \right)' = S_1\big|_{[-r,r]}.\]
    Iterando il procedimento per ogni $k$ segue la tesi.
\end{proof}
Esempi.
