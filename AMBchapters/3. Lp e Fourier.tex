\section{Spazi $L^{p}$}
% Osservazione 3.1. 
\begin{oss}
    
    Sia $(X, \mc{A}, \mu)$ uno spazio con misura e sia $f: X \rightarrow \overline{\mathbb{R}}$ una funzione misurabile. Indichiamo allora con $\mc{M}_{f}$ l'insieme dei maggioranti essenziali di $|f|$ e cioè:
    \[
    \begin{aligned}
    \mc{M}_{f}: & =\{M \in[0,+\infty]|M \geq| f(x) \mid \text { per } \mu \text {-q.o. } x\} \\
    & =\{M \in[0,+\infty] \mid \mu(\{x|M<| f(x) \mid\})=0\} .
    \end{aligned}
    \]
    
    Allora valgono i seguenti fatti:
    
    \begin{itemize}
      \item $\mc{M}_{f}$ è una semiretta destra;
      \item $\mc{M}_{f}$ è chiusa, i.e. $\inf \mc{M}_{f} \in \mc{M}_{f}$. Infatti, dato che $+\infty \in \mc{M}_f$, abbiamo $\mc{M}_f \neq \varnothing$. Posto quindi $m = \inf \mc{M}_f$ si ha che $\forall j \in \Z^+, m + \frac{1}{j} \in \mc{M}_f$ dunque $\mu\left(\left\{x \in X : |f(x)|>m + \frac{1}{j}\right\}\right) = 0$ e pertanto la loro unione, ovvero l'insieme $\{x\in X : |f|>m\}$ ha misura nulla per la $\sigma$-subaddittività, dunque $m\in \mc{M}_f$.
    \end{itemize}
\end{oss}

% Definizione 3.1. 
\begin{boxdef}
    Siano $(X, \mc{A}, \mu)$ uno spazio con misura e $p \in[1,+\infty]$. Per ogni funzione misurabile $f: X \rightarrow \overline{\mathbb{R}}$, poniamo
    \[
    \|f\|_{p}:= \begin{cases}\left(\int_{X}|f|^{p} \d \mu\right)^{1 / p} & \text { se } 1 \leq p<+\infty \\ \min \mc{M}_{f} & \text { se } p=+\infty\end{cases}
    \]
    Indicheremo con $L^{p}(X)$ la classe delle funzioni misurabili $f: X \rightarrow \overline{\mathbb{R}}$ tali che $\|f\|_{p}<\infty$.
\end{boxdef}
In particolare con questa definizione potremo scrivere $\mc M_f =[\|f\|_\infty, +\infty] $.
% Teorema $3.1\left(^{*}\right)$. 
\begin{shadedTheorem}[$*$]\label{thm: 3.1 pre-norma Lp}
    Siano $(X, \mc{A}, \mu)$ spazio con misura, $p \in[1,+\infty]$ e $f: X \rightarrow \overline{\mathbb{R}}$ misurabile. Allora
    \begin{itemize}
        \item $\|f\|_{p} \geq 0$;
        \item $\|f\|_{p}=0$ se e solo se $f=0$ quasi ovunque (rispetto a $\mu$ );
        \item $\|c f\|_{p}=|c\n f\|_{p}$, per ogni $c \in \mathbb{R}$.
    \end{itemize}
\end{shadedTheorem}
\begin{proof}
    \begin{itemize}
        \item $\|f\|_p \ge 0$ per definizione.
        \item Dimostriamo l'implicazione \say{$\Rarr$} in quanto l'altra è banale, dividendoci in due casi: \begin{itemize}
            \item Se $p = + \infty$, si ha che $\|f\|_\infty = 0 \in m_f \implies |f|\le_\mu 0 \implies f =_\mu 0$.
            \item Se $p \in [1,+\infty[$ allora abbiamo: \[\left(\int_{X}|f|^p \d\mu\right)^\frac{1}{p} = 0 \implies \int_{X}|f|^p \d\mu = 0 \implies_{\text{Prop. \ref{prop: 2.4}}} \implies |f|^p =_\mu 0 \implies f =_\mu 0\]
        \end{itemize}
        \item Se $c=0$ la tesi è banale, dunque supponiamo $c\neq0$ e ci dividiamo in due casi: \begin{itemize}
            \item Se $p = + \infty$, abbiamo $M\in m_f \iff \|f\|_\infty \le_\mu M \iff |cf|\le_\mu |c|M \iff |c|M \in m_f$ e dunque $m_{cf} = |c|m_f = |c|[\|f\|_\infty,+\infty] = [\|cf\|_\infty,+\infty] \implies \|cf\|_\infty = |c\n f\|_\infty$.
            \item Nel caso in cui $p \in [1,+\infty[$, ci dividiamo ancora in due casi:\begin{itemize}
                \item Se $\|f\|_p = +\infty$ abbiamo \[\int_X |f|^p \d\mu = +\infty \qquad\underset{\text{Teorema \ref{thm: 7pt}.\ref{7pt: 3}}}{\implies}\qquad \int_{X} |cf|^p \d\mu = + \infty = |c\n f\|_p\]
                \item Se $\|f\|_p < +\infty$ abbiamo (sempre per il Teorema \ref{thm: 7pt}.\ref{7pt: 3}) che anche $|c|^p|f|^p$ risulta sommabile, quindi possiamo tirare fuori la costante dall'integrale e dalla radice ottenendo la tesi.\qedhere
            \end{itemize}
        \end{itemize}
    \end{itemize}
\end{proof}

% Teorema 3.2 (Disuguaglianza di Hölder $\left({ }^{*}\right)$ ). 
\begin{shadedTheorem}[$**|$\,Disuguaglianza di Hölder]\label{thm: 3.2 holder}
    Sia $(X, \mc{A}, \mu)$ uno spazio con misura e siano $f, g: X \rightarrow \overline{\mathbb{R}}$ due funzioni misurabili. Allora
    \[
    \int_{X}|f g| \d \mu \leq\|f\|_{p}\|g\|_{p^{\prime}}
    \]
    dove $p, p^{\prime} \in[1,+\infty]$ sono coniugati, cioè verificano una fra le seguenti ipotesi alternative:
    \begin{enumerate}
        \item $p=1$ e $p^{\prime}=+\infty$ (o viceversa);
        \item  $p, p^{\prime} \in(1,+\infty)$ e
    \end{enumerate}
    \[
    \frac{1}{p}+\frac{1}{p^{\prime}}=1
    \]
    In particolare, se $f \in {L}^{p}(X)$ e $g \in  L^{p'}(X)$ allora $f g \in L^{1}(X)$.
\end{shadedTheorem}
\begin{proof}
    Ricordiamo innanzitutto che per ogni funzione $f$ misurabile, anche la funzione $|f|^p$ lo è in quanto per il Teorema di Chiusura $|f|$ è misurabile, e la funzione $y\mapsto y^p$ è continua.
    \begin{itemize}
        \item \textbf{Caso 1:} Se $\|f\|_p = 0$ o $\|g\|_{p'} = 0$, allora $f = 0$ $\mu$-q.o. oppure $g = 0\ \mu$-q.o. rispettivamente, quindi $\int_X|fg|\d\mu = 0\leq 0$.
    \end{itemize}
    Possiamo pertanto supporre $\|f\|_p > 0$ e $\|g\|_{p'} > 0$
    \begin{itemize}
        \item \textbf{Caso 2:} Se $\|f\|_p = +\infty$ o $\|g\|_{p'} = +\infty$, allora $\int_X|fg|\d\mu$ può valere al più $+\infty$, quindi la tesi è banale. 
    \end{itemize}
    Possiamo pertanto supporre inoltre $\|f\|_p < +\infty$ e $\|g\|_{p'} <+\infty$.
    \begin{itemize}
        \item \textbf{Caso 3:} $p = +\infty$ e $p'=1$ (o viceversa). Osserviamo che vale la stima
            \[|f(x)g(x)| = |f(x)|\, |g(x)|\leq \|f\|_\infty\qquad \text{per }\mu\text{q.o.}x\in X\]
            e la funzione $\|f\|_\infty |g|$ è integrabile. Per monotonia dell'integrale, ricordando che $g$ è sommabile, vale quindi
            \[\int_X|fg|\d \mu \leq \int_X\|f\|_\infty |g(x)|\d \mu(x) = \|f\|_\infty \int_X|g(x)|\d \mu(x) = \|f\|_\infty \|g\|_1.\]
    \end{itemize}
    Possiamo infine supporre $p,p'\in ]1,+\infty[$.
    \begin{itemize}
        \item \textbf{Caso 4:} Supponiamo $0<\|f\|_p\|g\|_{p'}<+\infty$ e $p,p'\in ]1,+\infty[$. 
        
        \textbf{MEMO:} Per completare questa dimostrazione è necessario ricordare una proprietà relativa alla concavità del grafico della funzione logaritmo. In particolare per ogni $a,b\in \R_{>0}$ e per ogni $t\in [0,1]$ vale 
        \[\ln(a+t(b-a)) \geq \ln(a) + t(\ln b-\ln a) = \ln\left(a\left(\frac{b}{a}\right)^t\right) = \ln(a^{1-t}b^t),\]
        dove per riscrivere il secondo membro della disuguaglianza abbiamo applicato alcune proprietà elementari del logaritmo. In particolare per monotonia del logaritmo vale
        \[a^{1-t}b^t \leq a+t(b-a) = a(1-t)+bt.\]
        In particolare questa ultima disuguaglianza vale per ogni $a,b\in [0,+\infty[$.

        Tornando alle norme, ricordiamo che per le condizioni espresse inizialmente su $\|f\|_p$ e $\|g\|_{p'}$, per il Teorema \ref{thm: 7pt}.\ref{7pt: 2}, $|f|^p$ e 
        $|g|^p$ sono finite $\mu$-q.o. (i.e. in $X\setminus Z$, con $Z\subset X$ e $\mu(Z)=0$). Consideriamo quindi $x\in X\setminus Z$ e poniamo 
        \[a:=\left( \frac{|f|}{\|f\|_p} \right)^p\qquad \text{ e }\qquad b:= \left( \frac{|g|}{\|g\|_{p'}} \right)^{p'}, \]
        in particolare $a$ e $b$ sono positivi e finiti. Per la proprietà del logaritmo appena discussa con $t = 1/p'$, vale
        \[\left( \frac{|f|}{\|f\|_p} \right)^{p\frac{1}{p}}\cdot \left( \frac{|g|}{\|g\|_{p'}} \right)^{p'\frac{1}{p'}}\leq \frac{1}{p}\left( \frac{|f|}{\|f\|_p}  \right)^p + \frac{1}{p'}\left( \frac{|g|}{\|g\|_{p'}} \right)^{p'} \qquad \forall x \in X\setminus Z\]
        che può essere riscritta equivalentemente come 
        \[\frac{|fg|}{\|f\|_p\|g\|_{p'}}\leq \frac{1}{p}\frac{|f|^p}{\|f\|_p^p}+\frac{1}{p'}\frac{|g|^{p'}}{\|g\|_{p'}^{p'}}.\]
        Osserviamo quindi che tutte le funzioni in gioco sono integrabili, per cui per monotonia dell'integrale vale
        \[\frac{\int_X|fg|\d \mu}{\|f\|_p\|g\|_{p'}}\leq \frac{1}{p}\cancel{\frac{\int_X|f|^p\d \mu}{\|f\|_p^p}} + \frac{1}{p'}\cancel{\frac{\int_X|g|^p\d \mu}{\|g\|_{p'}^{p'}}} = \frac{1}{p}+\frac{1}{p'} = 1\]
        da cui, moltiplicando ad entrambi i membri per la quantità positiva $\|f\|_p\|g\|_{p'}$, segue la tesi.
    \end{itemize}
\end{proof}

% Teorema $\left.3.3{ }^{(* *}\right)$. 
\begin{shadedTheorem}[$**$] \label{thm: 3.3 minkovski}
    Siano $(X, \mc{A}, \mu)$ spazio con misura e $p \in[1,+\infty]$. Allora, per ogni coppia di funzioni misurabili $f, g: X \rightarrow \overline{\mathbb{R}}$ tale che $f+g$ sia ben definita (per esempio $f, g \in L^{p}(X)$ ), vale la disuguaglianza triangolare
    \[
    \|f+g\|_{p} \leq\|f\|_{p}+\|g\|_{p} \quad \text { (Disuguaglianza di Minkowski). }
    \]
\end{shadedTheorem}

\begin{proof}
    All'interno della dimostrazione, per semplicità, ci limiteremo ad indicare $\|\cdot\| = \|\cdot \|_p$.
    \begin{itemize}
        \item \textbf{Caso 1:} Se $\|f\| = +\infty$ o $\|g\| = +\infty$, la disuguaglianza è banale.
        \item \textbf{Caso 2:} Se $p = +\infty$, per $\mu$-q.o. $x\in X$ vale 
        \[|f(x)+g(x)|\leq |f(x)|+|g(x)|\leq \|f\|+\|g\|\]
        quindi $\|f\|+\|g\|$ è un maggiorante essenziale di $f+g$. Di conseguenza per definizione di norma infinito vale 
        \[\|f\| +\|g\|\geq \min \mc M_{f+g} = \|f+g\|.\]
        \item \textbf{Caso 3:} Se $p=1$, applicando la monotonia dell'integrale alla disuguaglianza triangolare per il valore assoluto otteniamo 
        \[\int_X|f+g|\d\mu \leq \int_X\left( |f|+|g| \right)\d\mu = \int_X|f|\d\mu+ \int_X|g|\d\mu = \|f\|+\|g\|\]
    \end{itemize}
    Possiamo quindi supporre $\|f\|,\|g\|<+\infty$ e $p\in ]1,+\infty[$.
    \begin{itemize}
        \item \textbf{Caso 4:} Supponiamo $\|f\|,\|g\|<+\infty$ e $p\in ]1,+\infty[$. Allora per definizione, ricordando che $p>1$, quindi $p-1>0$,
        \begin{align*}
            \|f+g\|^p & = \int_X|f+g|^p\d\mu \leq \int_X|f+g|^{p-1}(|f|+|g|)\d \mu = \int_X|f+g|^{p-1}|f|\d \mu + \int_X|f+g|^{p-1}|g|\d \mu = \\
            &\underset{\text{Hölder}} \leq \left( \int_X|f+g|^{(p-1)p'}\d\mu \right)^\frac{1}{p'}\|f\|_p + \left( \int_X|f+g|^{(p-1)p'}\d\mu \right)^\frac{1}{p'}\|g\|_p = \\
            & = \left( \int_X|f+g|^{p}\d\mu \right)^\frac{1}{p'}(\|f\|_p+\|g\|_p) = \|f+g\|_p^{p/p'}(\|f\|_p+\|g\|_p),
        \end{align*}
        dove abbiamos fruttato il fatto che, poiché $1/p+1/p' =1$, vale $p = (p-1)p'$. Dividendo entrambi i membri per $\|f+g\|_p^{p/p'}$ si ottiene la tesi:
        \[\|f+g\|^{p-p/p'} \leq (\|f\|_p+\|g\|_p)\qquad \text{dove }p-\frac{p}{p'} = p\left( 1-\frac{1}{p'}=p\frac{p} = 1. \right)\]
        Non abbiamo tuttavia la garanzia di poter effettuare tale divisione: dobbiamo infatti prima verificare che $\|f+g\|<+\infty$ (non dobbiamo verificare che sia positivo in quanto è negativo per definizione e nonzero per ipotesi).
        \[|f(x)+g(x)|^p\leq \left( |f(x)|+|g(x)| \right)^p\leq 2^p\left(|f(x)|\vee |g(x)|\right)^p\leq 2^p\left(|f(x)|^p+|g(x)|^p\right)\qquad \text{per }\mu\text{-q.o. }x\in X.\]
        Poiché $|f|^p$ e $|g|^p$ sono sommabili, per monotonia dell'integrale vale la stima 
        \[\|f+g\|^p = \int_X|f+g|^p\d \mu \leq 2^p (\|f\|^p + \|g\|^p)<+\infty.\qedhere\]
    \end{itemize}
\end{proof}

% Osservazione 3.2. 
\begin{oss}\label{oss: 3.2}
    Facendo il quoziente di $ L^{p}(X)$ rispetto alla relazione di equivalenza
    \[
    f \sim g \quad \text { se e solo se }\quad  f=g\quad \mu\text{-q.o.}
    \]
    si ottiene in modo naturale uno spazio vettoriale $\mca L^p:= L^p_{/\sim}$. Inoltre la funzione
    \[
     L^{p}(X)_{/\sim}\rightarrow[0,+\infty), \quad[f] \mapsto \n [f]\n_{p}:=\|f\|_{p}
    \]
    è una norma.
    \begin{itemize}
        \item Buona definizione: se $f \sim g$ allora $f=g$ q.o. e quindi $\|f\|_p = \|g\|_p$.
        \item $\n F\n_p \geq 0$: ovvio.
        \item $\n F\n_p = 0 \iff F = 0$: $\Larr$ ovvio, $\Rarr$ segue da Proposizione \ref{prop: 2.4}.
        \item Omogeneità: $\n cF\n_p = |c|\,\n F\n_p$ segue dal Teorema \ref{thm: 3.1 pre-norma Lp}.
        \item Disuguaglianza triangolare: $\n F + G\n_p \leq \n F\n_p + \n G\n_p$ segue dal Teorema \ref{thm: 3.3 minkovski}.
    \end{itemize}
\end{oss}
Per semplificare la notazione, è consuetudine denotare entrambi gli spazi con $L^{p}(X)$ e identificare $[f]$ con $f$ tutte le volte in cui la formula non dipende dalla scelta della funzione nella classe di equivalenza. Per questo motivo la norma (1.1) della classe di equivalenza di $f$ si indica ancora con $\|f\|_{p}$. Tuttavia noi cercheremo il più possibile di distinguere tali notazioni, continuando ad indicare il pre-spazio con $( L^{p}(X),\|\cdot\|_{p})$ e lo spazio quoziente con $(\mca L^{p}(X),\n\cdot\n_{p})$.

% Teorema 3.4 (Fisher-Riesz $(* * *)$ ). 
\begin{shadedTheorem}[$***$|\,Fisher-Riesz]\label{thm: 3.4 Fisher-Riesz}
    Siano $(X, \mc{A}, \mu)$ spazio con misura e $p \in[1,+\infty]$. Allora lo spazio vettoriale normato $\left(\mca L^{p}(X),\n\cdot\n_{p}\right)$ è uno spazio di Banach.
\end{shadedTheorem}
\begin{proof}
    \TODO\ Avrebbe dovuto farla Troncana...
\end{proof}

Dalla dimostrazione di Teorema \ref{thm: 3.4 Fisher-Riesz} segue subito il seguente risultato, che enunciamo senza ricorrere alla semplificazione notazionale descritta in Osservazione \ref{oss: 3.2}

% Proposizione $3.1\left(^{\circ}\right)$. 
\begin{proposition}[$\circ$]
    Siano $(X, \mc{A}, \mu)$ spazio con misura e $p \in[1,+\infty]$. Allora ogni successione $\left\{f_{j}\right\} \subset L^{p}(X)$ tale che $\left\{\left[f_{j}\right]\right\}$ converge in $\mca L^{p}(X)$ ha una sottosuccessione convergente $\mu$-q.o. a una funzione di $L^{p}(X)$.
\end{proposition}
\begin{proof}
    La dimostrazione ricalca esattamente quella del Teorema di Fisher-Riesz (\ref{thm: 3.4 Fisher-Riesz}) fino a $\square$.
\end{proof}

% OSSERVAZIONE 3.3. 
\begin{oss}
    In generale una successione convergente in $\mca L^{p}(X)$ non converge q.o., fatta eccezione per il caso $p=+\infty$.
\end{oss}
\begin{ex}[Tendina]
    \tcbsubtitle{Mostriamo ora un esempio di successione $\{f_j\}_{j}$ convergente in $\mca L^1(\R)$ ma non convergente $\mc L^1$-q.o. in $L^1(\R)$.}
    Procedendo come in figura, ovvero aumentando sempre di più le suddivisioni dell'intervallo $[0,1]$, e ponendo $f_j$ affinché la \say{tendina} copra sempre tutto l'intervallo, costruiamo una successione di funzionni.
    \begin{center}
        \begin{tikzpicture}[scale = 0.7]
            \draw[->](-2,0)--(5,0) node[below]{$x$};
            \draw[->](0,-0.5)--(0,4) node[left]{$y$};
            \draw[dashed] (3,0) node[below]{$1$}--(3,3)--(0,3)node[left]{$1$};
            \draw[dashed] (1.5,0)node[below]{$\frac{1}{2}$}--(1.5,3);
            \filldraw[ultra thick, examplecolor] (-2,0)--(0,0) circle (0.1);
            \filldraw[ultra thick, examplecolor] (0,3)--(1.5,3) circle (0.1);
            \filldraw[ultra thick, examplecolor] (1.5,0)--(5,0);
            \draw[thick, examplecolor, fill = white] (0,3) circle(0.1);
            \draw[thick, examplecolor, fill = white] (1.5,0) circle(0.1);
            \node[examplecolor] at (-1,0.5) {$\mc G_{f_1}$};
        \end{tikzpicture}\qquad\qquad
        \begin{tikzpicture}[scale = 0.7]
            \draw[->](-2,0)--(5,0) node[below]{$x$};
            \draw[->](0,-0.5)--(0,4) node[left]{$y$};
            \draw[dashed] (3,0) node[below]{$1$}--(3,3)--(0,3)node[left]{$1$};
            \draw[dashed] (1.5,0)node[below]{$\frac{1}{2}$}--(1.5,3);
            \filldraw[ultra thick, examplecolor] (-2,0)--(1.5,0) circle (0.1);
            \filldraw[ultra thick, examplecolor] (1.5,3)--(3,3) circle (0.1);
            \filldraw[ultra thick, examplecolor] (3,0)--(5,0);
            \draw[thick, examplecolor, fill = white] (1.5,3) circle(0.1);
            \draw[thick, examplecolor, fill = white] (3,0) circle(0.1);
            \node[examplecolor] at (-1,0.5) {$\mc G_{f_2}$};
        \end{tikzpicture}
    \end{center}
    \begin{center}
        \begin{tikzpicture}[scale = 0.7]
            \draw[->](-2,0)--(4,0) node[below]{$x$};
            \draw[->](0,-0.5)--(0,4) node[left]{$y$};
            \draw[dashed] (3,0) node[below]{$1$}--(3,3)--(0,3)node[left]{$1$};
            \draw[dashed] (1,0)node[below]{$\frac{1}{3}$}--(1,3);
            \draw[dashed] (2,0)node[below]{$\frac{2}{3}$}--(2,3);
            \filldraw[ultra thick, examplecolor] (-2,0)--(0,0) circle (0.1);
            \filldraw[ultra thick, examplecolor] (0,3)--(1,3) circle (0.1);
            \filldraw[ultra thick, examplecolor] (1,0)--(4,0);
            \draw[thick, examplecolor, fill = white] (0,3) circle(0.1);
            \draw[thick, examplecolor, fill = white] (1,0) circle(0.1);
            \node[examplecolor] at (-1,0.5) {$\mc G_{f_3}$};
        \end{tikzpicture}\qquad
        \begin{tikzpicture}[scale = 0.7]
            \draw[->](-2,0)--(4,0) node[below]{$x$};
            \draw[->](0,-0.5)--(0,4) node[left]{$y$};
            \draw[dashed] (3,0) node[below]{$1$}--(3,3)--(0,3)node[left]{$1$};
            \draw[dashed] (1,0)node[below]{$\frac{1}{3}$}--(1,3);
            \draw[dashed] (2,0)node[below]{$\frac{2}{3}$}--(2,3);
            \filldraw[ultra thick, examplecolor] (-2,0)--(1,0) circle (0.1);
            \filldraw[ultra thick, examplecolor] (1,3)--(2,3) circle (0.1);
            \filldraw[ultra thick, examplecolor] (2,0)--(4,0);
            \draw[thick, examplecolor, fill = white] (1,3) circle(0.1);
            \draw[thick, examplecolor, fill = white] (2,0) circle(0.1);
            \node[examplecolor] at (-1,0.5) {$\mc G_{f_4}$};
        \end{tikzpicture}\qquad
        \begin{tikzpicture}[scale = 0.7]
            \draw[->](-2,0)--(4,0) node[below]{$x$};
            \draw[->](0,-0.5)--(0,4) node[left]{$y$};
            \draw[dashed] (3,0) node[below]{$1$}--(3,3)--(0,3)node[left]{$1$};
            \draw[dashed] (1,0)node[below]{$\frac{1}{3}$}--(1,3);
            \draw[dashed] (2,0)node[below]{$\frac{2}{3}$}--(2,3);
            \filldraw[ultra thick, examplecolor] (-2,0)--(2,0) circle (0.1);
            \filldraw[ultra thick, examplecolor] (2,3)--(3,3) circle (0.1);
            \filldraw[ultra thick, examplecolor] (3,0)--(4,0);
            \draw[thick, examplecolor, fill = white] (2,3) circle(0.1);
            \draw[thick, examplecolor, fill = white] (3,0) circle(0.1);
            \node[examplecolor] at (-1,0.5) {$\mc G_{f_5}$};
        \end{tikzpicture}
    \end{center}
    Osserviamo che 
    \begin{align*}
        \|f_1\|_1 &= \int_\R |f_1|\d \mc L^1 = \frac{1}{2} \qquad \|f_2\|_1 = \int_\R |f_2|\d \mc L^1 = \frac{1}{2} \\
        \|f_3\|_1 &= \|f_4\_1 = \|f_5\|_1 = \frac{1}{3}\\
        \|f_6\|_1 &= \|f_7\|_1 = \|f_8\|_1 = \|f_9\|_1  = \frac{1}{4} \qquad \text{e così via.}
    \end{align*}
    In particolare $\lim\limits_{n\to +\infty}\|f_n\|_1 = 0$, ovvero (ponendo $f:\equiv 0$), $\lim\limits_{n\to +\infty}\|f_n-f\|_1 = 0$. Equivalentemente,  $\lim\limits_{n\to +\infty}\n [f_n]-[f]\n_1 = [0]$, ovvero $\{[f_n]\}_{n}\rightarrow [0]$ in $\mca L^1(\R)$. Tuttavia la convergenza il $L^1$ non è verificata in quanto per ogni $x\in [0,1]$, la successione $\{f_n(x)\}_n$ contiene infiniti $1$ e infiniti $0$, dunque non converge. Poiché $\mc L^1([0,1])=1>0$, la successione non converge q.o. in $L^1(\R)$. 
\end{ex}

\section[Serie di Fourier in uno spazio di Hilbert]{Serie di Fourier in uno spazio di Hilbert, un prontuario minimo (meno di così non si può...)}
Introdurremo di seguito qualche elemento di teoria degli spazi di Hilbert (il minimo indispensabile per la trattazione delle serie di Fourier che ci siamo dati come obiettivo della parte finale del corso).

% Proposizione $3.2\left(^{*}\right)$. 
\begin{proposition}[$*$]\label{prop: 3.2}
    Se $V$ è uno spazio vettoriale con un prodotto scalare $(\cdot, \cdot)$, allora la funzione
    \[
    v \mapsto\|v\|:=\scp[v, v]^\frac{1}{2}, \quad v \in V
    \]
    è una norma in $V$ ed è detta \say{la norma indotta dal prodotto scalare $\scp$}.
\end{proposition}
\begin{proof}Verifichiamo gli assiomi di norma:
    \begin{enumerate}[label=$\roman*)$]
        \item $\scp[v,v]^{\frac{1}{2}}\geq 0$ e $\scp[v,v]^{\frac{1}{2}}= 0\implies v=0$ in quanto il prodotto scalare è una forma bilineare definita positiva.
        \item (Omogeneità) Fissati $v\in \R^n$ e $\lambda \in \R$, per linearità del prodotto scalare si ha 
        \[\scp[\lambda v,\lambda v]^{\frac{1}{2}} = (\lambda ^2\scp[v,v])^{\frac{1}{2}} = |\lambda|\scp[v,v]^{\frac{1}{2}}\]
        \item (Disuguaglianza triangolare) Per la disuguaglianza di Cauchy-Schwarz si ha
        \[\scp[v+w,v+w] = \scp[v,v] + 2\scp[v,w] + \scp[w,w] \leq \scp[v,v] + 2\scp[v,v]^{\frac{1}{2}}\scp[w,w]^{\frac{1}{2}} + \scp[w,w] = (\scp[v,v]^{\frac{1}{2}} + \scp[w,w]^{\frac{1}{2}})^2\qedhere\]
    \end{enumerate}
\end{proof}

% Definizione 3.2. 
\begin{boxdef}
    Uno "spazio di Hilbert" è uno spazio di Banach in cui la norma è indotta da un prodotto scalare.
\end{boxdef}

% Osservazione 3.4. 
\begin{oss}
    Se $(X, \mc{A}, \mu)$ è uno spazio con misura, la norma $\n \cdot\n _{2}$ è indotta dal prodotto scalare
    \[
    \scp[F, G]_{2}:=\int_{X} f g \d \mu \quad\left(F, G \in L^{2}(X) / \sim\right)
    \]
    dove $f \in F$ e $g \in G$. Allora $\left(L^{2}(X) / \sim,\n \cdot\n _{2}\right)$ è uno spazio di Hilbert.
    \begin{itemize}
        \item Buona definizione: Per il Teorema di Chiusura \eqref{thm: chiusura}, $f,g\in L^2(X)\implies |fg|$ misurabile e quindi integrabile in quanto non negativo. Inoltre per la disuguaglianza di Hölder \eqref{thm: 3.2 holder} si ha che 
        \[\int_Xfg\d\mu\leq \|f\|_2\|g\|_2<+\infty\]
        quindi $|fg|$ è sommabile. In particolare per il Teorema \ref{thm: 7pt}.\ref{7pt: 6}, anche $fg$ è sommabile, i.e. $\int_Xfg\d\mu\in\R$.
        \item Simmetria: segue dalla definizione.
        \item Bilinearità: Osserviamo innanzitutto che basta verificare la linearità rispetto al primo fattore. Siano quindi $\alpha, \beta\in \R$, $F_1,F_2,G\in \mca L^2(X)$ e $f_1\in F_1, f_2\in F_2$ e $g\in G$. Allora per quanto detto sopra $f_1g$ e $f_2g$ sono sommabili, per cui per il Teorema \ref{thm: 7pt}.\ref{7pt: 3} si ha che $\alpha f_1g + \beta f_2g$ è sommabile e 
        \begin{multline*}\scp[\alpha F_1+\beta F_2,G]_2 = \int_X(\alpha f_1+\beta f_2)g\d\mu = \int_X(\alpha f_1g+\beta f_2 g)\d\mu =\\= \alpha \int_Xf_1g\d\mu+  \beta \int_Xf_2g\d\mu = \alpha\scp[F_1,G]_2+\beta \scp[F_2,g]_2\end{multline*}
        \item Definitezza positiva: Sia $F\in \mca L^2(X)$ tale che $\scp[F,F]_2 = 0$ e $f\in F$. Allora 
        \[0 = \scp[F,F]_2 = \int_Xf^2\d\mu \quad\iff\quad f^2 = 0\quad\mu\text{-q.o.}\quad\iff\quad f=0\quad \mu\text{-q.o.} \quad\iff\quad F = [0].\]
        Inoltre per monotonia dell'integrale si ha che $\scp[F,F]_2\geq 0\ \forall F\in \mca L^2(X)$.
    \end{itemize}
    Rimane da verificare che $\scp_2$ induce effettivamente la norma $\n \cdot\n_2$. Siano $F\in \mca L^2(X)$ e $f\in F$, allora 
    \[\scp[F,F]_2^{\frac{1}{2}} = \left(\int_Xf^2\d\mu\right)^\frac{1}{2} = \|f\|_2 = \n F\n_2.\]
\end{oss}

% Definizione 3.3. 
\begin{boxdef}
    Sia $H$ uno spazio di Hilbert. Allora un sottoinsieme $F$ di $H$ è detto \say{famiglia ortonormale (in $H$)} se per ogni $x, y \in F$ si ha
    \[
    \scp[x, y]= \begin{cases}0 & \text { se } x \neq y \\ 1 & \text { se } x=y\end{cases}
    \]
    Una famiglia ortonormale $F \subset H$ si dice \say{completa (in $H$)} se soddisfa la seguente condizione: se $h \in H$ è tale che $\scp[h, x]=0$ per ogni $x \in F$ allora si ha $h=0$.
\end{boxdef}

% OssERvazione 3.5. 
\begin{oss}
    
    Sia $B:=\left\{u_{1}, \ldots, u_{n}\right\}$ una base ortonormale di $\mathbb{R}^{n}$ (dotato del canonico prodotto scalare Euclideo). Allora:
    \begin{enumerate}
        \item La base $B$ è una famiglia ortonormale completa;
        \item Se $F \subset B$ e $F \neq B$, allora $F$ è una famiglia ortonormale ma non è completa.
    \end{enumerate}
\end{oss}

Come applicazione del Lemma di Zorn si prova facilmente l'esistenza di famiglie ortonormali complete, e.g. [5, Theorem 8.44].

% Teorema 3.5. 
\begin{shadedTheorem}
    Ogni spazio di Hilbert $H$ contiene una famiglia ortonormale completa. Se $H$ è separabile allora ogni famiglia ortonormale (in particolare, ogni famiglia ortonormale completa) è numerabile.
\end{shadedTheorem}

% TEOREma $3.6\left(^{* * *}\right)$. 
\begin{shadedTheorem}[$***$\,|\, Serie di Fourier astratta]\label{thm: 3.6 serie di Fourier astratta}
    Sia $F=\left\{u_{1}, u_{2}, \ldots\right\}$ una famiglia ortonormale numerabile in uno spazio di Hilbert $H$. Valgono allora $i$ seguenti fatti:
    \begin{enumerate}
        \item Se $h \in H$ e $c_{1}, c_{2}, \ldots$ sono numeri reali, si ha
        \[
        \left\|h-\sum_{i=1}^{m}\scp[h, u_{i}] u_{i}\right\| \leq\left\|h-\sum_{i=1}^{m} c_{i} u_{i}\right\|
        \]    
        per ogni $m \geq 1$. Inoltre l'uguaglianza vale se e solo se $c_{i}=\scp[h, u_{i}]$, per $i=$ $1, \ldots, m$;
        \item Per ogni $h \in H$ si $h a \sum_{i}\scp[h, u_{i}]^{2} \leq\|h\|^{2}$ (disuguaglianza di Bessel);
        \item Siano $c_{1}, c_{2}, \ldots$ numeri reali. Allora $\sum_{i} c_{i} u_{i}$ converge in $H$ se $e$ soltanto se $\sum_{i} c_{i}^{2}<+\infty$. In particolare, per ogni $h \in H$, la serie $\sum_{i}\scp[h, u_{i}] u_{i}$ converge in $H$. Essa è detta "serie di Fourier di h (relativa alla famiglia $F$)";  
        \item Se $F$ è completa, per ogni $h \in H$ si ha $\sum_{i}\scp[h, u_{i}] u_{i}=h$. In particolare la serie $\sum_{i}\scp[h, u_{i}]u_{i}$ converge incondizionatamente, cioè la sua somma non dipende dall'ordine dei suoi addendi.
    \end{enumerate}
\end{shadedTheorem}
\begin{proof}
    \begin{enumerate}
        \item Applicando la definizione di norma indotta dal prodotto scalare e completando un quadrato si ha
        \begin{align*}
            \left\|h-\sum_{i=1}^m c_iu_i\right\| ^2 &= \scp[h-\sum_{i=1}^m c_iu_i,h-\sum_{j=1}^m c_ju_i] = \scp[h,h]-2\sum_{i=1}^m c_i\scp[u_i,h] + \sum_{i,j=1}^mc_ic_j\underbrace{\scp[u_i,u_j]}_{\delta_{ij}} = \\
            & = \|h\|^2 - 2\sum_{i=1}^m c_i\scp[u_i,h] + \sum_{i=1}^m c_i^2 = 
            \|h\|^2 + \underbrace{\sum_{i=1}^m (c_i - \scp[u_i,h])^2 }_{\geq 0}- \sum_{i=1}^m \scp[h,u_i]^2 \geq\\ & \underset{(\bullet)}{\geq} 
            \|h\|^2 - \sum_{i=1}^m \scp[h,u_i]^2 \underset{(*)}{=} \left\|h^2 - \sum_{i=1}^m \scp[h,u_i]^2\right\|^2
        \end{align*}
        dove l'uguaglianza $(*)$ è data dal fatto che per quanto appena detto \[\left\|h-\sum_{i=1}^m c_iu_i\right\| ^2  = \|h\|^2 + {\sum\limits_{i=1}^m (c_i - \scp[u_i,h])^2 }- \sum_{i=1}^m \scp[h,u_i]^2.\]
        In particolare ciò vale per $c_i = \scp[u_i,h]$, ovvero quando $c_i - \scp[u_i,h] = 0$, per cui otteniamo 
        \[\left\|h-\sum_{i=1}^m \scp[h,u_i]u_i\right\| ^2  = \|h\|^2 - \sum_{i=1}^m \scp[h,u_i]^2.\tag{$*$}\]
        Abbiamo quindi provato la disuguaglianza. Affnché valga l'uguaglianza, essa deve valere in $(\bullet)$, ovvero deve essere $\sum_{i=1}^m (c_i - \scp[u_i,h])^2 = 0$, il che (in quanto somma di quadrati) implica $c_i = \scp[u_i,h]$.
        \item Per quanto provato in $(1)$, 
        \[\|h\|^2 - \sum_{i=1}^m \scp[h,u_i]^2=\left\|h-\sum_{i=1}^m \scp[h,u_i]u_i\right\| ^2 \geq 0\qquad \forall m\in\N^*,\]
        quindi per ogni $m$ vale anche 
        \[\sum_{i=1}^m \scp[h,u_i]^2\leq \|h\|^2\qquad \overset{m\to +\infty}\implies \qquad \sum_{i=1}^{+\infty} \scp[h,u_i]^2\leq \|h\|^2.\]
        \item Poniamo $s_n = \sum_{i=1}^n c_iu_i$ e verifichiamo che è di Cauchy. Posto inoltre $\sigma_n = \sum_{i=1}^n c_i^2$, si ha
        \[\|s_{n+k}-s_n\|^2 = \left\|\sum_{i=n+1}^{n+k}c_iu_i\right\|^2 = \sum_{i,j=n+1}^{m}c_ic_j\underbrace{\scp[u_i,u_j]}_{\delta_{ij}} = \sum_{i=n+1}^{n+k}c_i^2=\sigma_{n+k}-\sigma_n\]
        quindi $s_n$ è di Cauchy se e solo se lo è $\sigma_n$. Poiché entrambe le funzioni sono in spazi di Banach, rispettivamente $H$ e $\R$, $\{s_n\}_{n}$ converge se e solo se $\{\sigma_n\}_{n}$ converge, ovvero
        \[\sum_{i=1}^{+\infty}c_iu_i \text{ converge}\qquad \iff \qquad \sum_{i=1}^{+\infty}c_i^2<+\infty\]
        \item Poiché la famiglia $\{u_i\}_{i}$ è completa, proviamo equivalentemente la \tesi{\scp[\sum_{i=1}^n\scp[h,u_i]-h,u_j] = 0\ \forall j}\\
        Poniamo $s_n = \sum_{i=1}^n \scp[h,u_i]u_i$ e $s = \sum_{i=1}^{+\infty} \scp[h,u_i]u_i$ (quindi per il punto precedente vale $\|s_n-s\|\underset{n\to +\infty}\longrightarrow 0$). 
        \begin{align*}
            |\scp[s-h,u_j]|&=|\scp[s,u_j]-\scp[h,u_j]| = |\scp[s-s_n,u_j]+\scp[s_n,u_j] - \scp[h,u_j]| \leq \\&\leq \underbrace{|\scp[s-s_n,u_j]|}_{\substack{\leq \|s-s_n\|\cancel{\|u_j\|}^1\\\text{Cauchy-Schwarz}}} +|\underbrace{\scp[s_n,u_j]}_{(\circ)}-\scp[h,u_j]|\leq \|s-s_n\|+\cancel{\|\scp[h,u_j]-\scp[h,u_j]\|}
        \end{align*}
        dove abbiamo sfruttato il fatto che
        \[\scp[s_n,u_j]=\sum_{i=1}^n\scp[h,u_i]\scp[u_i,u_j]=\scp[h,u_j]\tag{$\circ$}.\]
        Abbiamo quindi che $\forall n>j$ vale
        \[|\scp[s-h,u_j]|\leq \|s-s_n\|\underset{n\to +\infty}{\longrightarrow}0\]
        da cui $|\scp[s-h,u_j]|=0$ per ogni $j$, ovvero $\scp[s-h,u_j]=0$ per ogni $j$. Poiché $\{u_j\}_{j}$ è completa, $s-h=0$, ovvero $s=h$.

        Rimane da verificare la convergenza incondizionata. Basta tuttavia osservare che una qualsiasi permutazione di una famiglia ortonormale completa è ancora una famiglia ortonormale completa. Sia infatti $I$ la famiglia che indicizza gli $\{u_i\}_{i\in I}$ e $\phi : I\overset{\sim}{\to} I$ biiettiva. Allora anche $\{u_{\phi(i)}\}_{i}$ è una famiglia ortonormale completa. Di conseguenza per ogni $h\in H$ si ha che
        \[\begin{cases}
            h = \sum_{i\in I}\scp[h,u_{\phi(i)}]u_{\phi(i)}\\
            h = \sum_{i\in I}\scp[h,u_i]u_i
        \end{cases}\]
        quindi le due serie sono uguali e in particolare hano lo stesso carattere.\qedhere
    \end{enumerate}
\end{proof}

\paragraph{Teoria $L^2$ delle serie di Fourier.} Una importante applicazione della teoria precedente è la cosiddetta "teoria $L^{2}$ delle serie di Fourier" che qui descriveremo sommariamente. Prima di tutto si considera lo spazio con misura $\left([-\pi, \pi], \mc{M}_{\varphi},\left.\varphi\right|_{\mc{M}_{\varphi}}\right)$ indotto dalla misura esterna $\varphi:=\left.\mc{L}^{1}\right|_{\P([-\pi, \pi])}: \P([-\pi, \pi]) \rightarrow [0,+\infty]$ e quindi il corrispondente spazio Hilbert $\left(\mca{L}^{2}([-\pi, \pi]),\n\cdot\n_{2}\right)$, cfr. Osservazione 3.4. Coerentemente a quanto spiegato in Osservazione 3.2, quest'ultimo verrà indicato più semplicemente con la notazione "pre", cioè $\left(L^{2}([-\pi, \pi]),\|\cdot\|_{2}\right)$. Inoltre si scriverà spesso $L^{2}(-\pi, \pi)$ in luogo di $L^{2}([-\pi, \pi])$.

Utilizzando le formule di addizione e le formule di Werner, è facile provare che il \say{\textbf{sistema trigonometrico}}
\begin{equation}
F:=\left\{\frac{1}{\sqrt{2 \pi}}\right\} \cup\left\{\frac{\cos (n x)}{\sqrt{\pi}}, \frac{\sin (n x)}{\sqrt{\pi}}\right\}_{n=1}^{\infty}\label{eq: sistema trigonometrico}
\end{equation}
è una famiglia ortonormale in $L^{2}(-\pi, \pi)$. Osserviamo che se $f \in L^{2}(-\pi, \pi)$ allora le somme parziali (di ordine dispari) della serie di Fourier di $f$ relativa al sistema trigonometrico \eqref{eq: sistema trigonometrico} sono date da

\[
S_{2 N+1}(x)=\frac{a_{0}}{2}+\sum_{n=1}^{N}\left[a_{n} \cos (n x)+b_{n} \sin (n x)\right]
\]
dove
\[
    a_{n}=a_{n}(f):=\frac{1}{\pi} \int_{[-\pi, \pi]} f(t) \cos (n t)\ d \mc{L}^{1}(t) \quad(n=0,1,2, \ldots)
\]
e
\[
b_{n}=b_{n}(f):=\frac{1}{\pi} \int_{[-\pi, \pi]} f(t) \sin (n t)\d \mc{L}^{1}(t) \quad(n=1,2, \ldots)
\]

Dalle affermazioni (3) e (2) di Teorema 3.6, rispettivamente, si ottiene allora che se $f \in$ $L^{2}(-\pi, \pi)$ :

\begin{itemize}
    \item Vale la disuguaglianza di Bessel:
    \begin{equation}
    \frac{1}{2 \pi}(f, 1)_{2}^{2}+\frac{1}{\pi} \sum_{n=1}^{+\infty}\left[(f, \cos (n x))_{2}^{2}+(f, \sin (n x))_{2}^{2}\right] \leq\|f\|_{2}^{2}
    \end{equation}
    \item La serie di Fourier di $f$ relativa al sistema trigonometrico (2.1)
    \begin{equation}
    \frac{a_{0}}{2}+\sum_{n=1}^{+\infty}\left[a_{n} \cos (n x)+b_{n} \sin (n x)\right]
    \end{equation}
    converge in $L^{2}(-\pi, \pi)$. Ciò significa che esiste $g \in L^{2}(-\pi, \pi)$ tale che
    \[
        \lim_{N \rightarrow+\infty}\left\|g-S_{2 N+1}\right\|_{2}=0
    \]
\end{itemize}

%Osservazione 3.6.
\begin{oss}
    Dalla disuguaglianza di Bessel (2.2) segue subito che per ogni $f \in$ $L^{2}(-\pi, \pi)$ si ha
    \[
    \lim _{n \rightarrow+\infty} \int_{[-\pi, \pi]} f(t) \cos (n t) \d \mc{L}^{1}(t)=\lim _{n \rightarrow+\infty} \int_{[-\pi, \pi]} f(t) \sin (n t) \d \mc{L}^{1}(t)=0
    \]
\end{oss}

In realtà l'insieme $F$ definito in (2.1) è una famiglia ortonormale completa. Questo fatto si può dimostrare utilizzando i seguenti due risultati di approssimazione. Il primo si ottiene per regolarizzazione mediante prodotto di convoluzione [1, Corollario IV.23], mentre il secondo è una conseguenza del Teorema di Stone-Weierstrass $[\mathbf{6}, \mathbf{1 7}]$.

% Teorema 3.7. 
\begin{shadedTheorem}
    Lo spazio vettoriale $C_{c}(-\pi, \pi)$ è denso in $\left(L^{2}(-\pi, \pi),\|\cdot\|_{2}\right)$.
\end{shadedTheorem}

% TEOREma 3.8. 
\begin{shadedTheorem}
    Sia $\varphi \in C(K)$, con $K$ un sottoinsieme compatto di $\mathbb{R}^{n}$. Allora per ogni $\varepsilon>0$ esiste un polinomio $P: \mathbb{R}^{n} \rightarrow \mathbb{R}$ tale che $\sup _{K}|\varphi-P| \leq \varepsilon$.
\end{shadedTheorem}
Da (4) di Teorema 3.6 segue allora subito il seguente risultato.

% Corollario $3.1\left(^{\circ}\right)$. 
\begin{corollary}[$\circ$]
    Per ogni $f \in L^{2}(-\pi, \pi)$, la serie di Fourier (2.3) converge incondizionatamente a $f$ in $\left(L^{2}(-\pi, \pi),\|\cdot\|_{2}\right)$.
\end{corollary}
Combinando Corollario 3.1 e Proposizione 3.1, otteniamo infine:

%Corollario $3.2\left({ }^{\circ}\right)$. 
\begin{corollary}[$\circ$]
    Se $f \in L^{2}(-\pi, \pi)$, allora esiste una sottosuccessione di
    \begin{equation}
    S_{2 N+1}(x)=\frac{a_{0}}{2}+\sum_{n=1}^{N}\left[a_{n} \cos (n x)+b_{n} \sin (n x)\right] \tag{2.4}
    \end{equation}
    che converge puntualmente quasi ovunque in $[-\pi, \pi]$ alla funzione $f$.
\end{corollary}

% OSSERVAZIONE 3.7. 
\begin{oss}
    Nel 1915 Lusin pose la questione della convergenza quasi ovunque di "tutta" la successione (2.4). La risposta affermativa venne oltre cinquant'anni dopo, in un profondo lavoro di Lennart Carleson $[\mathbf{2}]$.
\end{oss}

\section{Convergenza puntuale della serie di Fourier per una funzione regolare a tratti}
Per questa ultima parte del corso, la bibliografia di riferimento è il secondo capitolo dell'opera $[\mathbf{9}]$.

% Definizione 3.4. 
\begin{boxdef}
    Sia data una funzione $2 \pi$-periodica $f: \mathbb{R} \rightarrow \mathbb{R}$. Allora:
    \begin{enumerate}[label=$\roman*)$]
        \item $f$ è detta "continua a tratti" se
        \begin{itemize}
            \item l'insieme $D$ dei punti di discontinuità di $f$ in $[-\pi, \pi)$ è vuoto o finito
            \item per ogni $x_{0} \in D$ esistono finiti i limiti sinistro e destro
        \end{itemize}
        \[f\left(x_{0}-0\right):=\lim _{x \rightarrow x_{0}-} f(x), \quad f\left(x_{0}+0\right):=\lim _{x \rightarrow x_{0}+} f(x) ;\]  
        \item $f$ è detta "regolare a tratti" se:
        \begin{itemize}
            \item è continua a tratti secondo la descrizione data in ($i$);
            \item esiste un sottoinsieme finito $E$ di $[-\pi, \pi)$ tale che $E \supset D$ e $f$ ha derivata continua e limitata in $[-\pi, \pi) \backslash E$.
        \end{itemize}
    \end{enumerate}
\end{boxdef}

Vale il seguente risultato sulla convergenza puntuale e sulla convergenza puntuale uniforme.

% Teorema 3.9. 
\begin{shadedTheorem}
    Sia $f: \mathbb{R} \rightarrow \mathbb{R}$ una funzione $2 \pi$-periodica e regolare a tratti. Allora:
    \begin{enumerate}
        \item Per ogni $x \in \mathbb{R}$, la serie di Fourier di $f$ in $x$ è uguale a
        \[
        \frac{f(x-0)+f(x+0)}{2}
        \]
        In particolare, se $f$ è continua in $x$, allora la serie di Fourier di $f$ in $x$ è uguale a $f(x)$.
        \item La serie di Fourier di $f$ converge uniformemente a $f$ in ogni intervallo chiuso in cui $f$ è continua;
        \item Se $f$ è continua, la sua serie di Fourier converge uniformemente a $f$.
    \end{enumerate}
\end{shadedTheorem}

% Osservazione 3.8. 
\begin{oss}
    Consideriamo la serie di Fourier
    \begin{equation}
    \frac{a_{0}}{2}+\sum_{n=1}^{\infty}\left(a_{n} \cos n x+b_{n} \sin n x\right) \quad(x \in \mathbb{R})
    \end{equation}
    di una funzione $2 \pi$-periodica e regolare a tratti $f: \mathbb{R} \rightarrow \mathbb{R}$. Ricordando che
    \[
    a_{n}=\frac{1}{\pi} \int_{-\pi}^{\pi} f(t) \cos n t d t \quad(n=0,1,2, \ldots)
    \]
    e
    \[
    b_{n}=\frac{1}{\pi} \int_{-\pi}^{\pi} f(t) \sin n t d t \quad(n=1,2, \ldots),
    \]
    si vede subito che:
    \begin{enumerate}
        \item Se $f$ è dispari, la (3.1) è una "serie di soli seni", cioè $a_{n}=0$ per ogni $n$ e si ha
        \[
        b_{n}=\frac{2}{\pi} \int_{0}^{\pi} f(t) \sin n t d t \quad(n=1,2, \ldots)
        \]
        \item Se $f$ è pari, la (3.1) è una "serie di soli coseni", cioè $b_{n}=0$ per ogni $n$ e si ha
        \[
        a_{n}=\frac{2}{\pi} \int_{0}^{\pi} f(t) \cos n t d t \quad(n=0,1,2, \ldots) .
        \]
    \end{enumerate}
\end{oss}

% OssERvaZione 3.9. 
\begin{oss}
    Naturalmente la teoria della serie di Fourier che abbiamo presentato per le funzioni $2 \pi$-periodiche può essere "riformulata" per le funzioni $2 L$-periodiche: basta rifare tutto applicando i risultati astratti allo spazio di Hilbert $H:=\left(\mc{L}^{2}([-L, L]), \n  \cdot\n _{2} \right)$, dove lo spazio con misura considerato stavolta è quello indotto da $\left.\mc{L}^{1}\right|_{2^{[-L, L]}}$. Per cominciare, il sistema trigonometrico da utilizzare in questo caso è
    
    \[
        \left\{\frac{1}{\sqrt{2 L}}\right\} \cup\left\{\frac{1}{\sqrt{L}} \cos \frac{\pi}{L} n x, \frac{1}{\sqrt{L}} \sin \frac{\pi}{L} n x\right\}_{n=1}^{\infty}
    \]
\end{oss}

Eccetera.

%Osservazione 3.10. 
\begin{oss}
    Dalle serie di Fourier di può ottenere una funzione continua in $\mathbb{R}$ (e $2 \pi$-periodica) che non è derivabile in alcun punto, si veda per esempio $[\mathbf{1 0}$, Cap. 2 , Sez. $6]$.
\end{oss}

Esempi.