\section{Serie notevoli}
\begin{itemize}
    \item \textbf{Serie geometrica:} $\sum_{n=0}^{+\infty}q^n$
    \begin{itemize}
        \item Converge se: $|q|<1$ con somma $\frac{1}{1-q}$
        \item Diverge positivamenete se: $q\geq 1$
        \item Indeterminata se:  $q\leq -1$
    \end{itemize}
    \item \textbf{Serie armonica generalizzata:} $\sum_{n=0}^{+\infty}\frac{1}{n^\alpha(\log n)^\beta}$
    \begin{itemize}
        \item Converge se: $\alpha>1, \forall \beta$ oppure $\alpha =1, \beta >1$
        \item Diverge positivamenete se: $\alpha< 1, \forall \beta$ oppure $\alpha =1, \beta \leq 1$
    \end{itemize}
    \item \textbf{Serie telescopiche:} $\sum_{n=0}^{+\infty}a_n=\sum_{n=0}^{+\infty}(b_n-b_{n+1})$\\
    Risulta quindi che la successione delle somme parziali è $s_n=b_0-b_{n+1}$, per cui il carattere della serie dipende da $\lim_{n\to+\infty}b_{n+1}$
\end{itemize}
\subsection{Somme finite}
\everymath{\displaystyle}
\begin{tasks}[label=\textbullet](2)
    \task $\sum_{k=1}^n k = \frac{n(n+1)}{2}$
    \task $\sum_{k=1}^n k^2 = \frac{n(n+1)(2n+1)}{6} = \frac{n^3}{3} + \frac{n^2}{2} + \frac{n}{6}  $
    \task $\sum_{k=1}^n k^3 = \left(\frac{n(n+1)}{2}\right)^2 = \frac{n^4}{4} + \frac{n^3}{2} + \frac{n^2}{4} = \left(\sum_{k=1}^n k\right)^2$
    \task $\sum_{k=m}^n x^k=\frac{x^{n+1}-x^m}{x-1} \quad\text{con } |x|< 1$
    \task $\sum_{i=1}^\infty i x^i = \frac{x}{(1-x)^2}\quad\text{con } |x|< 1$
    \task $ \sum_{k=0}^n \binom{n}{k} x^{k} = (1+x)^{n} $
\end{tasks}
\subsection{Serie}
\begin{tasks}[label=\textbullet](3)
    \task  $\sum^{\infty}_{k=1} \frac{1}{k^2}=\frac{\pi^2}{6}$
    \task $\sum^{\infty}_{k=1} \frac{1}{k^4}=\frac{\pi^4}{90}$
    \task $\sum^{\infty}_{k=1} \frac{1}{k^6}=\frac{\pi^6}{945}$
    \task $\sum_{k=0}^\infty x^{2k}= \frac{1}{1-x^2}$
    \task $\sum_{i=1}^\infty i x^i = \frac{x}{(1-x)^2}$
    \task $\sum_{n=0}^\infty (-1)^n x^n = \frac{1}{1+x}$
    \task $ \sum_{n=0}^\infty {\alpha \choose n} x^n = (1+x)^\alpha$
    \task $\sum_{i=0}^\infty {i+n \choose i} x^i = \frac{1}{(1-x)^{n+1}}$
    \task $\sum_{i=0}^\infty {2i \choose i} x^i = \frac{1}{\sqrt{1-4x}}$
    \task $\sum_{n=b+1}^{\infty} \frac{b}{n^2 - b^2} = \sum_{n=1}^{2b} \frac{1}{2n}$
\end{tasks}

\section{Criteri di convergenza}
\begin{shadedTheorem}[Condizione necessaria di convergenza]
        Se $\sum_na_n$ è convergente, allora $\lim_{n\to+\infty}a_n=0$
\end{shadedTheorem}
\subsection{Criteri di convergenza per serie a segno costante}
\begin{shadedTheorem}[Confronto]
    Siano $(a_n)_n$ e $(b_n)_n$ due successioni di numeri reali tali che 
    \[0\leq a_n\leq b_n~~~~~~\forall n\in \N\]
    allora
    \begin{enumerate}[label=\roman*\:\textnormal{)},itemindent=*]
        \item $\sum_{n=0}^{+\infty}b_n<+\infty~~\implies~~\sum_{n=0}^{+\infty}a_n<+\infty$
        \item $\sum_{n=0}^{+\infty}a_n=+\infty~~\implies~~\sum_{n=0}^{+\infty}b_n=+\infty$
    \end{enumerate}
\end{shadedTheorem}

\begin{shadedTheorem}[Confronto asintotico]
    Siano $(a_n)_n$ e $(b_n)_n$ due successioni di numeri reali positivi (almeno da un certo $\bar{n}$ in poi) tali che 
    \[\lim_{n\to +\infty}\frac{a_n}{b_n}=l\in \:]0,+\infty[~~~~\left( a_n\asymp b_n \text{ per } n\to+\infty \right)\]
    Allora $\sum_{n=0}^{+\infty}a_n$ e $\sum_{n=0}^{+\infty}b_n$ hanno lo stesso carattere.
\end{shadedTheorem}

\begin{shadedTheorem}[Criterio della radice $n$-esima]
    Sia $(a_n)_n$ una successione di numeri reali non negativi. Se esiste 
    \[\lim_{n\to +\infty }\sqrt[n]{a_n}=l\in \:]0,+\infty[\]
    e 
    \begin{itemize}
        \item $l<1$, allora $\sum_{n=0}^{+\infty}a_n<+\infty$
        \item $l>1$, allora $\sum_{n=0}^{+\infty}a_n=+\infty$
    \end{itemize}
\end{shadedTheorem}
Se $l=1$ non abbiamo informazioni circa il carattere della serie.
\begin{shadedTheorem}[Criterio del rapporto]
    Sia $(a_n)_n$ una successione di numeri reali positivi. Se esiste 
    \[\lim_{n\to +\infty }\frac{a_{n+1}}{a_n}=l\in \:]0,+\infty[\]
    e 
    \begin{itemize}
        \item $l<1$, allora $\sum_{n=0}^{+\infty}a_n<+\infty$
        \item $l>1$, allora $\sum_{n=0}^{+\infty}a_n=+\infty$
    \end{itemize}
\end{shadedTheorem}
Se $l=1$ non abbiamo informazioni circa il carattere della serie.
\subsection{Serie numeriche a segno qualsiasi}
\begin{shadedTheorem}[Criterio dell'assoluta convergenza]
    Sia $(a_n)_n$ una successione di numeri reali positivi. Se la serie $\sum_{n=0}^{+\infty}a_n$ è assolutamente convergente, allora è (semplicemente) convergente, e si ha 
    \[\left|\sum_{n=0}^{+\infty}a_n\right|\leq \sum_{n=0}^{+\infty}\left|a_n\right|\]
\end{shadedTheorem}
\subsection{Serie numeriche a segni alterni}
\begin{shadedTheorem}[Criterio di Leibnitz]
    Sia $(a_n)_n$ una successione di numeri reali tali che :
    \begin{enumerate}[label=\roman*\:\textnormal{)},itemindent=*]
        \item $a_n\geq 0 ~~~~\forall n\in \N$
        \item $(a_n)_n$ decrescente $\forall n\geq\bar n$
        \item $\lim_{n\to +\infty} a_n=0$
    \end{enumerate}
    Allora la serie $\sum_{n=0}^{+\infty}(-1)^na_n$ è convergente.
\end{shadedTheorem}

\subsection{Integrali e serie}
\begin{shadedTheorem}[Criterio integrale per le serie a termini positivi]
    Sia $f:[0;+\infty[\to [0;+\infty[$ decrescente. Poniamo $a_n=f(n)~~\forall n \in \N$
    Allora
    \[\sum_{n=0}^{+\infty}<+\infty~~\Harr~~\int_0^{+\infty}f(x)\d x<+\infty\]
    Inoltre
    \[\sum_{n=1}^{+\infty}a_n\leq \int_0^{+\infty}f(x)\d x\leq \sum_{n=0}^{+\infty}a_n\]
    (vale anche da un certo $\bar n$ in poi)
\end{shadedTheorem}

\section{Sviluppi di Taylor}
Centrati in $x_0=0$
\begin{itemize}
    \item $e^x=1+x+\frac{x^2}{2!}+\frac{x^3}{3!}+\dots+\frac{x^n}{n!}+o(x^n)$
    \item $\log(1+x)=x-\frac{x^2}{2}+\frac{x^3}{3}-\dots+\frac{(-1)^n}{n}x^n+o(x^n)$
    \item $\sen x = x-\frac{x^3}{3!}+\frac{x^5}{5!}-\dots+\frac{(-1)^n}{(2n+1)!}x^{2n+1}$
    \item $\cos x = 1-\frac{x^2}{2}+\frac{x^4}{4!}-\dots+\frac{(-1)^n}{(2n)!}x^{2n}+o\left(x^{2n+1}\right)$
    \item $\senh x = \frac{e^x-e^{-x}}{2}=\sum_{k=0}^n\frac{x^{2k+1}}{(2k+1)!}+o(x^{2n+2})$
    \item $\cosh x = \frac{e^x+e^{-x}}{2}=\sum_{k=0}^n\frac{x^{2k}}{(2k)!}+o(x^{2n+1})$
    \item $\arctg x = x-\frac{x^3}{3}+\frac{x^5}{5}-\dots+\frac{(-1)^nx^{2n+1}}{2n+1}+o(x^{2n+2})$
    \item $\tg x = x+\frac{x^3}{3}+\frac{2}{15}x^5+o(x^6)$
    \item $(1+x)^\alpha=1+\alpha x+\frac{\alpha(\alpha-1)}{2!}x^2+\dots+\frac{\alpha(\alpha-1)\cdot \ldots\cdot (\alpha-(n-1))}{n!}x^n+o(x^n)=\sum_{k=0}^n
    \binom{\alpha}{k}x^k$
\end{itemize}
\subsubsection*{Casi particolari}
\begin{itemize}
    \item[$\Rightarrow$] $\frac{1}{1+x}=1-x+x^2-x^3+\dots+(-1)^nx^n+o(x^n)$
    \item[$\Rightarrow$] $\frac{1}{1-x}=1+x+x^2+x^3+\dots+x^n+o(x^n)$
    \item[$\Rightarrow$] $\sqrt{1+x}=1+\frac{x}{2}-\frac{x^2}{8}+\frac{x^3}{16}+o(x^3)$
\end{itemize}