\section{Equazioni differenziali ordinarie}
Il seguente teorema di punto fisso (di Banach-Caccioppoli) sarà essenziale per provare il successivo teorema di esistenza di soluzioni per il problema di Cauchy.
% Teorema $5.1\left({ }^{* *}\right)$.
\begin{shadedTheorem}[$**$\,|\, Banach-Caccioppoli]\label{thm: 5.1 Banach Caccioppoli}
    Sia $(X,\|\cdot\|)$ uno spazio di Banach e sia $K$ un sottoinsieme chiuso di $X$. Inoltre sia $T: K \rightarrow K$ una contrazione, cioè esista $q<1$ tale che
   \[\left\|T\left(x_{1}\right)-T\left(x_{2}\right)\right\| \leq q\left\|x_{1}-x_{2}\right\|\]
   per ogni $x_{1}, x_{2} \in K$. Allora esiste uno e un solo $\bar{x} \in K$ tale che $T(\bar{x})=\bar{x}$.
\end{shadedTheorem}
\begin{proof}
    Fissato $x_0\in K$ poniamo $x_n:=T(x_{n-1})$ per ogni $n\in\mathbb{N}^*$. Osserviamo che 
    \[\|x_2-x_1\| = \|T(x_1)-T(x_0)\|\leq q \|x_1-x_0\|\]
    e, analogamente, per induzione 
    \[\|x_{n+1}-x_n\| = \|T(x_n)-T(x_{n-1})\|\leq q \|x_n-x_{n-1}\|\leq q^n\|x_1-x_0\|.\]
    Verifichiamo a questo punto che $\{x_n\}_{n\in\N^*}$ è di Cauchy. Siano $n,k\in\N^*$, allora
    \begin{align*}
        \|x_{n+k}-x_n\| &= \|x_{n+k}-x_{n+k-1}+x_{n+k-1}-x_{n+k-2}+x_{n+k-2}+\cdots+ x_{n+2}-x_{n+1}+x_{n+1}-x_n\|\leq\\
        & \leq \|x_{n+k}-x_{n+k-1}\|+\|x_{n+k-1}-x_{n+k-2}\|+\cdots+\|x_{n+2}-x_{n+1}\|+\|x_{n+1}-x_n\|\leq \\
        & \leq \left( q^{n+k-1}+q^{n+k-2} +\cdots + q^{n+1}+q^n\right)\|x_1-x_0\| = q^n \|x-x_0\|\sum_{j=0}^{k-1}q^j \leq q^n \|x-x_0\|\sum_{j=0}^{+\infty}q^j \\
        & = \frac{q^n}{1-q}\|x-x_0\|\xrightarrow[n\to +\infty]{}0\ \text{senza dipendenza da }k.
    \end{align*}
    Abbiamo ottenuto che $\{x_n\}_n$ è di Cauchy, quindi esiste un $\bar x \in X$ tale che $x_n\xrightarrow{\|\cdot\|}\bar x$, ovvero $\|x_n-\bar x\|\xrightarrow[n\to +\infty]{
    }0$. In particolare, poiché la successione è a valori in $K$ e $K$ è chiuso, ricordando che $K\subset \overline{K}^{seq}\subset \overline{K} =K$ si ha che $\bar x \in K$.

    Rimane da provare che $T(\bar x) = \bar x$. È sufficiente osservare che 
    \begin{align*}
        \|T(\bar x)-\bar x\| &= \|T(\bar x)-\undereq{x_{n+1}}{T(x_n)}+T(x_n)-\bar{x}\| \leq  \|T(\bar x)-T(x_n)\|+\|x_n-\bar x\|\leq q \|\bar x-x_n\| + \|x_n-\bar x\|\xrightarrow[n\to +\infty]{}0.
    \end{align*}
    Di conseguenza, deve essere $\|T(\bar x)-\bar x\| = 0$, ovvero $T(\bar x) = \bar x$.
\end{proof}

% TeOREma $5.2\left({ }^{* *}\right)$. 
\begin{shadedTheorem}[$**$]\label{thm: 5.2 Cauchy}
    Consideriamo un sottoinsieme aperto $A$ di $\mathbb{R}^{2}, F \in C(A)$ e $\left(x_{0}, y_{0}\right) \in$ A. Supponiamo inoltre che $F$ sia localmente $y$-Lipschitziana in $\left(x_{0}, y_{0}\right)$, cioè che esistano $r_{1}>0$ e $L>0$ tali che:
    \begin{enumerate}
        \item $Q:=\left[x_{0}-r_{1}, x_{0}+r_{1}\right] \times\left[y_{0}-r_{1}, y_{0}+r_{1}\right] \subset A$;
        \item $\left|F\left(x, y_{1}\right)-F\left(x, y_{2}\right)\right| \leq L\left|y_{1}-y_{2}\right|$ per ogni $x \in\left[x_{0}-r_{1}, x_{0}+r_{1}\right]$ e per ogni $y_{1}, y_{2} \in\left[y_{0}-r_{1}, y_{0}+r_{1}\right]$.
    \end{enumerate}
    Allora, posto
    \[M:=\max _{Q}|F|\]
    e scelto arbitrariamente
    \[0<r_{0}<\min \left\{\frac{r_{1}}{1+M}, \frac{1}{L}\right\}\]
    esiste una e una sola $u \in C^{1}\left(\left[x_{0}-r_{0}; x_{0}+r_{0}\right],\left[y_{0}-r_{1}, y_{0}+r_{1}\right]\right)$ tale che
    \[\begin{cases}u\left(x_{0}\right)=y_{0}\\u^{\prime}(x)=F(x, u(x))\end{cases}\tag{$*$}\label{eqPdC: teorema esistenza locale}\]
    per ogni $x \in\left[x_{0}-r_{0}, x_{0}+r_{0}\right]$.
\end{shadedTheorem}
\begin{proof}
    Vogliamo applicare il Teorema di Banach-Caccioppoli: introduciamo quindi $X:= C([x_{0}-r_{0}, x_{0}+r_{0}])$ con la norma $\|\cdot\|:=\|\cdot\|_{\infty, [x_0-r_0,x_0+r_0]}$ e osserviamo che $(X, \|\cdot\|)$ è uno spazio di Banach. Definiamo inoltre $K:=C([x_{0}-r_{0}, x_{0}+r_{0}] ; [y_0-r_1, y_0+r_1])$. Osserviamo che $K$ è un sottoinsieme chiuso di $X$. Definiamo infine l'operatore 
    \[T(u)(x) := y_0+\int_{x_0}^xF(t,u(t))\d t\qquad \qquad\text{per }x\in [x_0-r_0, x_0+r_0].\]
    Osserviamo che $(t,u(t))\in R\subseteq A$ (dove $R:=\left[x_{0}-r_{0}; x_{0}+r_{0}\right]\times\left[y_{0}-r_{1}, y_{0}+r_{1}\right]$) quindi $F(t,u(t))$ è ben definita e continua, per cui anche $T(u)$ è ben definita e continua.
    \begin{enumerate}
        \item Verifichiamo che $T(u)\in K$. Sia $x\in [x_0-r_0, x_0+r_0]$\tesi[ 1]{T(u)(x)\in \left[y_{0}-r_{1}, y_{0}+r_{1}\right]}
        \[|T(u)(x)-y_0| = \left|\int_{x_0}^xF(t,u(t))\d t\right| \overset{\boxed{\scriptstyle x\geq x_0}}{\leq} \int_{x_0}^x \underbrace{|F(t,u(t))|}_{\leq M}\d t \leq M|x-x_0| < \frac{M}{1+M}r_1<r_1.\]
        Analogamente si procede nel caso in cui $x\leq x_0$\hspace{\fill}$-$/$-$
        \item Verifichiamo che $T$ è una contrazione. Siano $u, v\in K$ \tesi[ 2]{\|T(u)-T(v)\|\leq q\|u-v\|}
        \begin{align*}\|T(u)-T(v)\|&\leq \left|\int_{x_0}^xF(t,u(t))\d t - \int_{x_0}^xF(t,u(t))\d t\right| = \left|\int_{x_0}^xF(t,u(t))-F(t,v(t))\d t\right|\overset{\boxed{\scriptstyle x\geq x_0}}{\leq}\\ &\leq \int_{x_0}^x|F(\underbrace{t,u(t)}_{\in R})- F(\underbrace{t,v(t)}_{\in R})|\d t \leq L \int_{x_0}^x \underbrace{|u(t)-v(t)|}_{\leq \|u-v\|}\d t = L|x-x_0|\|u-v\|\leq Lr_0\|u-v\|\end{align*}
        Passando all'estremo superiore ad entrambi i membri si ha 
        \[\|T(u)-T(v)\|\leq Lr_0\|u-v\|\]
        dove $r_{0}<\min \left\{\frac{r_{1}}{1+M}, \frac{1}{L}\right\}\leq \frac{1}{L}$, quindi $Lr_0<1$. Analogamente si procede nel caso in cui $x\leq x_0$\hspace{\fill}$-$/$-$
    \end{enumerate}
    Applicando il Teorema di Banach-Caccioppoli otteniamo che esiste uno e un solo $\bar{u}\in K$ tale che $T(\bar{u})=\bar{u}$, ovvero 
    \[T(\bar{u})(x) = y_0+\int_{x_0}^xF(t,\bar u(t)) \d t.\]
    In particolare $\bar u \in C^{1}\left(\left[x_{0}-r_{0}; x_{0}+r_{0}\right],\left[y_{0}-r_{1}, y_{0}+r_{1}\right]\right)$ e, derivando, per il Teorema Fondamentale del Calcolo
    \[\bar u'(x) = F(x,\bar u(x)) \qquad \qquad\text{per }x\in [x_0-r_0, x_0+r_0].\]
    e $\bar u(x_0)=y_0$, ovvero $\bar u$ è soluzione del problema di Cauchy \eqref{eqPdC: teorema esistenza locale}.
\end{proof}
\begin{exc}
    Dimostrare che $K$ come definito nella Dimostrazione è un sottoinsieme chiuso di $X$.
\end{exc}
% OSSERVAZIONE 5.1. 
\begin{oss}
    Consideriamo un sottoinsieme aperto $A$ di $\mathbb{R}_{x} \times \mathbb{R}_{z}^{n},\left(x_{0}, \eta_{0}, \ldots, \eta_{n-1}\right) \in$ $A$ e $F \in C(A)$. Inoltre osserviamo che la funzione
    \[\Phi(x, z):=\left(z_{2}, \ldots, z_{n}, F(x, z)\right), \quad(x, z) \in A\]
    è continua. Possiamo allora affermare che il seguente problema di Cauchy per una EDO di ordine $n$ in forma normale

    \begin{equation}\label{eq: 5.1.1}
        \begin{cases}
        y^{(n)}(x)=F\left(x, y(x), y^{\prime}(x), \ldots, y^{(n-1)}(x)\right)\\
        y\left(x_{0}\right)=\eta_{0}\quad \ldots\quad  y^{(n-1)}\left(x_{0}\right)=\eta_{n-1}
        \end{cases}
    \end{equation}

    è equivalente al seguente problema di Cauchy per una EDO vettoriale del primo ordine in forma normale

    \begin{equation}\label{eq: 5.1.2}
    \begin{cases}
    z^{\prime}(x)=\left(z_{2}(x), \ldots, z_{n}(x), F(x, z(x))\right)^{t}\\
    z\left(x_{0}\right)=\left(\eta_{0}, \ldots, \eta_{n-1}\right)^{t}
    \end{cases}
    \end{equation}

    con $z(x)=\left(z_{1}(x), \ldots, z_{n}(x)\right)^{t}$. Con ciò intendiamo che:
    \begin{enumerate}[i]
        \item  Se $I$ è un intervallo aperto contenente $x_{0}$ e $y(x) \in C^{n}(I)$ verifica \eqref{eq: 5.1.1}, allora $z(x):=\left(y(x), y^{\prime}(x), \ldots, y^{(n-1)}(x)\right)^{t} \in C^{1}\left(I, \mathbb{R}^{n}\right)$ e $z(x)$ verifica \ref{eq: 5.1.2};
        \item Se $I$ è un intervallo aperto contenente $x_{0}$ e $z(x)=\left(z_{1}(x), \ldots, z_{n}(x)\right)^{t} \in C^{1}\left(I; \mathbb{R}^{n}\right)$ verifica \eqref{eq: 5.1.2}, allora $y(x):=z_{1}(x) \in C^{n}(I)$ e $y(x)$ verifica \eqref{eq: 5.1.1}. Verifichiamo il caso $n=3$: consideriamo $z(x) =(z_1(x), z_2(x), z_3(x))$, $y:=z_1(x)$. Allora $z$ verifica \eqref{eq: 5.1.2} se e solo se 
        \[\begin{cases}
            z_1'(x) = z_2(x)\\
            z_2'(x) = z_3(x)\\
            z_3'(x) = F(x, z(x))\\
            z_1(x_0) = \eta_0, \quad 
            z_2(x_0) = \eta_1, \quad
            z_3(x_0) = \eta_2.
        \end{cases}\]
        da cui $y'(x) = z_1'(x) = z_2(x) \in C^1(I)$, quindi $y(x)\in C^2(I)$, e anche $y''(x) = z_2'(x) = z_3(x)\in C^1(I)$, quindi $y(x)\in C^3(I)$. Infine $y^{(3)} = (y''(x))' = z_3'(x) = F(x, z(x))$. Segue pertanto che $y(x)$ verifica \eqref{eq: 5.1.1}.
    \end{enumerate}
    Si dice quindi che i due problemi di Cauchy sono equivalenti.
\end{oss}

Lo stesso argomento che ha consentito di provare Teorema 5.2, permette di provare facilmente il seguente risultato relativo al problema di Cauchy per una EDO vettoriale del primo ordine in forma normale.

% TeORema $5.3\left({ }^{\circ}\right)$. 
\begin{shadedTheorem}[$\circ$]
    Consideriamo un sottoinsieme aperto $A$ di $\mathbb{R}_{x} \times \mathbb{R}_{z}^{n}, \Phi \in C\left(A, \mathbb{R}^{n}\right)$, $x_{0} \in \mathbb{R}_{x}$ e $\eta \in \mathbb{R}_{z}^{n}$ tali che $\left(x_{0}, \eta\right) \in A$. Supponiamo inoltre che $\Phi$ sia localmente $z$ Lipschitziana in $\left(x_{0}, \eta\right)$, cioè che esistano $r_{1}>0$ e $L>0$ tali che:
    \begin{enumerate}[i]
        \item $Q:=\left[x_{0}-r_{1}, x_{0}+r_{1}\right] \times\left(\eta+\left[-r_{1}, r_{1}\right]^{n}\right) \subset A$;
        \item $\left|\Phi\left(x, z^{\prime}\right)-\Phi\left(x, z^{\prime \prime}\right)\right| \leq L\left|z^{\prime}-z^{\prime \prime}\right|$ per ogni $x \in\left[x_{0}-r_{1}, x_{0}+r_{1}\right]$ e per ogni $z^{\prime}, z^{\prime \prime} \in$ $\eta+\left[-r_{1}, r_{1}\right]^{n}$.
    \end{enumerate}

    Allora, posto
    \[M:=\max _{Q}|\Phi|\]
    e scelto arbitrariamente
    \[0<r_{0}<\min \left\{\frac{r_{1}}{1+M}, \frac{1}{L}\right\}\]
    esiste una e una sola $u \in C^{1}\left(\left[x_{0}-r_{0}, x_{0}+r_{0}\right], \eta+\left[-r_{1}, r_{1}\right]^{n}\right)$ tale che $u\left(x_{0}\right)=\eta e$
    \[u^{\prime}(x)=\Phi(x, u(x))\]
    per ogni $x \in\left[x_{0}-r_{0}, x_{0}+r_{0}\right]$.
\end{shadedTheorem}
Tenendo conto di Osservazione 5.1, si ottiene ora facilmente il seguente corollario di Teorema 5.3 relativo all'esistenza e unicità locale della soluzione per il problema di Cauchy per una EDO di ordine $n$ in forma normale.
% Corollario $5.1\left(^{*}\right)$. 
\begin{corollary}[$*$]\label{cor: 5.1}
    
    Consideriamo un sottoinsieme aperto $A$ di $\mathbb{R}_{x} \times \mathbb{R}_{z}^{n}, F \in C(A)$ e $\left(x_{0}, \eta\right) \in A$. Supponiamo inoltre che $F$ sia localmente $z$-Lipschitziana in $\left(x_{0}, \eta\right)$, cioè che esistano $r_{1}>0$ e $L>0$ tali che:
    \begin{enumerate}[i]
        \item  $Q:=\left[x_{0}-r_{1}, x_{0}+r_{1}\right] \times\left(\eta+\left[-r_{1}, r_{1}\right]^{n}\right) \subset A$;
        \item $\left|F\left(x, z^{\prime}\right)-F\left(x, z^{\prime \prime}\right)\right| \leq L\left|z^{\prime}-z^{\prime \prime}\right|$, per ogni $x \in\left[x_{0}-r_{1}, x_{0}+r_{1}\right]$ e per ogni $z^{\prime}, z^{\prime \prime} \in \eta+\left[-r_{1}, r_{1}\right]^{n}$
    \end{enumerate}
    Allora esistono $r_{0} \in\left(0, r_{1}\right)$ e una-e-una-sola $u \in C^{n}\left(\left[x_{0}-r_{0}, x_{0}+r_{0}\right];\left[\eta_{0}-r_{1}, \eta_{0}+r_{1}\right]\right)$ tali che
    \[\begin{cases}
        u^{(n)}(x)=F\left(x, u(x), u^{\prime}(x), \ldots, u^{(n-1)}(x)\right), x \in\left[x_{0}-r_{0}, x_{0}+r_{0}\right] \\
        u\left(x_{0}\right)=\eta_{0}, \ldots, u^{(n-1)}\left(x_{0}\right)=\eta_{n-1}
    \end{cases}\]
\end{corollary}
% Corollario $5.2(*)$. 
\begin{corollary}[$*$]\label{cor: 5.2}
    Siano $a_{0}(x), a_{1}(x), \ldots, a_{n-1}(x), b(x)$ funzioni continue in un intervallo $(\alpha, \beta)$ e sia $\left(x_{0}, \eta\right)=\left(x_{0}, \eta_{0}, \ldots, \eta_{n-1}\right) \in(\alpha, \beta) \times \mathbb{R}^{n}$. Inoltre sia $r_{1}$ un numero reale positivo tale che $\left[x_{0}-r_{1}, x_{0}+r_{1}\right] \subset(\alpha, \beta)$. Allora esiste $r_{0} \in\left(0, r_{1}\right)$ tale che il problema di Cauchy

    \begin{equation}\label{eq: 5.1.3}
    \begin{cases}
    y^{(n)}(x)+a_{n-1}(x) y^{(n-1)}(x)+\ldots+a_{1}(x) y^{\prime}(x)+a_{0}(x) y(x)=b(x)  \\
    y\left(x_{0}\right)=\eta_{0}, \ldots, y^{(n-1)}\left(x_{0}\right)=\eta_{n-1}
    \end{cases}
    \end{equation}
    
    ammette una ed una sola soluzione in $C^{n}\left(\left[x_{0}-r_{0}, x_{0}+r_{0}\right],\left[\eta_{0}-r_{1}, \eta_{0}+r_{1}\right]\right)$.
\end{corollary}
Usando opportuni risultati sul prolungamento delle soluzioni del problema di Cauchy, si giunge a dimostrare il seguente teorema.

% TEorema 5.4. 
\begin{shadedTheorem}
    Nelle ipotesi di Corollario 5.2, esiste una e una sola funzione $u \in C^{n}((\alpha, \beta))$ che risolve il problema \eqref{eq: 5.1.3}.
\end{shadedTheorem}
