\section{Misure esterne, definizione e prime proprietà}

\begin{definition}{Misura esterna}{misura esterna}
    Sia $X$ un insieme e $2^X$ il suo insieme delle parti.\\
    Una \bemph{misura esterna} sull'insieme $X$ è una mappa $\varphi: 2^X \rightarrow[0,+\infty]$ tale che
    \begin{enumerate}
        \item $\varphi(\varnothing)=0$;
        \item $\varphi(E) \leq \varphi(F)$, se $E \subset F \subset X$;
        \item $\varphi\left(\cup_{j} E_{j}\right) \leq \sum_{j} \varphi\left(E_{j}\right)$, se $\left\{E_{j}\right\}$ è una famiglia numerabile di sottoinsiemi di $X$.
    \end{enumerate}
\end{definition}

\begin{example}{Misura esterna banale}{misura banale}
    Sia $X \neq \varnothing$ e sia \[\varphi(E) := \begin{cases} 
        0 \quad \text{se } E=\varnothing\\
        1 \quad \text{altrimenti}
    \end{cases}\] $\varphi$ è una misura esterna su $X$.
\end{example}
\begin{example}{Misura esterna di Dirac}{misura di Dirac}
    Sia $X \neq \varnothing$, $x_0 \in X$ e sia \[\varphi_{x_0}(E) := \begin{cases} 
        1 \quad \text{se } x_0 \in E\\
        0 \quad \text{altrimenti}
    \end{cases}\] $\varphi_{x_0}$ è una misura esterna su $X$.
\end{example}
\begin{example}{Misura del conteggio}{misura del conteggio}
    Sia $X$ un insieme. La mappa $\varphi(E):= \# E$ è una misura esterna su $X$.
\end{example}

\begin{remark}{Misure di Peano-Jordan}{}
    La misura (inferiore/superiore) di Peano-Jordan (come definita sotto) non è una misura.
    \begin{notation}
        Un intervallo in $\R^n$ è qualsiasi insieme che sia prodotto di intervalli di $\R$.
    \end{notation}
    \begin{notation}
        La misura elementare di un intervallo aperto (ma pure del chiuso) $(a,b)\subset \R$ è $b-a$ e la misura elementare di un intervallo prodotto è il prodotto delle misure elementari delle sue componenti.\\
        Se $\mc{F}$ è una famiglia di intervalli, denotiamo con $S(\mc{F})$ la somma delle misure elementari di ciascun intervallo.
    \end{notation}
    \begin{proof}
        Sia $A \subseteq\R^2$ e siano:\begin{itemize}
            \item \bemph{Misura inferiore di Peano-Jordan}: definiamo l'insieme di famiglie di intervalli
            \[\mc{I}_-(A) := \left\{ \{I_i\}_1^n : I_i \text{ intervalli},\quad \bigcup_{i=1}^n I_i \subseteq A, \quad I_i \cap I_j = \varnothing \quad \text{se }i \neq j \right\}\]
            Allora la misura inferiore di Peano-Jordan è data da
            \[J_-(A) := \begin{cases}
                0 \quad\text{se } \mc{I}_-(A) = \varnothing\\
                \sup_{\mc{I}_-(A)} S \quad\text{altrimenti}
            \end{cases}\]
            Osserviamo che se non richiedessimo gli intervalli disgiunti, questa esploderebbe sempre all'infinito. Verifichiamo gli assiomi di misura:\begin{enumerate}
                \item $J_-(\varnothing) = 0$ banalmente.
                \item Osserviamo che $\mc{I}_-$ è monotono per inclusione, come lo è $\sup$, dunque lo è anche $J_-$.
                \item Consideriamo $E_1 := \Q^2\cap (0,1)^2$ e $E_2 := (0,1)^2 \setminus E_1$. Per densità di $\Q^2$ abbiamo che $J_-(E_1) = J_-(E_2) = 0$ ma $J_-(E_1 \cup E_2)>0$, dunque cade la disuguaglianza richiesta.  
            \end{enumerate}
            \item \bemph{Misura superiore di Peano-Jordan}: definiamo l'insieme di famiglie di intervalli
            \[\mc{I}_+(A) := \left\{ \{I_i\}_1^n : I_i \text{ intervalli},\quad \bigcup_{i=1}^n I_i \supseteq A, \quad I_i \cap I_j = \varnothing \quad \text{se }i \neq j \right\}\]
            Notiamo che questo è definito per gli insiemi limitati, dunque per questi definiamo la misura superiore di Peano-Jordan
            \[J_+(A) := \begin{cases}
                0 \quad\text{se } \mc{I}_+(A) = \varnothing\\
                \inf_{\mc{I}_+(A)} S \quad\text{se }A\text{ limitato}\\
                \lim_{\rho \to +\infty} J_+(A\cap B_\rho(\mathbf{0})) \quad\text{altrimenti}
            \end{cases}\]
            L'esistenza del limite segue dalla monotonia inversa di $\mc{I}_+$ e $\inf$. Verifichiamo gli assiomi di misura esterna
            \begin{enumerate}
                \item $J_+{\varnothing} = 0$ banalmente.
                \item Abbiamo già dimostrato la monotonia.
                \item Osserviamo che $J_+(\Q\cap ]0,1[)>0$, mentre (indicizzando $\Q$) $\sum_j J_+(\{q_j\}) = 0$.
            \end{enumerate}
            \item \bemph{Misura di Peano-Jordan} sia $D \subset 2^{\R^2}$ la famiglia di sottoinsiemi tali che la misura inferiore e la misura superiore coincidono. Il dominio della mappa $J(A) := J_-(A)=J_+(A)$ non corrisponde a tutte le parti di $\R$ come visto sopra, dunque non è una misura esterna.
        \end{itemize}
    \end{proof}    
\end{remark}

\begin{definition}{Insiemi misurabili}{insiemi misurabili}
    Siano $X$ un insieme e $\varphi$ una misura esterna su $X$.\\
    Un sottoinsieme $E\subset X$ si dice \bemph{misurabile} per $\varphi$ se per ogni $A\subset X$ vale: $\varphi(A) = \varphi(A \cap E) + \varphi(A\cap E^c)$.\\
    Denotiamo la famiglia dei misurabili per $\varphi$ con $\M_\varphi$.
\end{definition}
\begin{remark}{}{}
    Per il terzo punto della definizione \ref{def:misura esterna} potremmo "rilassare" questa definizione con $\varphi(A) \ge \varphi(A \cap E) + \varphi(A\cap E^c)$
\end{remark}
\begin{example}{Insiemi misurabili per gli esempi precedenti}{esempi di misurabili}
    Negli esempi \ref{exmp:misura banale}, \ref{exmp:misura di Dirac} e \ref{exmp:misura del conteggio} abbiamo rispettivamente:
    \[\M_\varphi = \{\varnothing, X\}, \qquad \M_\varphi = 2^X, \qquad \M_\varphi = 2^X\].
\end{example}

\begin{theorem}{*** | Fondamentale sui misurabili}{fondamentale sui misurabili}
    Sia $X$ un insieme e $\varphi$ una misura esterna su $X$. Valgono i seguenti:\begin{enumerate}
        \item $\M_\varphi$ è chiusa per complemento.
        \item Gli insiemi di misura nulla sono misurabili.
        \item $\M_\varphi$ è chiusa per intersezione (unione) finita.
        \item $\M_\varphi$ è chiusa per unione numerabile di insiemi disgiunti.
        \item $\varphi$ è addittiva per unione numerabile di insiemi disgiunti.
    \end{enumerate}
    \begin{proof}
        \begin{enumerate}
            \item Banale per definizione di misurabile.
            \item Sia $E$ tale che $\varphi(E) = 0$. Abbiamo che $0\leq \varphi(A\cap E)\leq \varphi(E)=0$ e quindi $\varphi(A) \leq \varphi(E) + \varphi(A \cap E^c)\leq \varphi(A)$.
            \item TODO
            \item 
        \end{enumerate}
    \end{proof}
\end{theorem}