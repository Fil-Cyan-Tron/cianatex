\documentclass{article}
\usepackage{cianatex}

\title{Tentativi disperati di mettere una misura significativa sulle varietà, o sugli schemi se Dio vuole mannaggia alla topologia di Zariski}
\author{Filippo $\mc{L}$. Troncana}
\date{A.A. 2024/2025}

\begin{document}

\maketitle

\section{Definizioni preliminari}

\begin{definition}{Misura esterna}{misura esterna}
    Sia $X$ un insieme e sia $2^X$ il suo insieme delle parti. Una \bemph{misura esterna} su $X$ è una funzione $\mu : 2^X\to [0,+\infty]$ tale che:\begin{enumerate}
        \item $\mu(\varnothing) = 0$
        \item Se $E \subset F$ allora $\mu(E)\le \mu(F)$
        \item Per ogni $\{E_i\}_{i \in \N}$ vale
        \[ \mu\left( \bigcup_{i \in \N} E_i \right) \le \sum_{i \in \N} \mu(E_i) \]
    \end{enumerate}
\end{definition}

\begin{theorem}{Le misure di Radòn funzionano male su Zariski}{Incompatibilità Radòn-Zariski}
    Sia $(X,\tau)$ uno spazio topologico Noetheriano, sia $\B$ la famiglia dei suoi boreliani e sia $\mu : \B \to [0,+\infty]$ una misura di Radòn su $X$. Valgono i seguenti
    \begin{enumerate}
        \item $\mu(X) < +\infty$
        \item Se $V$ è un chiuso irriducibile, $\mu(V) = 0$
        \item Se $A$ è un aperto, $\mu(A) = \mu(X)$
    \end{enumerate}
    \proof 
    \begin{enumerate}
        \item Dato che in uno spazio topologico Noetheriano tutti i sottoinsiemi sono compatti, vale banalmente.
    \end{enumerate}
\end{theorem}

\end{document}