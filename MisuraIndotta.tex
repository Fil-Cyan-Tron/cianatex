\documentclass{article}
\usepackage{cianatex}

\title{TFA per cambiamenti di coordinate}
\author{Filippo $\L$. Troncana}
\date{A.A. 2023/2024}

\addbibresource{MisuraIndotta.bib}

\renewcommand\A{\mc{A}}
\renewcommand\F{\mc{F}}
\renewcommand\N{\mc{N}}
\newcommand\Nat{\mathbb{N}}


\begin{document}

\maketitle

\begin{abstract}
    Uno dei più importanti risultati della teoria geometrica della misura è la formula dell'area, strumento fondamentale per il calcolo delle misure di sottovarietà regolare di $\R^n$ e degli integrali su di esse.\\
    La dimostrazione classica, ad esempio quella riportata in \cite{EvansGariepy1991} fa uso di diverse stime estremamente tecniche, ma credo\footnote{o meglio, spero} che fare un giro leggermente più largo possa portare a una dimostrazione meno traumatica.\\
    Alcune fondamentali idee, come quella di considerare spazi misurabili "migliorati" (che noi chiameremo rinforzati), ovvero dotati di una famiglia di insiemi considerati trascurabili o nulli, per un'idea più "naturale" di equivalenza quasi ovunque vengono da \cite{Fremlin2000}.\\
    In questa tesi vengono presentati dei risultati di teoria della misura sviluppati con un approccio simile a quello usato per lo studio della topologia generale e successivamente questi vengono applicati allo studio dell'integrale di funzioni composte e alla formula dell'area.
\end{abstract}

\tableofcontents

\begin{notation}
    Useremo le seguenti convenzioni:\begin{itemize}
        \item Dato un insieme $X$, indicheremo con $2^X$ il suo insieme delle parti.
        \item Dato un insieme $X$ e un sottoinsieme $E\subset X$, indicheremo con $E^c$ il suo complementare $X\setminus E$.
        \item Dato un insieme $X$ e una sua famiglia di sottoinsiemi $\F \subset 2^X$, la notazione $\{A_i\}_{i \in I} \subset \F$ rappresenta una funzione $\varphi: I \to \F$ che a ciascun indice mappa un insieme di $\F$ e indichiamo:
        \[\cup_I E_i := \bigcup_{i \in I} E_i \qquad, \qquad \cap_I E_i := \bigcap_{i \in I} E_i \qquad \text{e}\qquad \Pi_I E_i := \prod_{i \in I} E_i \]
        In particolare, quest'ultimo è definito se $I$ è finito (non ci occuperemo di prodotti cartesiani infiniti) e se $E_i = E_j= E $ per ogni $i,j$, allora $\Pi_I E_i := E^{\# I}$
        \item Dato un campo $K$ e una successione di elementi del campo $\{a_i\}_{i \in I}\subset K$, indichiamo
        \[\Sigma_I a_i := \sum_{i \in I} a_i \qquad\text{e} \qquad \Pi_I a_i := \prod_{i \in I} a_i \]
        Almeno a livello formale, indipendentemente dalla loro esistenza o definizione.
    \end{itemize}
\end{notation}

\pagebreak
\section{Teoria astratta della misura indotta}

Le definizioni di teoria della misura usate si riferiscono a quelle date in \cite{Delladio2023}, meno che alcune che riportiamo qui, con opportuna motivazione

\begin{definition}{Funzione misurabile}{funzione misurabile}
    Siano $(X,\A)$ e $(Y,\B)$ due spazi misurabili e sia $f : X\to Y$ una funzione.\\
    $f$ si dice \bemph{misurabile} se per ogni $E \in \B$ vale $f^{-1}(B) \in \A$.
\end{definition}

In \cite{Delladio2023} le funzioni misurabili sono definite analogamente, ma l'ambiente di arrivo è uno spazio topologico e si richiede che la controimmagine di ogni aperto sia misurabile, in modo da poter usare alcuni strumenti di topologia dotando l'insieme di arrivo della $\sigma$-algebra Boreliana.\\
Tuttavia, ai fini della nostra trattazione sarà meglio usare la definizione più generale riportata qui sopra, che quindi è quella che adottiamo.

\subsection{$\sigma$-algebre e misure esterne indotte da funzioni}

Analogamente alle costruzioni di topologia iniziale e finale, definiamo 

\begin{definition}{$\sigma$-algebre indotte}{sigma-algebre indotte}
    Siano $X$ e $Y$ due insiemi, sia $f : X \to Y$ una funzione, sia $\A$ una $\sigma$-algebra su $X$ e sia $\B$ una $\sigma$-algebra su $Y$.\\
    Definiamo le seguenti famiglie:
    \[f_\sharp \A := \{E \in 2^Y : f^{-1}(E)\in \A\}\qquad \text{e}\qquad f_\flat\B := \{f^{-1}(E) \in 2^X : E \in \B \}\]
    Esse si dicono rispettivamente $\sigma$\bemph{-algebra finale e iniziale} di $f$ rispetto a $\A$ e $\B$.
\end{definition}

\begin{proposition}{}{}
    Nella situazione della definizione \ref{def:sigma-algebre indotte}, $f_\sharp\A$ e $f_\flat\B$ sono $\sigma$-algebre.
    \begin{proof}
        Segue banalmente dalla commutatività tra operatori insiemistici, immagine e preimmagine.
    \end{proof}
\end{proposition}

\begin{remark}{}{}
    La $\sigma$-algebra finale di $f$ rispetto a $\A$ è la più grande $\sigma$-algebra $\Omega$ tale che $f : (X,\A)\to (Y,\Sigma)$ sia misurabile.\\
    La $\sigma$-algebra iniziale di $f$ rispetto a $\B$ è la più piccola $\sigma$-algebra $\Sigma$ tale che $f : (X,\Sigma)\to (Y,\B)$ sia misurabile.
    \begin{proof}
        Sia $\Omega \subset 2^Y$ tale che $f : (X,\A)\to (Y,\Omega)$ sia misurabile. Per definizione di funzione misurabile, abbiamo che per ogni $E \in \Omega$, abbiamo che $f^{-1}(E)\in \A$, dunque $\Omega\subset f_\sharp\A$.\\
        Sia $\Sigma \subset 2^X$ tale che $f : (X,\Sigma)\to (Y,\B)$ sia misurabile. Per definizione di funzione misurabile, abbiamo che per ogni $E \in f_\flat\B$ si ha che $E = f^{-1}(F)$ con $F \in B$ e quindi che $E \in \Sigma$, dunque $f_\flat\B\subset \Sigma$.
    \end{proof}
\end{remark}

\begin{definition}{Misure esterne indotte}{misure esterne indotte}
    Siano $X$ e $Y$ due insiemi, siano $\mu$ e $\nu$ due misure esterne rispettivamente su $X$ e su $Y$ e sia $f:X\to Y$ una funzione.\\
    La \bemph{misura esterna finale} di $f$ rispetto a $\mu$ è la funzione
    \[f_\sharp\mu : 2^Y \to [0,+\infty] \quad \text{con} \quad f_\sharp\mu(E):= \mu(f^{-1}(E)) \]
    La \bemph{misura esterna iniziale} di $f$ rispetto a $\nu$ è la funzione
    \[f_\flat\nu : 2^X \to [0,+\infty] \quad \text{con} \quad f_\flat\nu(E) := \nu(f(E))\]
\end{definition}
\begin{proposition}{}{misure esterne indotte}
    Nella situazione della definizione \ref{def:misure esterne indotte}, $f_\sharp\mu$ è una misura esterna su $Y$ e $f_\flat\nu$ è una misura esterna su $X$.
    \begin{proof}
        Dimostriamo i tre assiomi di misura esterna per $f_\sharp\mu$:\begin{enumerate}
            \item $f^{-1}(\varnothing) = \varnothing \Rarr f_\sharp\mu(\varnothing) = 0$.
            \item Siano $E \subset F \subset Y$, allora $f^{-1}(E)\subset f^{-1}(F)$, dunque la monotonia di $f_\sharp\mu$ segue dalla monotonia di $\mu$.
            \item Siano $A,B \subset Y$, allora $f^{-1}(A\cup B)= f^{-1}(A) \cup f^{-1}(B)$ e la subaddittività segue da quella di $\mu$
        \end{enumerate}
        Ora per $f_\flat\nu$:\begin{enumerate}
            \item $f(\varnothing) = \varnothing \Rarr f_\flat(\varnothing) = 0$.
            \item Siano $E \subset F \subset X$, allora $f(E)\subset f(F)$, dunque la monotonia di $f_\flat\nu$ segue dalla monotonia di $\nu$.
            \item Siano $A,B \subset X$, allora $f(A \cup B) = f(A)\cup f(B)$ e la subaddittività segue da quella di $\nu$.
        \end{enumerate}
    \end{proof}
\end{proposition}

\begin{remark}{TOCORRECT: Bidualità delle $\sigma$-algebre}{bidualità algebre}
    Nella situazione della definizione \ref{def:sigma-algebre indotte}, $f_\flat f_\sharp \A = \A$ e $f_\sharp f_\flat \B = \B$ SONO INCLUSIONI NON UGUAGLIANZE PER L'UGUAGLIANZA VUOI INIETTIVITÀ E SURIETTIVITÀ.
    \begin{proof}
        Per definizione
        \[f_\flat f_\sharp \A = \{ f^{-1}(E) \in 2^X : E \in f_\sharp \A\} = \{f^{-1}(E) \in 2^X : f^{-1}(E) \in \A\} = \{E \in 2^X : E \in \A\} = \A\]
        Allo stesso modo
        \[f_\sharp f_\flat \B = \{ E \in 2^Y : f^{-1}(E) \in f_\flat \B \} = \{ E \in 2^Y : E \in \B\} = \B\]
    \end{proof}
\end{remark}

\begin{proposition}{Bidualità delle misure esterne indotte}{bidualità misure}
    Nella situazione della definizione \ref{def:misure esterne indotte}, $f_\flat f_\sharp\mu \ge \mu$ e $f_\sharp f_\flat\nu \le \nu$. In particolare, se $f$ è iniettiva vale $f_\flat f_\sharp\mu = \mu$ e se $f$ è suriettiva vale $f_\sharp f_\flat\nu = \nu$
    \begin{proof}
        Abbiamo che $f_\flat f_\sharp\mu(E) = f_\sharp\mu(f(E)) = \mu(f^{-1}(f(E)))\ge \mu(E)$ per monotonia di $\mu$.\\
        Allo stesso modo, $f_\sharp f_\flat\nu(E) = f_\flat\nu(f^{-1}(E)) = \nu(f(f^-1(E))) \le \nu(E)$\\
        L'uguaglianza nei casi particolari segue banalmente.
    \end{proof}
\end{proposition}
\begin{corollary}{}{}
    Sotto le ipotesi della proposizione \ref{prop:bidualità misure}, se $f$ è una funzione biettiva vale l'uguaglianza.
\end{corollary}

Analogamente alle costruzioni topologiche, usiamo questa teoria per parlare di sottospazi misurabili

\pagebreak
\subsection{Sottospazi misurabili}

\begin{definition}{Sottospazio misurabile}{sottospazio misurabile}
    Sia $(X,\A)$ uno spazio misurabile e $Z\subset X$. Allora, definita la famiglia $\A|_Z := \{E\cap Z \in 2^X : E\in \A\}$, $(Z, \A|_Z)$ si dice \bemph{sottospazio misurabile} di $X$.
\end{definition}
\begin{remark}{}{sottospazio misurabile}
    $(Z,\A|_Z)$ è effettivamente uno spazio misurabile.
    \begin{proof}
        La dimostrazione è banale.
    \end{proof}
\end{remark}

\begin{proposition}{Misurabili iniziali rispetto all'inclusione}{misurabili iniziali inclusione}
    Sia $(X,\A)$ uno spazio misurabile, sia $Z\subset X$ un suo sottoinsieme e sia $i : Z \to X$ l'inclusione canonica.\\
    Allora $\A|_Z = i_\flat \A$.
    \begin{proof}
        Per $E \in i_\flat \A$ vale se e solo se $E = i^{-1}(F)$ per qualche $F \in \A$, ma per ogni $F \in 2^X$ vale $i^{-1}(F) = F \cap Z$, dunque $E = F\cap Z$ per qualche $F \in \A$ e quindi $E \in \A|_Z$.
    \end{proof}
\end{proposition}

\begin{definition}{Sottomisura esterna}{sottomisura esterna}
    Sia $X$ un insieme, $Z\subset X$ un suo sottoinsieme, $i:Z\to X$ l'inclusione canonica e sia $\mu : 2^X \to [0,+\infty]$ una misura esterna su $X$.\\
    Allora $i_\flat \mu$ si dice \bemph{sottomisura esterna} su $Z$ rispetto a $X$.
\end{definition}

Notiamo che è effettivamente una misura esterna per la proposizione \ref{prop:misure esterne indotte}, adesso specifichiamo 

\begin{proposition}{Sottomisura esterna e restrizione}{sottomisura esterna}
    Nella situazione della definizione \ref{def:sottomisura esterna}, vale $i_\flat \mu = \mu|_{2^Z} = \mu \cdot \chi_Z$.
\end{proposition}

\pagebreak
\subsection{Spazi misurabili prodotto}

\begin{definition}{Spazio misurabile prodotto}{spazio misurabile prodotto}
    Siano $(X,\A)$ e $(Y,\B)$ due spazi misurabili, $X\times Y$ il prodotto cartesiano dei due insiemi e $\pi_X : X\times Y \to X$ e $\pi_Y : X\times Y \to Y$ le proiezioni canoniche.\\
    Lo \bemph{spazio misurabile prodotto} $(X,\A)\otimes (X,\B)$ è lo spazio $(X\times Y, \A \otimes \B)$ dove $\A \otimes\B$ è la più piccola $\sigma$-algebra che contenga gli insiemi della forma $A\times B$ con $A \in \A$ e $B \in \B$.
\end{definition}
\begin{remark}{}{}
    Siano $(X,\A)$ e $(Y,\B)$ due spazi misurabili, $X\times Y$ il prodotto cartesiano dei due insiemi e $\pi_X : X\times Y \to X$ e $\pi_Y : X\times Y \to Y$ le proiezioni canoniche.\\
    $\A \otimes \B$ è la più piccola $\sigma$-algebra che renda misurabili sia $\pi_X$ che $\pi_Y$.
    \begin{proof}
        Notiamo che la tesi può essere riscritta come $\A\otimes\B \subset \langle\pi_{X\flat} \A \cup \pi_{Y\flat} \B\rangle $, in quanto una $\sigma$-algebra che renda misurabili le proiezioni deve necessariamente contenere l'unione delle $\sigma$-algebre iniziali\footnote{Notiamo che $\pi_{X\flat}\A = \{A \times Y \in 2^{X\times Y} : A \in \A\}$}, ma quindi per l'ipotesi di minimalità di $\A\otimes\B$ possiamo semplicemente richiedere $\{A\times B : A \in \A, B \in \B\}\subset \langle\pi_{X\flat} \A \cup \pi_{Y\flat} \B\rangle$.\\
        Notiamo che in generale, $A \times B = (A\times Y) \cap (X \times B)$, rispettivamente elementi di $\pi_{X\flat} \A$ e $\pi_{Y\flat} \B$, quindi deve appartenere alla $\sigma$-algebra generata dalla loro unione.
    \end{proof}
\end{remark}


\pagebreak
\subsection{Spazi misurabili rinforzati}

Un concetto fondamentale in teoria della misura è quello proprietà valide $\mu$-quasi ovunque, ma sorge il problema della scelta di una misura. In realtà è possibile "indebolire" questo requisito, specificando la famiglia degli insiemi nulli di uno spazio misurabile e imponendo un requisito di "fedeltà" per le misure che vorremo definire su di esso.

\begin{definition}{$\sigma$-ideale}{sigma-ideale}
    Sia $X$ un insieme e $I \subset 2^X$ una famiglia di insiemi tale che:\begin{enumerate}
        \item $\varnothing \in I$.
        \item Se $N \in I$ e $M\subset N$ allora $M \in I$.
        \item Se $\{N_i\}_{i \in \Nat}\subset I$ allora $\cup_\Nat N_i \in I$.
    \end{enumerate}
    Allora $I$ si dice $\sigma$\bemph{-ideale} su $X$. In particolare, se $X \notin I$, allora $I$ si dice $\sigma$-\bemph{ideale proprio}, altrimenti improprio\footnote{In quanto avremmo $I=2^X$, non particolarmente utile nel migliore dei casi.}.
\end{definition}

\begin{definition}{Spazio fortemente misurabile}{spazio fortemente misurabile}
    Sia $X$ un insieme, $\M$ una $\sigma$-algebra su $X$ e $\N$ un $\sigma$-ideale su $X$ tale che $\N\subset \M$.\\
    Allora $(X,\M,\N)$ si dice \bemph{spazio fortemente misurabile} e gli insiemi di $\N$ si dicono \bemph{nulli} o trascurabili. La coppia $(\M,\N)$ è detta \bemph{struttura fortemente misurabile}.
\end{definition}

Ovviamente ogni spazio misurabile rinforzato è uno spazio misurabile e una misura esterna $\mu$ su un insieme $X$ induce su di esso una struttura fortemente misurabile allo stesso modo in cui induce una normale struttura misurabile, con la coppia $(\M_\mu, \N_\mu)$ dove $\N_\mu := \{E \in \M : \mu(E) = 0\}$.

\begin{definition}{Validità quasi ovunque}{validità quasi ovunque}
    Sia $(X,\M,\N)$ uno spazio fortemente misurabile.\\
    Una proprietà $P$ sugli elementi di $X$ si dice \bemph{valida quasi ovunque} se $\{x \in X : \neg P(x)\}\in \N$ e scriviamo $\forall_\N x \in X, P(x)$.
\end{definition}

Notiamo che se $\N = \N_\mu$ per una misura $\mu$, questa diventa la definizione di validità $\mu$-quasi ovunque

\begin{notation}
    Sia $(X,\M,\N)$ uno spazio fortemente misurabile e siano $f, g : X \to \R$. Allora $=_\N$, $\ge_\N$, $>_\N$, $\le_\N$ e $<_\N$ si riferiscono alle stesse relazioni intese quasi ovunque.\\
    Se $\N = \N_\mu$ per qualche misura esterna, invece di scrivere $\N$ al pedice scriveremo $\mu$.
\end{notation}

\pagebreak
\section{Teoria dell'integrazione}

\subsection{Integrazione indotta}

La situazione che studiamo in questa sezione è la seguente

\[\begin{tikzcd}
	{(X,\A,\mu)} && {(Y, f_\sharp\A, f_\sharp\mu)} && {(Y, \B, \nu)} && {(X, f_\flat\B, f_\flat\nu)} \\
	\\
	&& {(\R, \M_\L, \L)} && {(\R,\M_\L, \L)}
	\arrow["f"{description}, from=1-1, to=1-3]
	\arrow["{g \circ f}"{description}, from=1-1, to=3-3]
	\arrow["g"{description}, from=1-3, to=3-3]
	\arrow["g"{description}, from=1-5, to=3-5]
	\arrow["f"{description}, from=1-7, to=1-5]
	\arrow["{g \circ f}"{description}, from=1-7, to=3-5]
\end{tikzcd}\]

Dove $\L$ è la misura di Lebesgue sui numeri reali.

\begin{theorem}{Integrazione indotta}{integrazione indotta}
    Sia $(X,\mc{A},\mu)$ uno spazio con misura, sia $Y$ un insieme, sia $f:X\to Y$ una funzione biettiva e sia $g:(Y,f\mc{A},f\mu)\to (\R,\B(\R),\L^1)$ una funzione $f\mc{A}$-misurabile.\\
    Allora $g$ è $f\mu$-integrabile se e solo se $g\circ f$ è $\mu$-integrabile, e vale l'identità
    \[\int\limits g \de f\mu = \int\limits g\circ f \de \mu\]
    \begin{proof}
        Assumiamo che $g$ sia $f\mu$-integrabile. Allora vale
        \[\int\limits g \de f\mu = \int\limits_* g \de f\mu = \sup\left\{I_{f\mu}(\varphi) : \varphi \in \Sigma_-(g)\right\}=\sup\left\{ \sum_i a_i f\mu(\varphi^{-1}(\{a_i\})) : \varphi \in \Sigma_-(g) \right\} =\]
        \[ = \sup\left\{ \sum_i a_i \mu(f^{-1}(\varphi^{-1}(\{a_i\}))) : \varphi \in \Sigma_-(g) \right\} = \sup\left\{ \sum_i a_i \mu((\varphi \circ f)^{-1}(\{a_i\})) : \varphi \circ f \in \Sigma_-(g\circ f) \right\}\]
        \[\text{con } \psi := \varphi \circ f, \quad \int\limits_* g \de f\mu =\sup\left\{ I_\mu(\psi) : \psi \in \Sigma_-(g\circ f) \right\} = \int\limits_* g \circ f \de \mu\]
        La dimostrazione è assolutamente analoga per l'integrale superiore e nella direzione opposta assumendo l'integrabilità di $g\circ f$. Le varie uguaglianze seguono dalla biettività di $f$.
    \end{proof}
\end{theorem}

\begin{remark}{Girotondone per il TFA}{}
    L'obiettivo di questo scherzetto è dimostrare il TFA per cambiamenti di coordinate, ovvero
    \[\int\limits g \de \L^n = \int\limits (g \circ f)\cdot J_f \de \L^n  \]
    Ma c'è un problema: noi abbiamo dimostrato un risultato dalla forma leggermente diversa, ovvero 
    \[\int\limits g \de f\mu = \int\limits g\circ f \de \mu \]
    Osservando il TFA ci aspettiamo che la nostra $\de f\mu$ corrisponda a $J_f \de \L^n$, dunque dobbiamo fare un piccolo giretto usando la biettività di $f$:
    \[\int\limits g \de \lambda = \int\limits g \circ f \circ f^{-1} \de \lambda = \int\limits g \circ f \de f^{-1}\lambda\]
    In questo modo ci basta riuscire a far corrispondere $J_f \de \L^n$ a $\de f^{-1} \L^n$
\end{remark}

\pagebreak
\section{Derivata di Radòn-Nikodym}

\begin{theorem}{Teorema di Radòn-Nikodym}{Radòn-Nikodym}
    Sia $(X,\mc{A})$ uno spazio misurabile e siano $\nu, \mu$ misure su $(X,\mc{A})$ tali che $\mu$ sia $\sigma$-finita e $\nu$ sia assolutamente continua rispetto a $\mu$. Allora esiste una funzione $\mc{A}$-misurabile $f: X \to [0,+\infty[$ tale che per ogni $E \in \mc{A}$ si abbia
    \[\nu(A) = \int\limits_A f \de \mu \]
    E per una funzione $\nu$-integrabile $g : (X,\mc{A},\nu)\to \R$ vale
    \[\int\limits g \de \nu = \int\limits g \cdot f \de \mu\]
\end{theorem}
\begin{definition}{Derivata di Radòn-Nikodym}{derivata di Radòn-Nikodym}
    Nella situazione precedente, la funzione $f$ si dice \bemph{derivata di Radòn-Nikodym} di $\nu$ rispetto a $\mu$ e si indica con
    \[f = \frac{\de \nu}{\de \mu}\]
\end{definition}

% \begin{corollary}{CONGETTURA Di Banach-Caccioppoli e Radòn-Nikodym}{}
%     Sia $(X,\mc{A},\mu, ||\cdot||)$ uno spazio di misura di Banach, sia $\mu$ una misura di Radòn invariante per isometrie e sia $f : (X,||\cdot||)\to(X,||\cdot||)$ una contrazione biettiva.\\
%     Allora $f^{-1}\mu$ soddifa le ipotesi del teorema \ref{th:Radòn-Nikodym} e $||f||$ è la sua derivata di R-N rispetto a $\mu$.
%     \begin{proof}
%         Sia $0<L<1$ la costante di Lipschitz di $f$ e sia $A\in \mc{A}$ un insieme di misura nulla.\\
%         Abbiamo che $f^{-1}\mu(A) = \mu(f(A))$. Osserviamo che $f(A)$ è isometricamente equivalente a un sottoinsieme di $A$, dunque $f(A)$ ha misura nulla per monotonia di $\mu$. Abbiamo dunque che esiste una funzione $\mu$-integrabile $g : X \to [0,+\infty]$  tale che 
%         \[f^{-1}\mu(E) = \int\limits_E g\de \mu = \int g \chi_E \de \mu = \int\limits_* g \chi_E \de \mu = \sup\left\{ I_\mu (\varphi) : \varphi \in \Sigma_-(g\chi_E)\right\}\]
%         Combinando con la definizione di misura finale
%         \[f^{-1}\mu(E) = \mu (f(E)) = \int\limits_{f(E)}\de\mu\]
%     \end{proof}
% \end{corollary}

\begin{definition}{Funzioni R-N}{funzioni R-N}
    Siano $(X,\mc{A},\mu)$ e $(Y,\mc{B},\nu)$ due spazi con misure $\sigma$-finite.\\
    Una funzione $f:(X,\mc{A},\mu)\to(Y,\mc{B},\nu)$ si dice \bemph{funzione R-N} se:\begin{enumerate}
        \item $f$ è misurabile
        \item Per ogni $E \in \mc{B}$ tale che $\nu(E) = 0$ si ha $\mu(f^{-1}(E))=0$
    \end{enumerate}    
\end{definition}
\begin{remark}{Categoria degli spazi con misure $\sigma$-finite e delle funzioni R-N}{}
    La classe degli spazi con misure $\sigma$-finite con la classe delle funzioni R-N e l'usuale composizione di funzioni è una categoria, che chiamiamo $\catname{Mea}_{R-N}$.
    \begin{proof}
        Controlliamo le varie proprietà:
        \begin{itemize}
            \item Sia $(X,\mc{A},\mu)$ uno spazio con misura $\sigma$-finita. La funzione identità $\id{X}$ è evidentemente una funzione R-N.
            \item Siano $f:(X,\mc{A},\lambda)\to (Y,\mc{B},\mu)$ e $g:(Y,\mc{B},\mu)\to(Z,\mc{C},\nu)$ due funzioni R-N. Notiamo che per ogni $E \in \mc{C}$ tale che $\nu(E) = 0$ si ha $(g\circ f)^{-1} = f^{-1}(g^{-1}(E))$ e $\mu(g^{-1}(E)) = 0$, dunque $\lambda((g\circ f)^{-1}(E))=0$.
            \item La composizione eredita l'associatività dalla composizione di funzioni in $\setcat$.
        \end{itemize}
    \end{proof}
\end{remark}

% \begin{proposition}{}{}
%     Sia $(X,\mc{A},\mu)$ uno spazio con misura $\sigma$-finita e $f:(X,\mc{A},\mu)\to X$ tale che $f:(X,\mc{A},\mu)\to(X,f\mc{A},f\mu)$ sia una funzione R-N.
%     \begin{proof}
%         Applicando il teorema \ref{th:Radòn-Nikodym} sappiamo che esiste $g:(X,\mc{A},\mu)\to [0,+\infty[$ tale che
%         \[f\mu(E) = \int\limits_E g \de \mu = \int\limits_{f^{-1}(E)}\de \mu\]
%     \end{proof}
% \end{proposition}

\begin{proposition}{Derivata di R-N per Lipschitziane}{R-N-Lipschitz}
    Sia $(X,d,\mu)$ uno spazio metrico di dimensione $n \in \Z_+$ con una misura $\mu$ di Radòn (rispetto alla $\sigma$-algebra Boreliana indotta dalla metrica) invariante per traslazioni (ovvero, $\mu(B_r(x)) = \mu(B_r(y))$ per ogni $x,y$ in $X$)\footnote{Onestamente non so se questa "uniformità" vada codificata come una proprietà dello spazio o della misura, dato che il nostro fine è quello di applicarlo alla misura di Lebesgue su $\R^n$ non ci poniamo troppi problemi in quanto $\R^n$ è tutto piatto e $\L^n$ è invariante per traslazioni.} e sia $F:X \to X$ una funzione biettiva di Lipschitz con costante di Lipschitz $L>0$.\\
    Allora $F^{-1}\mu \ll \mu$ e $L^n$ e la derivata di Radòn-Nikodym di $F^{-1}\mu$ rispetto a $\mu$ è maggiorata $\mu$-quasi ovunque da $L^n$.
    \begin{proof}
        È sufficiente dimostrarlo sulle palle aperte, dato che queste costituiscono una base della topologia e dunque della $\sigma$-algebra Boreliana.\\
        Per ogni $r>0$ e ogni $x \in X$ abbiamo che  $F(B_r(x)) \subset B_{Lr}(F(x))$ che implica $\mu(F(B_r(x)))\le \mu(B_{Lr}(F(x))) = L^n\mu(B_r(F(x)))$ il che implica che per ogni insieme, $F^{-1}(E) \le \mu(E)$, dunque sappiamo che deve esistere $g: (X,d,\mu)\to [0,+\infty[$ tale che
        \[F^{-1}\mu(E) = \int\limits_E g \de \mu\]
        Ancora una volta lavoriamo sulle palle
        \[\forall r>0,\forall x \in X, 0\le \int\limits_{B_r(x)} g(y)\de \mu(y) \le \int\limits_{B_r(x)} L^n \de\mu(y) \Rarr g \le_\mu L^n \]
    \end{proof}
\end{proposition}

\pagebreak
\section{Il viaggio verso il TFA}

Cercheremo di dimostrare il TFA per cambiamenti di coordinate \bemph{lineari} con la speranza di estendere questo ragionamento a cambiamenti di coordinate \bemph{differenziabili}, ovvero localmente lineari. Per fare questo, ci permetteremo di sostituire i plurirettangoli nella definizione della misura di Lebesgue ai pluriparallelogrammi

\begin{lemma}{Misura indotta da un cambiamento di coordinate lineare}{misura indotta lineare}
    Sia $F: (\R^n, \M_\L, \L^n) \to \R^n$ una mappa lineare invertibile.\\
    Allora $F^{-1}\L^n = |\det F|\cdot \L^n$ e dunque
    \[F^{-1}\L^n(E) = \int\limits_E |\det F|\de \L^n\]
    \begin{proof}
        Sia $E \in F\M_\L$. Per definizione di misura indotta, abbiamo che $F^{-1}\L^n(E) = \L^n(F(E))$ e che, come visto nel corso di Geometria A è uguale a $|\det F|\cdot \L^n(E)$.
    \end{proof}
\end{lemma}

\begin{theorem}{TFA per cambiamenti di coordinate lineari}{}
    Sia $F : \R^n \to \R^n$ una mappa lineare invertibile e sia $g: \R^n \to \R$ una funzione $\L^n$-integrabile. Vale il seguente fatto:
    \[\int\limits g \de \L^n = \int\limits (g \circ F) \cdot |\det F| \de \L^n \] 
\end{theorem}

% \begin{lemma}{NON SO SE SERVA DAVVERO}{}
%     Sia $E\subset \R^n$ un insieme misurabile e per $r>0$ sia $\{B_r(x_i)\}_{i \in I_r}$ un suo ricoprimento numerabile di palle aperte. Vale
%     \[\L^n(E) \le \sum_{i \in I_r} \L^n(B_r(x_i)) \quad \text{e in particolare}\quad \L^n(E) = \lim_{r \to 0} \sum_{i \in I_r} \L^n(B_r(x_i))\]
%     \begin{proof}
%         Segue dal teorema di Lindelof e dalla compatibilità tra $\H^n$ e $\L^n$
%     \end{proof}
% \end{lemma}

\begin{theorem}{Derivata R-N di una misura finale di diffeomorfismi}{R-N per misure finali di diffeomorfismi}
    Sia $\varphi : (\R^n, \M_\L, \L^n) \to \R^n$ un diffeomorfismo locale e sia $E \subset \R^n$ un aperto. Allora 
    \[\varphi^{-1}\L^n(E) = \int\limits_E|\det D_\varphi|\de \L^n\]
    Equivalentemente
    \[\frac{\de \varphi^{-1} \L^n}{\de \L^n} = |\det D_\varphi|\]
    Nel senso della definizione \ref{def:derivata di Radòn-Nikodym} della derivata di Radòn-Nikodym.
    \begin{proof}
        Il fatto che $\varphi^{-1}\L^n \ll \L^n$ segue dalla proposizione \ref{prop:R-N-Lipschitz}, infatti se $\varphi$ è un diffeomorfismo è almeno localmente lipschitziana e in ogni insieme limitato $V$ ha costante di Lipschitz $\sup_V |\det D_\varphi|$.\\
        Poniamo $|\det D_\varphi(x)| =: J(x)$.\\
        Sia $E\subset \R^n$ un aperto. Localmente la trasformazione $\varphi$ agisce come una trasformazione lineare $D_\varphi$, dunque in intorni $V_i$ sufficientemente piccoli di punti $x_i\in E$ indicizzati su un insieme numerabile $I$ applichiamo il lemma \ref{lemma:misura indotta lineare} e abbiamo $\varphi^{-1} \L^n \sim D_\varphi \L^n = J(x)\cdot \L^n$. Dunque posti possiamo scrivere
        \[\varphi^{-1}\L^n(E) = \sum_{i \in I} \int\limits_{V_i} J(x_i)\de \L^n = \sum_{i \in I} \int\limits_E J(x_i) \chi_{V_i}(y)\de \L^n(y)\]
        Facendo una \textit{mossa alla Gottinga} riconosciamo una regolarità sufficiente ad applicare uno strano genere di integrale di Riemann rendendo sempre più piccoli i nostri intorni aumentando il loro numero e otteniamo
        \[\varphi^{-1}\L^n(E) = \int\limits_E J \de \L^n = \int\limits_E |\det D_\varphi|\de \L^n\]
    \end{proof} 
\end{theorem}

\begin{theorem}{TFA}{}
    Sia $\varphi : \R^n \to \R^n$ un diffeomorfismo locale e $g : \R^n \to \R$ una funzione $\L^n$-integrabile. Vale
    \[\int\limits g \de \L^n = \int (g \circ \varphi)\cdot |\det D_\varphi|\de \L^n\]
    \begin{proof}
        La dimostrazione è banale combinando i non banali teoremi \ref{th:integrazione indotta}, \ref{th:Radòn-Nikodym} e \ref{th:R-N per misure finali}.
    \end{proof}
\end{theorem}

\printbibliography

\end{document}