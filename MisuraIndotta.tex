\documentclass{article}
\usepackage{cianatex}

\title{TFA per cambiamenti di coordinate}
\author{Filippo $\L$. Troncana}
\date{A.A. 2023/2024}

\begin{document}

\maketitle

\section{Misura indotta e $\sigma$-algebra finale}

\begin{definition}{$\sigma$-algebra finale}{sigma-algebra finale}
    Sia $(X,\mc{A},\mu)$ uno spazio con misura, sia $Y$ un insieme e sia $f:X\to Y$ una funzione biettiva.\\
    La $\sigma$\bemph{-algebra} finale indotta da $f$ rispetto a $\mc{A}$ è la famiglia
    \[f\mc{A} := \{E \in 2^Y : f^{-1}(E) \in \mc{A}\}\]
\end{definition}
\begin{remark}{}{}
    La $\sigma$-algebra finale di $f$ rispetto a $\mc{A}$ è la più grande $\sigma$-algebra $\Sigma$ tale che $f : (X,\mc{A})\to (Y,\Sigma)$ sia misurabile.
    \begin{proof}
        Sia $\Sigma \subset 2^Y$ tale che $f : (X,\mc{A})\to (Y,\Sigma)$ sia misurabile. Per definizione di funzione misurabile, abbiamo che per ogni $E \in \Sigma$, abbiamo che $f^{-1}(E)\in \mc{A}$, dunque $\Sigma\subset f\mc{A}$.
    \end{proof}
\end{remark}

\begin{definition}{Misura esterna indotta}{misura esterna indotta}
    Siano $X$ e $Y$ due insiemi, sia $\mu$ una misura esterna su $X$ e sia $f:X\to Y$ una funzione biettiva.\\
    La \bemph{misura indotta} da $f$ rispetto a $\mu$ è la funzione
    \[f\mu : 2^Y \to [0,+\infty] \quad \text{con} \quad f\mu(E):= \mu(f^{-1}(E)) \]
\end{definition}
\begin{proposition}{}{misura esterna indotta}
    $f\mu$ è una misura esterna su $Y$.
    \begin{proof}
        Dimostriamo i tre assiomi di misura esterna.\begin{enumerate}
            \item $f^{-1}(\varnothing) = \varnothing \Rarr f\mu(\varnothing) = 0$.
            \item Siano $E \subset F \subset Y$, allora $f^{-1}(E)\subset f^{-1}(F)$, dunque la monotonia di $f\mu$ segue dalla monotonia di $\mu$.
            \item Siano $A,B \subset Y$, allora $f^{-1}(A\cup B)= f^{-1}(A) \cup f^{-1}(B) $ e la "disuguaglianza triangolare" segue da quella di $\mu$
        \end{enumerate}
    \end{proof}
\end{proposition}
\begin{proposition}{}{misurabili indotti}
    Se $f\mu$ è la misura indotta da $f$ rispetto a $\mu$, allora $\M_{f\mu} = f \M_\mu$.
    \begin{proof}
        TODO
    \end{proof}
\end{proposition}

\section{Integrazione indotta}

La situazione che studiamo in questa sezione è la seguente

\[\begin{tikzcd}
	{(X,\mc{A},\mu)} && {(Y,f\mc{A},f\mu)} \\
	\\
	&& {(\R,\M_\L,\L)}
	\arrow["f"{description}, from=1-1, to=1-3]
	\arrow["{g \circ f}"{description}, from=1-1, to=3-3]
	\arrow["g"{description}, from=1-3, to=3-3]
\end{tikzcd}\]

\end{document}