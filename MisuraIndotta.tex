\documentclass{article}
\usepackage{cianatex}

\title{TFA per cambiamenti di coordinate}
\author{Filippo $\L$. Troncana}
\date{A.A. 2023/2024}

\begin{document}

\maketitle

\section{Misure e $\sigma$-algebre indotte}

\begin{definition}{$\sigma$-algebra finale}{sigma-algebra finale}
    Sia $(X,\mc{A})$ uno spazio misurabile, sia $Y$ un insieme e sia $f:X\to Y$ una funzione suriettiva.\\
    La $\sigma$\bemph{-algebra} finale indotta da $f$ rispetto a $\mc{A}$ è la famiglia
    \[f\mc{A} := \{E \in 2^Y : f^{-1}(E) \in \mc{A}\}\]
\end{definition}
\begin{remark}{}{}
    La $\sigma$-algebra finale di $f$ rispetto a $\mc{A}$ è la più grande $\sigma$-algebra $\Sigma$ tale che $f : (X,\mc{A})\to (Y,\Sigma)$ sia misurabile.
    \begin{proof}
        Sia $\Sigma \subset 2^Y$ tale che $f : (X,\mc{A})\to (Y,\Sigma)$ sia misurabile. Per definizione di funzione misurabile, abbiamo che per ogni $E \in \Sigma$, abbiamo che $f^{-1}(E)\in \mc{A}$, dunque $\Sigma\subset f\mc{A}$.
    \end{proof}
\end{remark}

\begin{definition}{Misura esterna indotta}{misura esterna indotta}
    Siano $X$ e $Y$ due insiemi, sia $\mu$ una misura esterna su $X$ e sia $f:X\to Y$ una funzione suriettiva.\\
    La \bemph{misura indotta} da $f$ rispetto a $\mu$ è la funzione
    \[f\mu : 2^Y \to [0,+\infty] \quad \text{con} \quad f\mu(E):= \mu(f^{-1}(E)) \]
\end{definition}
\begin{proposition}{}{misura esterna indotta}
    $f\mu$ è una misura esterna su $Y$.
    \begin{proof}
        Dimostriamo i tre assiomi di misura esterna.\begin{enumerate}
            \item $f^{-1}(\varnothing) = \varnothing \Rarr f\mu(\varnothing) = 0$.
            \item Siano $E \subset F \subset Y$, allora $f^{-1}(E)\subset f^{-1}(F)$, dunque la monotonia di $f\mu$ segue dalla monotonia di $\mu$.
            \item Siano $A,B \subset Y$, allora $f^{-1}(A\cup B)= f^{-1}(A) \cup f^{-1}(B) $ e la subaddittività segue da quella di $\mu$
        \end{enumerate}
    \end{proof}
\end{proposition}
\begin{proposition}{}{misurabili indotti}
    Se $f\mu$ è la misura indotta da $f$ rispetto a $\mu$, allora $\M_{f\mu} = f \M_\mu$.
    \begin{proof}
        TODO
    \end{proof}
\end{proposition}

TODO: è possibile definire una duale $\sigma$-algebra iniziale e una misura iniziale richiedendo l'iniettività, ma per la nostra trattazione è sufficiente la versione finale.

\begin{lemma}{Isomorfismo di $\sigma$-algebre indotte}{isomorfismo di sigma-algebre indotte}
    Siano $(X,\mc{A})$ uno spazio misurabile, $Y$ un insieme e $f: (X,\mc{A})\to Y$ una funzione biettiva. Allora $F\mc{A} \cong \mc{A}$.
    \begin{proof}
        Banale dimostrazione di insiemistica.
    \end{proof}
\end{lemma}

\section{Integrazione indotta}

La situazione che studiamo in questa sezione è la seguente

\[\begin{tikzcd}
	{(X,\mc{A},\mu)} && {(Y,f\mc{A},f\mu)} \\
	\\
	&& {(\R,\M_\L,\L)}
	\arrow["f"{description}, from=1-1, to=1-3]
	\arrow["{g \circ f}"{description}, from=1-1, to=3-3]
	\arrow["g"{description}, from=1-3, to=3-3]
\end{tikzcd}\]

\begin{theorem}{Integrazione indotta}{integrazione indotta}
    Sia $(X,\mc{A},\mu)$ uno spazio con misura, sia $Y$ un insieme, sia $f:X\to Y$ una funzione biettiva e sia $g:(Y,f\mc{A},f\mu)\to (\R,\B(\R),\L^1)$ una funzione $f\mc{A}$-misurabile.\\
    Allora $g$ è $f\mu$-integrabile se e solo se $g\circ f$ è $\mu$-integrabile, e vale l'identità
    \[\int\limits g \de f\mu = \int\limits g\circ f \de \mu\]
    \begin{proof}
        Assumiamo che $g$ sia $f\mu$-integrabile. Allora vale
        \[\int\limits g \de f\mu = \int\limits_* g \de f\mu = \sup\left\{I_{f\mu}(\varphi) : \varphi \in \Sigma_-(g)\right\}=\sup\left\{ \sum_i a_i f\mu(\varphi^{-1}(\{a_i\})) : \varphi \in \Sigma_-(g) \right\} =\]
        \[ = \sup\left\{ \sum_i a_i \mu(f^{-1}(\varphi^{-1}(\{a_i\}))) : \varphi \in \Sigma_-(g) \right\} = \sup\left\{ \sum_i a_i \mu((\varphi \circ f)^{-1}(\{a_i\})) : \varphi \circ f \in \Sigma_-(g\circ f) \right\}\]
        \[\text{con } \psi := \varphi \circ f, \quad \int\limits_* g \de f\mu =\sup\left\{ I_\mu(\psi) : \psi \in \Sigma_-(g\circ f) \right\} = \int\limits_* g \circ f \de \mu\]
        La dimostrazione è assolutamente analoga per l'integrale superiore e nella direzione opposta assumendo l'integrabilità di $g\circ f$. Le varie uguaglianze seguono dalla biettività di $f$.
    \end{proof}
\end{theorem}

\begin{remark}{Girotondone per il TFA}{}
    L'obiettivo di questo scherzetto è dimostrare il TFA per cambiamenti di coordinate, ovvero
    \[\int\limits g \de \L^n = \int\limits (g \circ f)\cdot J_f \de \L^n  \]
    Ma c'è un problema: noi abbiamo dimostrato un risultato dalla forma leggermente diversa, ovvero 
    \[\int\limits g \de f\mu = \int\limits g\circ f \de \mu \]
    Osservando il TFA ci aspettiamo che la nostra $\de f\mu$ corrisponda a $J_f \de \L^n$, dunque dobbiamo fare un piccolo giretto usando la biettività di $f$:
    \[\int\limits g \de \lambda = \int\limits g \circ f \circ f^{-1} \de \lambda = \int\limits g \circ f \de f^{-1}\lambda\]
    In questo modo ci basta riuscire a far corrispondere $J_f \de \L^n$ a $\de f^{-1} \L^n$
\end{remark}

\section{Derivata di Radòn-Nikodym}

\begin{theorem}{Teorema di Radòn-Nikodym}{Radòn-Nikodym}
    Sia $(X,\mc{A})$ uno spazio misurabile e siano $\nu, \mu$ misure su $(X,\mc{A})$ tali che $\mu$ sia $\sigma$-finita e $\nu$ sia assolutamente continua rispetto a $\mu$. Allora esiste una funzione $\mc{A}$-misurabile $f: X \to X$ tale che per ogni $E \in \mc{A}$ si abbia
    \[\nu(A) = \int\limits_A f \de \mu \]
    E per una funzione $\nu$-integrabile $g : (X,\mc{A},\nu)\to \R$ vale
    \[\int\limits g \de \nu = \int\limits g \cdot f \de \mu\]
\end{theorem}
\begin{definition}{Derivata di Radòn-Nikodym}{derivata di Radòn-Nikodym}
    Nella situazione precedente, la funzione $f$ si dice \bemph{derivata di Radòn-Nikodym} di $\nu$ rispetto a $\mu$ e si indica con
    \[f = \frac{\de \nu}{\de \mu}\]
\end{definition}

\begin{corollary}{CONGETTURA Di Banach-Caccioppoli e Radòn-Nikodym}{}
    Sia $(X,\mc{A},\mu, ||\cdot||)$ uno spazio di misura di Banach, sia $\mu$ una misura di Radòn invariante per isometrie e sia $f : (X,||\cdot||)\to(X,||\cdot||)$ una contrazione biettiva.\\
    Allora $f^{-1}\mu$ soddifa le ipotesi del teorema \ref{th:Radòn-Nikodym} e $||f||$ è la sua derivata di R-N rispetto a $\mu$.
    \begin{proof}
        Sia $0<L<1$ la costante di Lipschitz di $f$ e sia $A\in \mc{A}$ un insieme di misura nulla.\\
        Abbiamo che $f^{-1}\mu(A) = \mu(f(A))$. Osserviamo che $f(A)$ è isometricamente equivalente a un sottoinsieme di $A$, dunque $f(A)$ ha misura nulla per monotonia di $\mu$. Abbiamo dunque che esiste una funzione $\mu$-integrabile $g : X \to [0,+\infty]$  tale che 
        \[f^{-1}\mu(E) = \int\limits_E g\de \mu = \int g \chi_E \de \mu = \int_* g \chi_E \de \mu = \sup\left\{ I_\mu (\varphi) : \varphi \in \Sigma_-(g\chi_E)\right\}\]
        Combinando con la definizione di misura finale
        \[f^{-1}\mu(E) = \mu (f(E)) = \int\limits_{f(E)}\de\mu\]
    \end{proof}
\end{corollary}

\section{Il viaggio verso il TFA}

Cercheremo di dimostrare il TFA per cambiamenti di coordinate \bemph{lineari} con la speranza di estendere questo ragionamento a cambiamenti di coordinate \bemph{differenziabili}, ovvero localmente lineari. Per fare questo, ci permetteremo di sostituire i plurirettangoli nella definizione della misura di Lebesgue ai pluriparallelogrammi

\begin{lemma}{Misura indotta da un cambiamento di coordinate lineare}{misura indotta lineare}
    Sia $F: (\R^n, \M_\L, \L^n) \to \R^n$ una mappa lineare invertibile.\\
    Allora $F^{-1}\L^n = |\det F|\cdot \L^n$ e dunque
    \[F^{-1}\L^n(E) = \int\limits_E |\det F|\de \L^n\]
    \begin{proof}
        Sia $E \in F\M_\L$. Per definizione di misura indotta, abbiamo che $F^{-1}\L^n(E) = \L^n(F(E))$ e che, come visto nel corso di Geometria A è uguale a $|\det F|\cdot \L^n(E)$.
    \end{proof}
\end{lemma}

\begin{theorem}{TFA per cambiamenti di coordinate lineari}{}
    Sia $F : \R^n \to \R^n$ una mappa lineare invertibile e sia $g: \R^n \to \R$ una funzione $\L^n$-integrabile. Vale il seguente fatto:
    \[\int\limits g \de \L^n = \int\limits (g \circ F) \cdot |\det F| \de \L^n \] 
\end{theorem}

\begin{lemma}{NON SO SE SERVA DAVVERO}{}
    Sia $E\subset \R^n$ un insieme misurabile e per $r>0$ sia $\{B_r(x_i)\}_{i \in I_r}$ un suo ricoprimento numerabile di palle aperte. Vale
    \[\L^n(E) \le \sum_{i \in I_r} \L^n(B_r(x_i)) \quad \text{e in particolare}\quad \L^n(E) = \lim_{r \to 0} \sum_{i \in I_r} \L^n(B_r(x_i))\]
    \begin{proof}
        Segue dal teorema di Lindelof e dalla compatibilità tra $\H^n$ e $\L^n$
    \end{proof}
\end{lemma}

\begin{theorem}{Derivata R-N di una misura finale}{R-N per misure finali}
    Sia $\varphi : (\R^n, \M_\L, \L^n) \to \R^n$ un diffeomorfismo locale e sia $E \subset \R^n$ un aperto. Allora 
    \[\varphi^{-1}\L^n(E) = \int\limits_E|\det D_\varphi|\de \L^n\]
    Equivalentemente
    \[\frac{\de \varphi^{-1} \L^n}{\de \L^n} = |\det D_\varphi|\]
    Nel senso della definizione \ref{def:derivata di Radòn-Nikodym} della derivata di Radòn-Nikodym.
    \begin{proof}
        Poniamo $|\det D_\varphi(x)| =: J(x)$.\\
        Sia $E\subset \R^n$ un aperto. Localmente la trasformazione $\varphi$ agisce come una trasformazione lineare $D_\varphi$, dunque in intorni $V_i$ sufficientemente piccoli di punti $x_i\in E$ indicizzati su un insieme numerabile $I$ applichiamo il lemma \ref{lemma:misura indotta lineare} e abbiamo $\varphi^{-1} \L^n \sim D_\varphi \L^n = J(x)\cdot \L^n$. Dunque posti possiamo scrivere
        \[\varphi^{-1}\L^n(E) = \sum_{i \in I} \int\limits_{V_i} J(x_i)\de \L^n \sum_{i \in I} \int\limits_E J(x_i) \chi_{V_i}(y)\de \L^n(y)\]
        Facendo una \textit{mossa alla Gottinga} riconosciamo una regolarità sufficiente ad applicare uno strano genere di integrale di Riemann rendendo sempre più piccoli i nostri intorni e otteniamo
        \[\varphi^{-1}\L^n(E) = \int\limits_E J \de \L^n = \int\limits_E |\det D_\varphi|\de \L^n\]
    \end{proof} 
\end{theorem}

\begin{theorem}{TFA}{}
    Sia $\varphi : \R^n \to \R^n$ un diffeomorfismo locale e $g : \R^n \to \R$ una funzione $\L^n$-integrabile. Vale
    \[\int\limits g \de \L^n = \int (g \circ \varphi)\cdot |\det D_\varphi|\de \L^n\]
    \begin{proof}
        La dimostrazione è banale combinando i non banali teoremi \ref{th:integrazione indotta}, \ref{th:Radòn-Nikodym} e \ref{th:R-N per misure finali}.
    \end{proof}
\end{theorem}

\section{Delirio categorico}

Questa nostra costruzione può essere formalizzata come $\Phi : (X,\bullet) \times \catname{Sur}(X,\star) \to \catname{Meable}$ che mappa la coppia $((X,\mc{A}), f:X\to Y)$ in $(Y,f\mc{A})$, non credo sia più estensibile perchè è necessario che il supporto del primo spazio misurabile sia lo stesso insieme di partenza della suriezione

\end{document}