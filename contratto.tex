\documentclass{article}

\usepackage[depression]{cianatex}

\title{Contratto contro le lamentele}
\author{Filippo $\L$. Troncana}
\date{12 aprile 2025}

\begin{document}

\maketitle

\section{Risoluzioni e definizioni}

Il sottoscritto Filippo $\L$. Troncana, di seguito detto il sottoscritto, si impegna a \bemph{non} \textit{frignare} (termine definito più in basso) con le seguenti persone o in presenza di esse, di seguito dette \textit{partecipanti}:\begin{itemize}
    \item Borra Davide
    \item Brusco Jessica
    \item Calabri Matilde
    \item Peterle Peterle Alessia
    \item Sarzi Puttini Filippo, conte di Povo
\end{itemize}
dalla firma di questo contratto fino ad una sua eventuale abrogazione tramite un contratto analogo, il quale non potrà entrare in vigore prima del 29 luglio 2025.

\begin{definition}{Frignare}{frignare}
    Viene definita \bemph{frignare} l'espressione orale o scritta di uno stato di disagio fisico o spirituale. Sono da considerarsi eccezioni:\begin{itemize}
        \item Le lamentele riconducibili ad una forma di espressione artistica, il suonare una canzone triste ad esempio.
        \item Le lamentele accompagnate da un chiaro e materiale proposito di soluzione, per esempio "non mi piacciono le situazioni $\{X_\alpha\}_\alpha$ per i motivi $\{Y_\beta\}_\beta$, dunque farò le cose $\{Z_\gamma\}_\gamma$ per risolverle/migliorarle".
        \item Le lamentele riguardanti situazioni assolutamente fuori dal controllo del sottoscritto accompagnate dalla ricerca di un lato positivo.
    \end{itemize}
    E altri casi sottoposti alla discrezione dei partecipanti.
\end{definition}

Saranno permesse eccezioni nei seguenti casi, a discrezione dei partecipanti:\begin{itemize}
    \item Il manifestarsi di \textit{colpi di scena} sufficientemente rilevanti.
    \item La frustrazione a fine giornata per lo studio di materie particolarmente ostili come l'Analisi Numerica, ma solo se questo studio avrà avuto effettivamente luogo.
    \item Momenti in cui il disagio a cui si fa riferimento nella definizione \ref{def:frignare} dovesse essere giudicato assolutamente insostenibile dal sottoscritto.
    \item Momenti in cui il sottoscritto non dovesse essere nel pieno delle sue facoltà di intendere e di volere, ad esempio quando in uno stato di alterazione psicofisica dovuto all'utilizzo di sostanze psicoattive.
\end{itemize}

\section{Misure punitive}

Nel caso in cui il sottoscritto dovesse violare i termini di questo contratto, questi si impegna a offire un giro di spritz presso il bar di Povo (o cibo per valore equivalente presso la stessa sede) a tutti i partecipanti per ogni violazione del contratto.

\pagebreak

\section{Clausole e firme}

\paragraph{Clausola di emendamento:} Tutte le misure di questo contratto saranno successivamente emendabili da contratti analoghi, che una volta entrati in vigore assumerebbero la priorità a livello legale.

\paragraph{Il sottoscritto}:
\\
\bigskip
\\
Filippo $\L$. Troncana: \hrulefill

\paragraph{I partecipanti}:
\\
\bigskip
\\
Borra Davide: \hrulefill
\\
\bigskip
\\
Brusco Jessica: \hrulefill
\\
\bigskip
\\
Calabri Matilde: \hrulefill
\\
\bigskip
\\
Peterle Peterle Alessia: \hrulefill
\\
\bigskip
\\
Sarzi Puttini Filippo, conte di Povo: \hrulefill
\\
\bigskip
\\
\begin{flushright}
    Trento, 12 aprile 2025.
\end{flushright}

\end{document}