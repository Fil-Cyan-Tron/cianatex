\documentclass{article}
\usepackage{cianatex}

\title{Sui numeri binari quadrati}
\author{Troncana F.}
\date{}

\begin{document}

\maketitle

\section*{Introduzione}

Qualche tempo fa, Matilde mi ha presentato un'interessante congettura:

\begin{proposition}{Calabri I}{}
    Gli unici numeri della forma $a_0+10a_1+100a_2+...$ con $a_i \in \{0,1\}$ che sono quadrati perfetti sono della forma $10^{2k}$ per qualche $k \in \N$.
\end{proposition}

Dopo un infruttuoso attacco a forza di congurenze modulari, ho deciso di utilizzare tecniche a me più familiari, ovvero provare a ragionare in termini di polinomi.\\
Stabiliamo un pochino di linguaggio:

\begin{definition}{Numeri e polinomi binari}{binary}
    Un numero \bemph{binario in base} $\beta$, o numero $\beta$\bemph{-binario}, è un numero della forma
    \[\sum_{i=0}^n a_i \beta^i \qquad \text{con } a_i \in\{0,1\}\]
    Analogamente, un \bemph{polinomio binario} è un polinomio della forma
    \[\sum_{i=0}^n a_i x^i \qquad \text{con } a_i \in\{0,1\}\]
    Definiremo $2[x]$ l'insieme dei polinomi binari nella variabile $x$.
\end{definition}

Possiamo vedere che quindi la congettura di Matilde riguarda i numeri $10$-binari; procediamo a dimostrare il

\begin{theorem}{Teorema del treno per polinomi}{politreno}
    Sia $p \in \N[x]$ di grado $d$ tale che $p^2 \in 2[x]$.\\
    Allora $p = x^d$ se $d\ge 0$, altrimenti $p=0$.
    \proof 
    I casi $d\in\{-\infty, 0\}$ sono banali, assumiamo $d\ge 1$ e scriviamo per esteso $p$ e $p^2$:
    \[p = \sum_{i=0}^d a_i x^i, \qquad p^2 = \sum_{k=0}^{2d} b_k x^k \quad\text{dove}\quad b_k = \sum_{i=0}^k a_i a_{k-i} \quad\text{e}\quad \forall j > d, a_j=0\]
    Dimostriamo prima che $p \in 2[x]$ (lemma della locomotiva) e successivamente che $p = x^d$ (lemma della ferrovia):\begin{itemize}
        \item Dato che abbiamo assunto la binarietà di $p^2$, per ogni $k$ bisogna avere $b_k \in\{0,1\}$, dunque deve esistere al più un $i$ tale che $a_i a_{k-i} > 0$, poichè altrimenti $b_k$ sarebbe maggiore di $1$; inoltre, dato che l'unico caso in cui il prodotto di due numeri naturali è uguale a $1$ è quello in cui questi sono entrambi uno, deve valere $a_i = a_{k-i} = 1$, dunque per ogni $j$ deve valere $a_j \in \{0,1\}$ e dunque $p \in 2[x]$.
        \item Per l'unicità di $i$ vista nel punto precedente, deve valere anche $i = k-i$, ovvero $k = 2i$ e quindi $b_k$ può essere uguale a $1$ soltanto per $k$ pari. Scriviamo quindi
        \[ p^2 = \sum_{k=0}^d b_{2k} x^{2k} \quad \text{con}\quad b_{2k} = \sum_{i=0}^{2k}a_{i}a_{2k-i} \]
        Notiamo che abbiamo "gratis" $b_{2d} = 1$ e $b_0 = a_0^2$ e supponiamo per assurdo che per un qualche $0<k'<d$ si abbia $b_{2k'}=1$; questo significherebbe che $a_{k'}=1$ e che quindi $b_{d+k'}\neq 0$, e in particolare
        \[ b_{d+k'} = \sum_{i=0}^{d+k'} a_i a_{d+k'-i} = \underbrace{\sum_{i=0}^{k'-1} a_ia_{d+k'-i}}_{=0} + \underbrace{a_{k'}a_{d}}_{=1} + \underbrace{\sum_{i = k'+1}^{d-1} a_{i} a_{d+k'-i} }_{\ge 0} + \underbrace{a_{d}a_{k'}}_{=1} + \underbrace{\sum_{i=d+1}^{d+k'+1}a_i a_{d+k'-i}}_{\ge 0}\ge 2\]
        assurdo per ipotesi di binarietà di $p^2$, dunque dobbiamo concludere che $b_{2k'} = 0$ per ogni $0<k'<d$ e perciò $p^2 = x^{2d} + a_0$, ma dato che $x^{2d} +1$ è irriducibile in $\R[x]$ chiaramente lo è anche in $\N[x]$ e ovviamente in $2[x]$ e quindi $a_0 = 0$.
    \end{itemize}
    \qed
\end{theorem}

\begin{corollary}{Teorema del treno per numeri $\beta$-binari}{numetreno}
    Sia $p$ un quadrato perfetto $\beta$-binario.\\ 
    Allora $p$ è della forma $\beta^{2n}$ per qualche $n\in \N$ oppure $p=0$.
    \proof 
    Scriviamo $p$ in base $\beta$, vediamo le sue cifre come coefficienti di un polinomio $p(x) \in 2[x]$ e applichiamo il teorema del treno per polinomi, ottenendo che $p(x) = x^{2n}$ per qualche $n \in \N$ oppure $p(x)\equiv 0$.\\
    Valutiamo $p(\beta)$ per ottenere $p = \beta^{2n}$ oppure $p = 0$.
    \qed
\end{corollary}

\begin{corollary}{Calabri I}{}
    Gli unici numeri della forma $a_0+10a_1+100a_2+...$ con $a_i \in \{0,1\}$ che sono quadrati perfetti sono della forma $10^{2k}$ per qualche $k \in \N$.
    \proof 
    Applichiamo il teorema del treno per numeri $\beta$-binari nel caso $\beta=10$.
\end{corollary}

\end{document}