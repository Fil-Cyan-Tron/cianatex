\documentclass{article}
\usepackage{cianatex}

\title{Teoria della misura per la Geometria Algebrica}
\date{A.A. 2024/2025}
\author{Filippo $\L$ Troncana}

\newcommand\spec{\operatorname{Spec}}
\newcommand\spech{\spec _h}
\newcommand\specm{\spec_m}

\begin{document}

\maketitle

\begin{abstract}
    Durante lo studio della teoria della misura, generalmente si cercano misure $\sigma$-finite, boreliane, Borel-regolari, di Radon e così via, ma il problema è che la topologia di Zariski (o le topologie noetheriane in generale) non si comportano bene da questo punto di vista, in quanto tutti i sottoinsiemi sono compatti e tutti gli aperti sono densi, quindi i risultati di teoria della misura applicabili su queste topologie risultano... poco interessanti?
\end{abstract}

\begin{notation}
    $\P^n$ sarà lo spazio proiettivo su $\C^{n+1}$.\\
    $S$ sarà l'anello di $\C[x_0,...,x_n]$.
    $\spec(S)$ è l'insieme degli ideali primi, $\spech(S)$ di quelli omogenei e $\specm(S)$ di quelli massimali in $S$.
\end{notation}

\section{Richiami di geometria algebrica e teoria della misura}

\begin{definition}{Topologia di Zariski}{zariski top}
    Sia $\zeta \subset 2^{\P^n}$ la famiglia definita da 
    \[\zeta = \{X \subset \P^n : \exists I \in \spech(S) : X = Z(I)\}\]
    Questa è la famiglia dei chiusi di una topologia (definita per chiusi) detta \bemph{topologia di Zariski}.
\end{definition}

\end{document}