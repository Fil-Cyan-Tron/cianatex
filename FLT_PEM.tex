\documentclass{article}

\usepackage{cianatex}

\title{PEM}
\author{F. Troncana, sotto la supervisione del prof. R. Zunino}
\date{TBD}

\renewcommand\C{\mc{C}}
\newcommand\D{\mc{D}}

\begin{document}

\maketitle

\subsection{Nozioni fondamentali}

\begin{definition}{Categoria e dualità}{categoria}
    Una \bemph{categoria} $\C$ è una struttura munita di due classi: $\ob\C$ e $\hom\C$, dette rispettivamente \bemph{oggetti} (o elementi) e \bemph{morfismi} (o mappe o freccie) tali che \begin{itemize}
        \item Ogni morfismo $f \in \hom\C$ abbia associati due oggetti $A,B \in \ob\C$ detti rispettivamente \bemph{dominio} e \bemph{codominio} di $f$, che verrà indicato come $f : A\to B$.
        \item Per ogni coppia di morfismi $f:A\to B$ e $g : B\to C$ sia definita la loro \bemph{composizione}, ovvero un morfismo $g\circ f : A \to C$ (spesso indicato solo con $gf$).
        \item Per ogni oggetto $X \in \ob\C$ esista un morfismo $\id_X \in \hom\C$ detto \bemph{identità} di $X$ tale che per ogni morfismo $f:A\to B$ valga $\id_B f = f \id_A = f$.
        \item Per ogni terna di morfismi componibili $f,g,h \in \hom\C$, valga $h(gf) = (hg)f =: hgf$, ovvero la composizione sia \bemph{associativa}.
    \end{itemize}
    Fissati due oggetti $A,B \in \ob\C$ denoteremo con $\hom(A,B)$ o $\C(A,B)$ la collezione dei morfismi $A\to B$ di $\hom\C$.\\
    Per ogni categoria $\C$ è definita la sua \bemph{duale} (o opposta) $\C^\op$, i cui oggetti sono gli stessi di $\C$ e i cui morfismi sono quelli di $\C$ ma invertiti di direzione, ovvero ad ogni $f:A\to B$ in $\C$ corrisponde un $f^\op : B\to A$ in $\C^\op$.\\
    Una categoria $\C$ si dice:\begin{itemize}
        \item \bemph{Piccola} se la classe $\hom\C$ è un insieme.
        \item \bemph{Grande} se non è piccola.
        \item \bemph{Localmente piccola} se, una volta fissati due oggetti $X,Y \in \ob\C$, la classe $\hom(X,Y)$ è un insieme.
    \end{itemize}
\end{definition}
\begin{remark}{}{}
    Banalmente:\begin{itemize}
        \item Le identià sono uniche.
        \item La categoria duale è essenzialmente unica e vale $(\C^\op)^\op = \C$.
        \item Dato che $\ob\C$ inietta sempre in $\hom\C$ con la mappa $X\mapsto \id_X$, in generale la classe degli oggetti di una categoria non è una buona misura della sua grandezza.
    \end{itemize} 
    \proof 
    Dimostriamo solo l'ultimo punto con un esempio, gli altri sono banali. Sia $V$ la categoria formata da un unico oggetto $\bullet$ e la cui classe dei morfismi corrisponde alla classe dei cardinali, dove la composizione di due morfismi è data dalla loro somma come cardinali.
\end{remark}

Da ora in avanti, assumeremo sempre che le nostre categorie siano almeno localmente piccole per evitare \textit{troppi} problemi di fondazione (anche se come vedremo rimarranno delle grandi criticità)

\begin{definition}{Sapori di morfismi}{morfismi}
    Sia $\C$ una categoria e sia $f: A\to B$ un morfismo. Esso può dirsi:\begin{itemize}
        \item \bemph{Monomorfismo} (o monico o mono) se la precomposizione è iniettiva, ovvero per ogni coppia di morfismi $g_1, g_2 : C\to A$ vale $fg_1 = fg_2 \Rarr g_1=g_2$.
        \item \bemph{Epimorfismo} (o epico o epi) se la postcomposizione è iniettiva, ovvero se per ogni coppia di morfismi $g_1, g_2 : B\to C$ vale $g_1f = g_2f \Rarr g_1=g_2$.
        \item \bemph{Endomorfismo} (o endo) se $A=B$.
        \item \bemph{Sezione} (o split mono) se ha un'inversa sinistra, ovvero se esiste un morfismo $g:B\to A$ tale che $gf = \id_A$.
        \item \bemph{Retrazione} (o split epi) se ha un'inversa destra, ovvero se esiste un morfismo $g:B\to A$ tale che $fg = \id_B$.
        \item \bemph{Isomorfismo} (o iso) se ha un'inversa destra e sinistra. In particolare, $A$ e $B$ si dicono \bemph{isomorfi} (attraverso $f$) e li indicheremo con $f:A\cong_\C B$ omettendo usualmente $f$ o $\C$.
        \item \bemph{Automorfismo} (o auto) se è iso e endo.
    \end{itemize}    
\end{definition}
\begin{remark}{}{}
    Valgono le seguenti:\begin{itemize}
        \item Le sezioni sono mono.
        \item Le retrazioni sono epi.
        \item iso $\Harr$ split mono + epi $\Harr$ mono + split epi $\Rarr$ epi e mono,ma nell'ultimo caso non vale l'implicazione inversa.
        \item Tutte le inverse sono uniche quando esistono.
        \item Un mono è un epi nella categoria opposta e viceversa.
    \end{itemize}
    \proof 
    Forniamo solo due esempi di morfismi che sono epici e monici ma non isomorfismi (le altre verifiche sono assolutamente automatiche):\begin{itemize}
        \item Poniamoci nella categoria $\catname{Haus}$ degli spazi topologici di Hausdorff i cui morfismi sono le funzioni continue tra questi e consideriamo l'inclusione $\iota: [0,1]\cap \Q\inj[0,1]$ (entrambi con la topologia euclidea); questa è chiaramente mono in quanto iniettiva, ed è epi in quanto una funzione continua in $\catname{Haus}$ è completamente determinata dal suo valore su un sottospazio denso, ma non è iso dato che non è suriettiva.
        \item Consideriamo la categoria
        \[\begin{tikzcd}
        	\bullet && \bullet
    	    \arrow["\id", from=1-1, to=1-1, loop, in=145, out=215, distance=10mm]
        	\arrow["f", from=1-1, to=1-3]
    	    \arrow["\id", from=1-3, to=1-3, loop, in=325, out=35, distance=10mm]
        \end{tikzcd}\]
        Dato che a destra o sinistra possiamo comporre solo con l'identità, $f$ è sia mono che epi, ma non è iso in quanto non ha inversa.
    \end{itemize}
\end{remark}

\begin{definition}{Funtore}{functor}
    Siano $\C$ e $\D$ due categorie. Un \bemph{funtore covariante} $F: \C\to\D$ consiste in due mappe $F:\ob\C\to\ob\D$ e $F:\hom\C\to\hom\D$ che rispettino la composizione, ovvero:\begin{itemize}
        \item Per ogni morfismo $f :X\to Y$ vale $Ff: FX \to FY$.
        \item Per ogni oggetto $X \in \ob\C$ vale $F\id_X = \id_{FX}$.
        \item Per ogni coppia di morfismi componibili $f,g\ in \hom\C$ vale $F(gf) = FgFf$.
    \end{itemize}
    Un \bemph{funtore controvariante} da $\C$ a $\D$ è un funtore covariante $F: \C^\op\to\D$.\\
    Un funtore si dice:\begin{itemize}
        \item \bemph{Fedele} se per ogni $X,Y \in \ob\C$ la sua restrizione $F_{X,Y}:\C(X,Y)\to \D(FX,FY)$ è iniettiva.
        \item \bemph{Pieno} se per ogni $X,Y \in \ob\C$ la sua restrizione $F_{X,Y}:\C(X,Y)\to \D(FX,FY)$ è suriettiva.
        \item \bemph{Pienamente fedele} se è pieno e fedele.
        \item \bemph{Essenzialmente suriettivo sugli oggetti} se per ogni oggetto $Y \in \D$ esiste un oggetto $X \in \C$ tale che $FX\cong_\D Y$
    \end{itemize}
\end{definition}
\begin{proposition}{}{riflessione di isomorfismi}
    Sia $F:\C\to\D$ un funtore di qualsiasi varianza. Se $f:X\to Y$ è un isomorfismo, allora $Ff:FX\to FY$ è un isomorfismo.\\
    Se $F$ è pienamente fedele, vale anche l'implicazione inversa.
    \proof 
    La prima implicazione è banale, dunque assumiamo che $F$ sia pienamente fedele e $g:FX\to FY$ sia un isomorfismo; dato che la mappa $\varphi := F_{X,Y} : \C(X,Y) \to \D(FX,FY)$ è una biezione, esiste $f: X\to Y$ tale che $\varphi(f) = g$, dunque definiamo $f' := \varphi^{-1}(g^{-1})$. Dato che $F$ è un funtore, vale
    \[f'f = \varphi^{-1}(\varphi(f'f) )=\varphi^{-1}(\varphi(f')\varphi(f)) = \varphi^{-1}(g^{-1}g) = \varphi^{-1}(\id_{FX}) = \id_X\ ,\]
    Dimostrare che $f'$ è anche l'inversa destra di $f$ è assolutamente analogo, così come il caso controvariante.
    \qed
\end{proposition}

\section{Prodotti}

Ci sono (almeno) tre definizioni (quasi) equivalenti del prodotto in una categoria. La prima

\subsection{Proprietà universale}
\begin{definition}{Prodotto tramite proprietà universale}{prodotto pu}
    Sia $\C$ una categoria e siano $X,Y \in \ob\C$ due oggetti.\\
    Si dice \bemph{prodotto} di $X$ e $Y$ in $\C$ un oggetto $X\times Y$ munito di due morfismi $\pi_X : X\times Y\to X$ e $\pi_Y:X\times Y \to Y$ detti \bemph{proiezioni} tali che per ogni oggetto $Z$ e ogni coppia di morfismi $f:Z\to X$ e $g:Z\to Y$ esista un unico morfismo $f\times g: Z\to X\times Y$ tale che il seguente diagramma commuti:
    \[\begin{tikzcd}
	    && {X\times Y} \\
    	X &&&& Y \\
    	&& Z
    	\arrow["{\pi_X}"{description}, from=1-3, to=2-1]
    	\arrow["{\pi_Y}"{description}, from=1-3, to=2-5]
    	\arrow["{f\times g}"{description}, from=3-3, to=1-3]
    	\arrow["f"{description}, from=3-3, to=2-1]
    	\arrow["g"{description}, from=3-3, to=2-5]
    \end{tikzcd}\]
\end{definition}

\subsection{Categoria dei coni}

\begin{definition}{Cono}{cono}
    Sia $\C$ una categoria e sia $J$ un diagramma commutativo in $\C$.\\
    Si dice \bemph{cono} su $J$ un oggetto $C \in \C$ con una collezione di morfismi $\{\rho_j : C \to j\}_{j \in J}$ tale che per ogni morfismo $f : X\to Y$ in $J$, il seguente diagramma commuti:
    \[\begin{tikzcd}
	    & C \\
	    \\
	    X && Y
	    \arrow["{\rho_X}"{description}, from=1-2, to=3-1]
	    \arrow["{\rho_Y}"{description}, from=1-2, to=3-3]
	    \arrow["f"{description}, from=3-1, to=3-3]
    \end{tikzcd}\]
    La collezione dei coni su un diagramma $J$ e dei morfismi tra loro forma una categoria, detta $\cone(J)$: i suoi oggetti sono i coni $(C,\{\rho_j\}_{j \in J})$ e i suoi morfismi sono definiti da 
    \[ \hom((C,\{\rho_j\}_{j \in J}), (C',\{\rho_j'\}_{j \in J})) = \{m : C\to C' : \forall j \in J, \rho_j = \rho_j' m\}\]
    Questi sono detti morfismi \bemph{medianti}.
\end{definition}

\begin{definition}{Prodotto tramite coni}{prodotto coni}
    Sia $\C$ una categoria e siano $X,Y \in \ob\C$ due oggetti.
    Si dice \bemph{prodotto} di $X$ e $Y$ l'oggetto $X\times Y$ di $\C$ tale che $(X\times Y, \{\pi_X, \pi_Y\})$ sia l'oggetto iniziale di $\cone(\{X,Y\})$, ovvero della categoria dei coni sul diagramma discreto ${X,Y}$.
\end{definition}

\subsection{Aggiunzioni}

Prima di dare la prossima definizione, osserviamo che questa catena di biezioni è verificata (e naturale in $A$, $B$ e $C$)
\[\begin{aligned}
    \underbrace{\hom_\C(A, B\times C)}_{\C^\op \times \C\times\C \to \setcat} & \cong_\setcat \hom_\C(A,B)\times \hom_\C(A,C)\\
    & \cong_\setcat \hom_{\C^2}((A,A),(B,C))\\
    & \cong_\setcat \hom_{\C^2}(\Delta A,(B,C))
\end{aligned}\]
Dove $\Delta:\C\to\C^2$ è il funtore diagonale, che manda un oggetto $A$ nell'oggetto $(A,A)$ e un morfismo $f$ nel morfismo $(f,f)$: abbiamo ottenuto dunque che $\hom_\C(A, B\times C) \cong_\setcat \hom(\Delta A, (B,C))$, ovvero che interpretando il prodotto come un funtore $\times: \C^2\to\C$ questo è aggiunto destro al funtore diagonale, ovvero $\Delta \dashv \times$, quindi definiamo

\begin{definition}{Prodotto III}{prodotto adj}
    Sia $\C$ una categoria e siano $X,Y \in \ob\C$ due oggetti. 
    Si dice \bemph{prodotto} di $X$ e $Y$ l'immagine della coppia $(X,Y) \in \ob\C^2$ attraverso un funtore $\Pi$ aggiunto destro al funtore $\Delta$.
\end{definition}

Prima di procedere con questa definizione però dobbiamo dimostrare alcune cosine: innanzitutto ci serve che il prodotto sia almeno essenzialmente unico
\begin{lemma}{Essenziale unicità degli aggiunti}{}
    Siano $\C,\D$ due categorie e siano $F:\C\to\D$ e $G_1,G_2: \D\to\C$ funtori tali che $F\dashv G_1$ e $F\dashv G_2$ oppure $G_1\dashv F$ e $G_2\dashv F$.\\
    Allora $G_1\cong G_2$.
    \begin{proof}
        
    \end{proof}
\end{lemma}
\end{document}