\documentclass[openany]{book}
\usepackage{cianatex}
\addbibresource{variazioni.bib}


\title{Calcolo delle Variazioni}
\author{Filippo $\mc{L}$ Troncana\\dalle lezioni del prof. Marco Bonacini dell'omonimo corso per il corso di laurea in Matematica}
\date{A.A. 2024/2025}

\newcommand\di{\text{d}}

\begin{document}

\nocite{*}
\maketitle

\tableofcontents

\pagebreak

%Lezione 25/02/2025

\section*{Introduzione}

Il calcolo delle variazioni è quella branca della matematica che affronta il problema di trovare in una data famiglia (di funzioni, superfici, curve...) l'oggetto o gli oggetti che minimizzano una certa grandezza ad essi associata, ad esempio il problema della brachistocrona è uno degli esempi più classici

\subsection*{Esempi introduttivi}

\subsubsection*{Metodi classici: funzioni reali}

Supponiamo di avere una funzione $f:[a,b]\to\R$ della quale vogliamo trovare i punti di minimo. Se la nostra funzione è differenziabile su $]a,b[$ possiamo usare il teorema di Fermat che ci dà una condizione \textbf{necessaria ma non sufficiente} affinchè un punto $x_0 \in ]a,b[$ sia un punto di massimo, ovvero $f'(x_0) = 0$.\\
Se la nostra funzione è doppiamente differenziabile, possiamo ottenere un'altra condizione necessaria, ovvero che $f'(x_0) = 0$ e che $f''(x_0) \ge 0$; inoltre sempre lavorando sulla derivata seconda otteniamo quella che è una condizione \textbf{sufficiente ma non necessaria}, ovvero che $f'(x_0) = 0$ e $f''(x_0) > 0$.\\
Scartando l'ipotesi di doppia derivabilità, possiamo sostituirla con l'ipotesi di convessità, rendendo $f'(x_0) = 0$ una condizione sufficiente per la minimalità di $x_0$.\\
I metodi classici (o indiretti) si basano sulla generalizzazione di questo approccio a spazi di funzioni, come vediamo ora.

\subsubsection*{Metodi classici: integrale di Dirichlet}

Sia $\Omega\subset\R^n$ un aperto a chiusura compatta con frontiera $\partial\Omega$ regolare e sia $g : \partial\Omega\to\R$ una funzione continua. Definiamo il nostro spazio $X$ e il nostro funzionale $F : X\to\R$ come:
\[ X = \left\{ u \in \mc{C}^1(\bar{\Omega}) : u|_{\partial\Omega} = g \right\}, \qquad F(u) = \intop_\Omega \|\nabla u\|^2 \de \L^n, \qquad\text{inoltre assumiamo che esista } u_0 = \arg\min_{u \in X} F(u). \]
Analogamente a quanto visto per le funzioni reali, quali condizioni necessarie o sufficienti possiamo identificare per il nostro punto di minimo $u_0$? Ragionando sull'approccio del teorema di Fermat, possiamo formulare la condizione al primo ordine della nostra funzione reale come 
\[ 0 = f'(x_0) = \lim_{t\to 0} \frac{f(x_0+t) - f(x_0)}{t} \]
Consideriamo lo spazio $\mc{C}^1_c(\Omega)$ delle funzioni differenziabili a supporto compatto contenuto in $\Omega$ e per una $\varphi \in \mc{C}^1_c(\Omega)$ e $t \in \R$ definiamo la funzione $u_t := u_0 + t\varphi$, che appartiene a $X$ per ogni $t \in \R$\footnote{Banalmente, in quanto $\varphi|_{\partial\Omega} \equiv 0$}. Usiamo la nostra $\varphi$ a mo' di "vettore della base canonica" come facevamo in $\R^n$:
\[0 = \lim_{t\to 0} \frac{F(u_t) - F(u_0)}{t} = \lim_{t\to 0} \frac{1}{t} \left( \intop_\Omega \|\nabla u_t\|^2 \de \L^n - \intop_\Omega \|\nabla u_0\|^2 \de \L^n \right)\]
Sviluppando i quadrati e usando la linearità dell'integrale otteniamo
\[ \frac{1}{t} \left( \intop_\Omega \|\nabla u_t\|^2 \de \L^n - \intop_\Omega \|\nabla u_0\|^2 \de \L^n \right) =  \intop_\Omega \cancel{\frac{\|\nabla u_0\|^2}{t}} + \frac{2\cancel{t} \nabla u_0 \cdot \nabla \varphi}{\cancel{t}} + \frac{t^{\cancel{2}}\|\nabla \varphi\|^2}{\cancel{t}} -\cancel{\frac{\|\nabla u_0\|^2}{t}} \de \L^n \xrightarrow{t \to 0} 2\intop_\Omega \nabla u_0 \nabla \varphi \de\L^n.\]
Battezziamo questa quantità che abbiamo trovato \bemph{variazione prima di} $F$ \bemph{rispetto a} $\varphi$ \bemph{in} $u_0$ e la indichiamo con $\delta F(u_0,\varphi)$, sarà analoga alla nostra derivata direzionale; inoltre, se avessimo qualche ragione di assumere che $u_0$ sia anche $\mc{C}^2(\bar{\Omega})$ potremmo usare il teorema della divergenza per scrivere anche
\[ 0 = \delta F(u_0,\varphi) = \intop_{\Omega} \nabla u_0 \cdot \nabla\varphi \de\L^n = \intop_{\Omega}(-\Delta u)\varphi \de\L^n + \cancel{\intop_{\partial\Omega} \varphi \frac{\de u_0}{\de \nu} \de S } = 0\]
Sfruttando il lemma \ref{lem:lemma fondamentale cdv} otteniamo che
\[ \intop_{\Omega} (-\Delta u_0) \varphi \de\L^n = 0\ \forall \varphi \in \mc{C}^1_c(\Omega)  \Harr -\Delta u_0 = 0 \]
Perbacco! Assumendo che il nostro punto di minimo esista, abbiamo ottenuto che questo deve soddisfare il problema di Dirichlet
\[ \begin{cases} -\Delta u = 0 & \text{su }\Omega \\ u|_{\partial\Omega} = g \end{cases}\]
Ci sono due criticità tuttavia: abbiamo assunto tante cose belle sulla nostra $u_0$ (in primo luogo, che questa esista) e siamo arrivati a scrivere una PDE, oggetti che in generale non sono di facilissima trattazione e figuriamoci risoluzione. Per questo nel ventesimo secolo si sono sviluppati i cosiddetti metodi diretti.

\subsubsection*{Metodi diretti: teorema di Weierstrass}

Tornando all'esempio della nostra funzione $f:[a,b]\to \R$, potremmo ricordarci che abbiamo un teorema che ci garantisce l'esistenza del minimo assumendo semplicemente la continuità di $f$, ovvero il teorema di Weierstrass, la cui dimostrazione si riassume in questi step:\begin{enumerate}
    \item Sia $(x_n)_{n\in\N}$ una successione minimizzante, ovvero tale che $f(x_n)\to \inf_{[a,b]}f$
    \item L'intervallo $[a,b]$ è compatto, dunque esiste una sottosuccessione $(x_{n_k})_{k \in \N}$ che converge a $\hat{x} \in [a,b]$.
    \item Dato che $f$ è continua, $f(\hat{x}) = \inf_{[a,b]}f = \min_{[a,b]}f$.
\end{enumerate}
Notiamo che sarebbe bastata la semicontinuità inferiore di $f$, e che questo approccio dipende dalla topologia di $[a,b]$: i metodi diretti si basano proprio su questo, ovvero su una forma più generale del teorema di Weierstrass (sostituendo $[a,b]$ con uno spazio topologico sequenzialmente compatto) e scegliendo sulla nostra famiglia di oggetti la topologia adeguata.\\
Chiaramente abbiamo un piccolo trade-off: se la nostra topologia è molto fine (= tanti aperti), è facile dimostrare la continuità del nostro funzionale ma è difficile avere la compattezza della nostra famiglia; al contrario, con topologie meno fini abbiamo una compattezza più semplice da dimostrare ma una continuità più difficile, per questo è utile ridurre le ipotesi (ad esempio con la semicontinuità inferiore invece della continuità).

\begin{exercise}{Dimostrare che esiste una successione minimizzante}{successione minimizzante}
    Sia $X$ un insieme non vuoto e sia $f : X\to \R$ una funzione.\\
    Si mostri che esiste una successione $(x_n)_{n \in \N}$ in $X$ tale che $f(x_n)\to \inf_{X} f$.
    \solution 
    Sia $y_0 \in f(X)$. Se $y_0 = \inf_X f$, prendiamo una qualsiasi successione in $f^{-1}(y_0)$; altrimenti, e per ogni $n \in \N_{>0}$ prendiamo\footnote{Probabilmente c'è un modo per aggirare l'utilizzo di scelta dipendente ma non ho davvero voglia di pensarci.} $y_{n+1} \in f(X) \cap ]-\infty, y_n[$, fermandoci se dovessimo arrivare a $\inf_X f$.\\
    Per ogni $y_n$ scegliamo\footnote{Idem ma con scelta numerabile.} $x_n \in f^{-1}(y_n)$ e abbiamo ottenuto la nostra successione minimizzante.
    \solved
\end{exercise}

\subsubsection*{Cosa tratteremo?}

Durante il corso tratteremo per lo più problemi unidimensionali in forma integrale, ovvero:
\[ X = \left\{ u : [a,b]\to \R : u\text{ ammette una qualche } u' \right\}, \quad F(u) = \intop_{[a,b]} f(x,u(x), u'(x))\de\L^1(x), \quad\text{con } f: [a,b]\times\R^2\to\R \]
%Lezione 26/02/25
Il nostro funzionale $F$ spesso sarà chiamato \bemph{energia}, mentre la funzione $f$ avrà spesso come variabili $f(x,s,p)$ e sarà chiamata \bemph{lagrangiana}; le sue derivate parziali saranno indicate con $f_x, f_s, f_p$.

\subsection*{Ulteriori esempi introduttivi}

\subsubsection*{La brachistocrona}

\subsubsection*{Problema di Didone, o isoperimetrico}

\subsubsection*{Superfici di rivoluzione di area minima}

\subsubsection*{Problema di Plateau, o delle superfici minime}

\subsubsection*{Principio di minima azione}

Si veda l'esercizio \ref{exc:E-L per il moto}

\chapter{Metodi classici}

\section{Le equazioni di Eulero--Lagrange}
La situazione che analizziamo adesso è questa:

\begin{situation}{Eulero--Lagrange con estremi vincolati}{E-L estremi vincolati}
    \[ X = \{u \in \mc{C}^1([a,b]):u(a)=\alpha, u(b)=\beta\} \qquad \text{con }a,b,\alpha,\beta \in \R, a<b, \]
    \[ F(u) = \intop_a^b f(x,u(x),u'(x))\de x \qquad \text{con } f \in \mc{C}^1([a,b]\times\R\times\R)\]
    Con $u_0 \in X$ punto di minimo per $F$.
\end{situation}

Fissiamo una $\varphi \in \mc{C}_c^1(]a,b[)$ e definiamo per $t \in \R$ la funzione $u_t = u_0 + t\varphi$, che chiaramente appartiene a $X$ come visto nei primi esempi. Per ipotesi di minimalità, allora la funzione $g(t) := F(u_t)$ deve avere un punto stazionario in $t=0$. Se $g$ fosse $\mc{C}^1$ (cosa che vediamo sarà verificata grazie al prossimo teorema) potremmo applicare il teorema di Fermat per ottenere la condizione:
\[0 = g'(0) = \frac{\di}{\di t}F(u_0 + t \varphi)\bigg|_{t=0} = \frac{\di}{\di t} \intop_a^b f(x, u_0(x) + t\varphi(x), u_0'(x) + t\varphi'(x)) \de x \bigg|_{t=0}.\]

\begin{theorem}{Derivazione sotto il segno di integrale}{deriva sotto integrale}
    Sia $A \subset \R^n$ un insieme $\L^n$-misurabile, $I\subset \R$ un intervallo aperto e $h: A\times I \to \R$ una funzione tale che\begin{itemize}
        \item $\forall t \in I$, la mappa $x\mapsto h(x,t)$ è $\L^n$-integrabile;
        \item $\forall x \in A$, la mappa $t\mapsto h(x,t)$ è derivabile;
        \item Esiste una funzione $H : A\to\R$ maggiorante essenziale $\L^n$-integrabile della mappa $x\mapsto \frac{\partial}{\partial t} h(x,t)$ per ogni $t \in I$.
    \end{itemize}
    Allora vale
    \[ \frac{\di}{\di t}\intop_A h(x,t)\de \L^n(x) = \intop_A \frac{\partial}{\partial t} h(x,t) \de\L^n(x) .\]
\end{theorem}

Applicando il teorema alla nostra $h(x,t) = f(x,u_0(x)+t\varphi(x), u_0'(x) + t\varphi'(x))$ e facendo un po' di derivate otteniamo

\begin{definition}{Variazione prima}{variazione prima}
    Nella situazione \ref{situaz:E-L estremi vincolati}, la quantità
    \[\delta F(u, \varphi) := \intop_a^b f_s(x,u(x),u'(x))\varphi(x) + f_p(x,u(x),u'(x))\varphi'(x)\de x \]
    è detta \bemph{variazione prima} in $u$ del funzionale $F$ lungo $\varphi$.
\end{definition}

E quindi con quanto fatto finora abbiamo dimostrato 

\begin{proposition}{Equazione di Eulero--Lagrange debole}{E-L debole}
    Nella situazione \ref{situaz:E-L estremi vincolati}, se $u_0$ è un punto di minimo allora per ogni $\varphi \in \mc{C}_c^1(]a,b[)$ soddisfa l'\bemph{equazione di Eulero--Lagrange debole}:
    \[\delta F(u_0,\varphi) = 0 = \intop_a^b f_s(x,u(x),u'(x))\varphi(x) + f_p(x,u(x),u'(x))\varphi'(x)\de x\ .\]
    Una funzione che soddisfa l'equazione si dice \bemph{estremale}.
\end{proposition}

Adesso supponiamo anche di sapere\footnote{Per qualche motivo} che $f_p(x, u_0(x), u_0'(x))$ sia $\mc{C}^1([a,b])$ (ad esempio, se avessimo motivo di credere che $f$ e $u_0$ siano $\mc{C}^2$); integrando per parti otteniamo:
\[ 0 = \delta F(u_0,\varphi) = \intop_a^b f_s(x,u_0(x),u_0'(x))\varphi(x)\de x + \cancel{[f_p(x,u_0(x),u_0'(x))\varphi(x)]_a^b} - \intop_a^b \frac{\partial}{\partial x} [f_p(x, u_0(x), u_0'(x))]\varphi(x)\de x = \]
\[ = \intop_a^b \left[ f_s(x,u_0(x),u_0'(x)) - \frac{\partial}{\partial x} f_p(x, u_0(x), u_0'(x)) \right]\varphi(x)\de x = 0\ . \]
Adesso useremo un piccolo lemma, ovvero:

\begin{lemma}{Lemma fondamentale del calcolo delle variazioni}{lemma fondamentale cdv}
    Sia $v : [a,b]\to\R$ una funzione continua. Se per ogni $\varphi \in \mc{C}_c^1(]a,b[)$ vale
    \[\intop_a^b v(x)\varphi(x)\de x = 0,\]
    allora $v$ è identicamente nulla su $[a,b]$.
    \proof 
    Supponiamo per assurdo che esista $x_0 \in ]a,b[$\footnote{Se fosse sugli estremi varrebbe comunque un argomento assolutamente analogo.} tale che $v(x_0)>0$\footnote{WLOG}. Allora per continuità di $v_0$ esiste un intorno aperto $I$ tale che per ogni $x \in I$ vale $v(x) \ge v(x_0)/2$; prendo una $\varphi \in \mc{C}^1_c(]a,b[)$ tale che $\varphi > 0$ su $I$ e $\varphi = 0$ su $]a,b[\setminus I$, allora vale
    \[ 0 = \intop_a^b v(x) \varphi(x) \de x \ge \intop_I \underbrace{v(x)\varphi(x)}_{>0} \de x > 0, \]
    assurdo, dunque $v \equiv 0$.
    \qed
\end{lemma}
\begin{remark}{Lemma fondamentale del Calcolo delle Variazioni super saiyan}{lemma fondamentale cdv super}
    In realtà è sufficiente controllare le $\varphi$ in $\mc{C}_c^\infty$, anche se useremo questo risultato non lo dimostreremo.
\end{remark}

Allora possiamo applicare questo lemma alla relazione di prima per ottenere

\begin{theorem}{Equazioni di Eulero--Lagrange I}{equazioni di E-L I}
    Nella situazione \ref{situaz:E-L estremi vincolati}, vale la proposizione \ref{prop:E-L debole}.\\
    Inoltre, se $f_p \in \mc{C}^1([a,b]\times \R\times \R)$ vale anche
    \[ f_s(x,u_0(x),u_0'(x)) - \frac{\partial}{\partial x} f_p(x, u_0(x), u_0'(x)) = 0.\]
    Questa condizione si dice \bemph{equazione di Eulero--Lagrange forte}.
\end{theorem}

Vedremo nelle prossime pagine che in realtà queste ipotesi più forti ci sono sempre garantite.

\begin{exercise}{Eulero--Lagrange per il moto}{E-L per il moto}
    Sia $\gamma : [t_a,t_b]\to \R^3$ la legge oraria della traiettoria di un punto materiale di massa $m>0$ sottoposto a un campo di forze $E : \R^3 \to \R^3$ conservativo, ovvero tale che esista $V : \R^3\to \R$ tale che $E = \nabla V$.\\ Sappiamo che $\gamma$ soddisfa le equazioni del moto
    \[ m \gamma''(t) = E(\gamma(t)); \]
    definiamo il funzionale \bemph{azione} come
    \[S(\gamma) = \intop_{t_a}^{t_b} \frac{1}{2} m (\gamma'(t))^2 - V(\gamma(t)) \de t .\]
    Si mostri (supponendo adeguate regolarità) che i minimi di $S$ risolvono le equazioni del moto.
    \solution 
    Risolveremo il caso unidimensionale, in quanto il caso tridimensionale è semplicemente un sistema di tre casi unidimensionali indipendenti tra loro.\\
    Supponiamo che $\gamma$ sia un minimo di $S$ e osserviamo che la lagrangiana di $S$ è data da:
    \[ f(x,s,p) = \frac{1}{2}m p^2 - V(s) \Rarr\begin{cases} f_s(x,s,p) = \nabla V(s) = E(s) \\ f_p(x,s,p)= m p\end{cases}. \]
    Applicando il teorema \ref{th:equazioni di E-L I} abbiamo dunque 
    \[ E(\gamma(t)) - \frac{\partial}{\partial t} m \gamma'(t) = 0 \Harr E(\gamma(t)) = m \gamma''(t). \]
    \solved
\end{exercise}

\begin{example}{TODO}{}
    Poniamo
    \[ F(u) := \intop_0^{2\pi} (u'(x))^2 - (u(x))^2 \de x, \qquad X := \{ u \in \mc{C}^1([0,2\pi]) : u(0)=u(2\pi)=0 \}. \]
\end{example}

\begin{example}{TODO}{}
    Poniamo
    \[ F(u) := \intop_{-1}^1 u^2(x)\cdot(2x-u'(x))^2 \de x, \qquad X := \{ u \in \mc{C}^1([-1,1]):u(-1) = 0, u(1) = 1 \}.\]
\end{example}

%Lezione 04/03/25

\subsection{Lemma di Du Bois--Raymond}

Poniamoci sempre nella situazione \ref{situaz:E-L estremi vincolati} e fissiamo un pochino di notazione:

\[a(x) := f_s(x, u_0(x), u_0'(x)), \quad b(x) := f_p(x, u_0(x), u_0'(x)), \quad A(x) := \intop_a^x a(t)\de t \]
Che ci permettono di riscrivere la forma debole delle equazioni di Eulero--Lagrange come

\[\begin{aligned}
    0 & = \intop_a^b a(x)\varphi(x) - b(x)\varphi'(x)\de x = \\ & = \cancel{[A(x)\varphi(x)]_a^b} - \intop_a^b A(x)\varphi'(x) \de x + \intop_a^b b(x)\varphi'(x) \de x = \\ & = \intop_a^b [b(x)-A(x)]\varphi'(x) \de x = 0 \qquad \forall \varphi \in \mc{C}_c^1([a,b]).
\end{aligned}\]
Ora sfoderiamo un lemmino dal cilindro, ovvero il 

\begin{lemma}{Lemma di Du Bois--Raymond}{DB-R}
    Sia $v \in \mc{C}^0([a,b])$ tale che 
    \[ \intop_a^b v(x) \varphi'(x) \de x = 0 \qquad \forall \varphi \in \mc{C}_c^\infty(]a,b[).\]
    Allora $v$ è costante su $[a,b]$.
    \proof 
    Sia $\varphi$ come da ipotesi e $\psi = \varphi'$: dato che $\varphi$ è nulla negli estremi di $[a,b]$, l'integrale di $\psi$ su $[a,b]$ è nullo; allo stesso modo, integrando una $\psi$ che abbia integrale nullo su $[a,b]$ da $a$ a $x$ possiamo recuperare una sua primitiva $\varphi$ che soddisfi le ipotesi del lemma, dunque possiamo riformulare l'ipotesi equivalentemente come:
    \[ \intop_a^b v(x) \psi(x) \de x = 0 \qquad \forall \psi \in \mc{C}_c^\infty(]a,b[) : \intop_a^b \psi(x) \de x = 0\ . \]
    Fissiamo una $w \in \mc{C}_c^\infty(]a,b[)$ tale che il suo integrale su $[a,b]$ sia $1$ e data $w \in \mc{C}_c^\infty(]a,b[)$ poniamo
    \[ \psi(x) := \varphi(x) - w(x) \intop_a^b\varphi(t)\de t\ ,\]
    Integrando per parti vediamo che $\psi$ soddisfa le nostre ipotesi, dunque analogamente vediamo che 
    \[ 0 = \intop_a^b v(x)\psi(x)\de x = \intop_a^b v(x)\varphi(x) \de x- \intop_a^b v(x)w(x)\intop_a^b \varphi(t)\de t \de x = \intop_a^b \left[v(x) - \intop_a^b v(t)w(t)\de t \right]\varphi(x)\de x\ . \]
    Infine applicando il lemma \ref{rem:lemma fondamentale cdv super} nella sua forma più forte, in quanto per l'arbitrarietà di $\varphi$ vale 
    \[ v - \underbrace{\intop_a^b v(x)w(x) \de x}_{\in \R} \equiv 0 \Rarr v \equiv \intop_a^b v(x)w(x) \de x\ . \]
    \qed
\end{lemma}

Applicandolo otteniamo che $b(x)-A(x) \equiv c \in \R$, dunque $f_p(x, u_0(x), u_0'(x)) = b(x) = c+A(x) \in \mc{C}^1([a,b])$, il che significa che abbiamo ottenuto "gratis" le ipotesi più forti per il teorema \ref{th:equazioni di E-L I}.

%Lezione 05/03/25

\subsection{Equazioni di Eulero--Lagrange con estremi liberi}

Proviamo ad alleggerire la situazione \ref{situaz:E-L estremi vincolati} e lavoriamo con questa nuova situazione, ovvero:
\[X = \mc{C}^1([a,b]), \qquad F(u) = \intop_a^b f(x,u(x),u'(x))\de x \qquad \text{con } f \in \mc{C}^1([a,b]\times\R\times\R) .\]
Avendo reso più generale $X$, cosa dobbiamo richiedere su $F$ per poter usare ancora la tesi del teorema \ref{th:equazioni di E-L I}?
Assumendo come al solito $u_0$ come minimo di $F$, definiamo per $\varphi \in \mc{C}^1([a,b])$\footnote{attenzione, non necessariamente a supporto compatto!} e $t \in\R$ la funzione $u_t = u_0 + t\varphi$ che ovviamente appartiene a $X$ e quindi $g(t) = F(u_t)$ che quindi ha minimo in $t=0$. Vediamo che:
\[\begin{aligned}
 0 = g'(0) & = \intop_a^b f_s(x, u_0(x), u_0'(x))\varphi(x) - f_p(x,u_0(x),u_0'(x))\varphi'(x)\de x = \\
 & = \intop_a^b \underbrace{f_s(x, u_0(x), u_0'(x))\varphi(x) - \left[\frac{\di}{\di x}f_p(x,u_0(x),u_0'(x))\right]\varphi(x)}_{=0\text{ per la tesi che vogliamo}} \de x + \left[ f_p(x,u_0(x),u_0'(x))\varphi(x) \right]_a^b = \\
 & = f_p(b,u_0(b), u_0'(b))\varphi(b) - f_p(a,u_0(a), u_0'(a))\varphi(a)\qquad \forall \varphi \in X
\end{aligned}\]
Da ciò otteniamo delle cosiddette condizioni al bordo "naturali", ovvero:
\[ \begin{cases}
    f_p(b,u_0(b), u_0'(b))=0\\ f_p(a,u_0(a), u_0'(a)) = 0
\end{cases}\]
Che quindi andiamo a fissare in questa situazione:

\begin{situation}{Eulero--Lagrange con estremi liberi}{E-L estremi liberi}
    \[ X = \mc{C}^1([a,b]),\]
    \[ F(u) = \intop_a^b f(x,u(x),u'(x))\de x \quad \text{con } f \in \mc{C}^1([a,b]\times\R\times\R) : \begin{cases}
        f_p(b,u_0(b), u_0'(b))=0\\ f_p(a,u_0(a), u_0'(a)) = 0
    \end{cases} \]
    Con $u_0 \in X$ punto di minimo per $F$.
\end{situation}

Ecco dunque dimostrato:
\begin{theorem}{Equazioni di Eulero--Lagrange II}{equazioni di E-L II}
    La tesi del teorema \ref{th:equazioni di E-L I} vale anche nella situazione \ref{situaz:E-L estremi liberi}.
\end{theorem}

\section{Minimi locali}

\begin{definition}{Norme e distanze per funzioni continue e differenziabili}{norma e distanza C0 e C1}
    Siano $u,v \in \mc{C}^1([a,b])$, definiamo:\begin{itemize}
        \item $\|u\|_{\mc{C}^0([a,b])} := \max_{x \in[a,b]} |u(x)|$;
        \item $\dist_{\mc{C}^0([a,b])}(u,v) := \| v-u \|_{\mc{C}^0([a,b])}$;
        \item $\| u \|_{\mc{C}^1([a,b])} := \max_{x \in[a,b]} |u(x)| + \max_{x \in[a,b]} |u'(x)|$;
        \item $\dist_{\mc{C}^1([a,b])}(u,v) := \| v-u \|_{\mc{C}^1([a,b])}$.
    \end{itemize}
    Per alleggerire la notazione, nei casi dove non avremo timore di ambiguità scriveremo occasionalmente $\|u\|_0$ o anche solo $\|u\|$ e vale lo stesso per $\dist(u,v)$.
\end{definition}

\chapter{Metodi diretti}

\printbibliography[heading=bibintoc]

\end{document}