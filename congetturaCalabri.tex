\documentclass{article}
\usepackage{cianatex}

\title{Progressi sulla congettura di Calabri}
\author{Matilde Calabri, Filippo $\mc{L}$. Troncana}
\date{A.A. 2024/2025}

\begin{document}

\maketitle

\begin{definition}{Numeri binari}{}
    Sia $n \in \Z_+$, questo si dice \bemph{numero binario in base} $b$ se 
    \[n = \sum_{i \in I} b^i\]
    Con $I \subset \N$ finito. Chiamiamo \bemph{rango} di $n$ il valore $\rk(n) := \max{I}$\\
    Alternativamente possiamo definire l'insieme $B_b$ dei numeri binari per induzione
    \[\frac{}{1}[B_b0] \qquad \frac{n}{b n}[B_b1] \qquad \frac{n}{b n +1}[B_b2]\]
\end{definition}
\begin{remark}{}{}
    Tutti i numeri binari in base $b$ rappresentati in base $b$ hanno come cifre solo $0$ e $1$.
\end{remark}

\begin{definition}{Funzione conta divisori}{}
    Definiamo la funzione 
    \[ D : \Z_+ \to \Z_+ \qquad \text{come} \qquad D(n) = \#\{d \in \Z_+ : d|n\}\]
\end{definition}
\begin{lemma}{Parità di $D$}{parità}
    Abbiamo che $D(n)$ è dispari se e solo se $n$ è un quadrato perfetto.
    \begin{proof}
        Automaticamente, $D(n) = 1 \Harr n = 1$, quindi poniamo $n > 1$.\\
        Per il teorema fondamentale dell'aritmetica possiamo scrivere $n$ come
        \[n = \prod_{i \in I}p_i^{q_i} \qquad \text{dove } p_i \text{ è primo per ogni }i \in I \]
        E dato che $D$ è evidentemente moltiplicativa sui coprimi, abbiamo:
        \[D(n) = \prod_{i \in I} D(p_i^{q_i}) = \prod_{i \in I} \{p^0,...,p^{q_i}\} = \prod_{i \in I}(q_i +1)\]
        Dato che un prodotto di interi è dispari se e solo se tutti i fattori sono dispari, abbiamo che ogni $q_i$ deve essere $2k_i$ per qualche $k$, ovvero 
        \[n = \prod_{i \in I}p_i^{2k_i} = \left(\prod_{i \in I} p_i^{k_i} \right)^2 = m^2 \qquad \text{per qualche } m \in \Z_+\]
    \end{proof}
\end{lemma}

\begin{conjecture}{Congettura di Calabri I}{Calabri I}
    Sia $n \in B_{10}$ tale che $n \cong 1 \mod 2$. Allora $D(n) \cong 0 \mod 2$.
\end{conjecture}

Assumiamo che $n \in B_{10}$ sia un controesempio della congettura di Calabri, dunque $D(n)$ è dispari, e sappiamo già che $n$ deve essere dispari; ricordando che un quadrato è dispari se e solo se la sua radice è dispari, abbiamo che $n = (2k+1)^2 = 4k^2 + 4k$ per qualche $k \in \Z_+$.\\
A questo punto possiamo notare che $n-1 = 4(k^2 + k)$ dunque $4|n-1$ e al contempo $10| n-1$, perciò vale $100|n-1$ e perciò $25 | k^2 + k$. Automaticamente possiamo vedere che i casi sono due:
\begin{enumerate}
    \item \[ k \cong 0 \mod 25  \Rarr k = 25x \Rarr n = 4((25x)^2 + 25x) \Rarr n = 100(25x^2 + x )\]
    \item \[ k \cong -1 \mod 25 \Rarr k = 25x-1 \Rarr n = 4((25x-1)^2 + 25x - 1) \Rarr n = 100(25x^2 - x) \]
\end{enumerate}

Dato che $n-1 \in B_{10}$, dobbiamo indagare i numeri binari della forma $25x^2 \pm x$. Osserviamo che devono essere necessariamente pari, in quanto
\[ 25x^2 \pm x \cong x^2 \pm x \cong x(x \pm 1) \cong 0 \mod 2 \]
Allora $x = 2y$ per qualche $y$ e dunque $n-1 = 100(100y^2 + 2y)$ e quindi $n = (100y)^2 + 2(100y) + 1 = (100y +1)^2$. Abbiamo dunque l'uguaglianza
\[ n = (2k+1)^2 = (100y + 1)^2 \Rarr k = 50y \qquad \text{per l'iniettività di } m \mapsto (m+1)^2 \text{ su } \Z_+\] 
Abbiamo quindi che $\frac{n-1}{100} \in B_{10}$ e inoltre $\frac{n-1}{100} \cong 0 \mod 2$, dunque 
$1000|n-1$, perciò Abbiamo

\begin{conjecture}{Congettura di Calabri II}{Calabri II}
    Gli unici numeri binari in base 10 che sono anche quadrati perfetti sono della forma $10^{2n}$.
\end{conjecture}
Osserviamo che la congettura \ref{conjecture:Calabri II} implica la \ref{conjecture:Calabri I}



\end{document}