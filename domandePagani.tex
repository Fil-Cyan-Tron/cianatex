\documentclass{article}
\usepackage{cianatex}
\usepackage{cianacolors}
\usepackage{cianatheorems}

\title{Domande Pagani FFM2}
\author{Filippo Troncana}
\date{A.A. 2024/2025}

\renewcommand\C{\mc{C}}
\newcommand\del{\partial}

\begin{document}

\maketitle

\tableofcontents

\section{Classificazione delle PDE quasilineari del secondo ordine}

Fissato $U \subset \R^2$ un aperto, si dice \bemph{equazione differenziale alle derivate parziali del secondo ordine quasilineare} con incognita $u \in \C^2(U)$ un'espressione della forma 

\[ a\frac{\del^2 u}{\del x^2} + 2b\frac{\del^2 u}{\del x\del y} + c\frac{\del^2 u}{\del y^2} = d\]

dove $a,b,c,d$ sono funzioni della forma $f(x,y,u,u_x,u_y)$. Consideriamo una curva regolare $\gamma : I \to U$ con $\gamma(s) = (x(s), y(s))$ e, con l'abuso di notazione $f|_\gamma := f \circ \gamma$, i dati iniziali 

\[ u|_\gamma = h(s), \quad u_x|_\gamma  = \phi(x), \quad u_y|_\gamma = \psi(x)\]

Con $h,\phi,\psi$ funzioni date. Queste devono rispettare certi vincoli, infatti applicando la regola della catena otteniamo

\[ \frac{\di u|_\gamma}{\di s}= u_x|_\gamma \cdot x'+ u_y|_\gamma \cdot y' \Harr h' = \phi\cdot x' + \psi\cdot y' \]

Assumendo che una soluzione $u$ esista e tutto sia sufficientemente regolare, consideriamo le sue derivate seconde

\[ \frac{\di u_x|_\gamma}{\di s} = u_{xx}|_\gamma x' + u_{xy}|_\gamma y'=\phi',\quad \frac{\di u_y|_\gamma}{\di s} = u_{yx}|_\gamma x' + u_{yy}|_\gamma y'=\psi'\]

Quindi ricordando l'equazione iniziale, gli assunti di regolarità e componendo con $\gamma$ otteniamo

\[ a|_\gamma u_{xx}|_\gamma + 2b|_\gamma u_{xy}|_\gamma + c|_\gamma u_{yy}|_\gamma = d|_\gamma\]

Otteniamo un sistema lineare nelle derivate seconde di $u$:

\[\underbrace{\begin{pmatrix} a|_\gamma & 2b|_\gamma & c|_\gamma \\ x' & y' & 0 \\ 0 & x' & y' \end{pmatrix}}_{A} \underbrace{\begin{pmatrix} u_{xx}|_\gamma \\ u_{xy}|_\gamma \\ u_{yy}|_\gamma \end{pmatrix}}_{u''} = \underbrace{\begin{pmatrix} d|_\gamma \\ \phi' \\ \psi' \end{pmatrix}}_{k} \]

Denotando con $\Delta(s) := \det A(s)$ otteniamo tre casi:\begin{itemize}
    \item $\Delta \neq 0$ su tutta la curva, ovvero esiste ed è unico $u''(s)$ che soddisfa l'equazione.
    \item $\Delta = 0$ su tutta la curva \bemph{in generale} non ci dà una soluzione, ma solamente se $\rk(A|k) = \rk(A)$, che comunque non ci garantisce l'unicità; in particolare, calcolando esplicitamente il determinante, vediamo che equivale a dire:
    \[ a|_\gamma \cdot (y')^2 - 2b|_\gamma \cdot x' \cdot y' + c|_\gamma \cdot (x')^2=0  \]
    E in questo caso si dice che la curva $\gamma$ è una \bemph{curva caratteristica}. In particolare, rinominando la variabile $s$ in $t$ e assumendo che  $(x(t),y(t)) = (t, f(t))$ ottengo 
    \[ a|_\gamma \cdot (f')^2 - 2b|_\gamma \cdot f' + c|_\gamma =0 \Rarr f' = \frac{ 2b|_\gamma \pm \sqrt{(2b|_\gamma)^2 - 4 a|_\gamma \cdot c|_\gamma} }{2a|_\gamma} \]
    Questa condizione ci divide in tre casi:\begin{itemize}
        \item $(2b|_\gamma)^2 - 4 a|_\gamma \cdot c|_\gamma > 0$, detto \bemph{iperbolico}, dove ho due derivate distinte e dunque due famiglie di curve caratteristiche.
        \item $(2b|_\gamma)^2 - 4 a|_\gamma \cdot c|_\gamma = 0$, detto \bemph{parabolico}, in cui ho una sola famiglia di derivate e dunque di curve caratteristiche.
        \item $(2b|_\gamma)^2 - 4 a|_\gamma \cdot c|_\gamma > 0$, detto \bemph{ellittico}, in cui non ho curve caratteristiche.
    \end{itemize}
    \item Negli altri casi coi nostri strumenti non abbiamo considerazioni interessanti dal punto di vista fisico.
\end{itemize}

\section{Superfici caratteristiche per PDC e teorema di Cauchy--Kovalevskaja}

Assumendo che la curva caratteristica $\gamma$ della nostra equazione sia della forma $(t,f(t))$ (o simmetricamente della forma $(f(t),t)$), proviamo a descrivere $\gamma$ come luogo di zeri di una funzione $F(x,y)$. Sappiamo che 
\[ a|_\gamma (f')^2 - 2b|_\gamma f' + c|_\gamma = 0 \]
Supponendo che $F(t,f(t))=0$ posso derivare totalmente e ottengo:
\[ \frac{\di F}{\di t} = F_x \cdot 1 + F_y f' \Rarr f' = - \frac{F_x}{F_y} \]
E dunque la condizione di cui sopra diventa (moltiplicando per $(F_y)^2$ per eliminare i denominatori)
\[ a|_\gamma (F_x)^2 + 2b|_\gamma F_x F_y + c|_\gamma (F_y)^2 = 0 \]
Questa condizione vale per le curve, ma in realtà si generalizza abbastanza facilmente a quello delle (iper)superfici in $\R^{n+1}$ grazie al teorema di invertibilità locale. Generalizziamo la nostra equazione ad una scrittura della forma
\[ \sum_{i,j = 1}^n a^{i,k} u_{i,j} = d \]
Dove $a^{i,k}$ e $d$ sono funzioni in $(x_1,..., x_n, u_1, ... , u_n)$; posta $F(x_1,..., x_n)$ la funzione di cui ipotizziamo l'esistenza, otteniamo la condizione
\[ \sum_{i,j=1}^n a^{i,k} F_i F_j = 0 \]
E da questa una "trasmissione di regolarità" analoga a quanto visto per le curve.

\begin{theorem}{Teorema di Cauchy--Kovalevskaja}{TdC--K}
    Data una curva $\gamma : I \to \R^2$ regolare, $a,b,c$ e $d$ funzioni analitiche nelle variabili $x,y, u_x, u_y$ e dati $u|_\gamma, u_x|_\gamma$ e $u_y|_\gamma$ analitici, esiste ed è unica in un intorno del supporto di $\gamma$ una soluzione analitica $u$ dell'equazione:
    \[ a\cdot \cdot u_{xx} + 2b \cdot u_{xy} + c \cdot  u_{yy} = d \]
\end{theorem}

\section{Ricavare da un modello fisico a scelta l'equazione delle onde. Scrivere la soluzione di d'Alembert per la retta e la semiretta con condizioni al bordo di Dirichlet e Von Neumann}

\subsection{Derivazione (onde trasversali ad una corda tesa)}

Procediamo a derivare l'equazione delle onde per le vibrazioni trasversali su una corda tesa tra due punti $(0,0)$ e $(L,0)$ nel piano $xu$. Assumiamo che la posizione di ciascun punto della corda vari ortogonalmente alla corda stessa, permettendoci di descrivere lo spostamento di un punto $x$ con la funzione $u = u(x,t)$, in modo che la curva $\gamma_t =(x, u(x,t))$ rappresenti la corda al momento $t$.\\
Posti due punti $x_1, x_2$ sulla corda, la quantità di moto tra i due è diretta lungo il versore $\hat{e}_u$ trasversale alla corda ed è data da

\[ \intop_{x_1}^{x_2} \rho(\xi) u_t(\xi, t)\de \xi \]

Dove $\rho(\xi)$ è la densità della corda e $u_t(\xi,t)$ è la velocità del punto $\xi$ lungo $\hat{e}_u$ al tempo $t$.\\
I versori tangenti alla corda nei due punti sono dati da:

\[ T(x_i,t) = \frac{ \hat{e}_x + \hat{e}_u u_{x}(x_i,t) }{\sqrt{ 1+(u_{x}(x_i,t)) }} \]

Possiamo fare tre considerazioni:\begin{enumerate}
    \item Chiamando $A,B$ e $C$ i tratti della corda rispettivamente precedente $x_1$, compreso tra $x_1$ e $x_2$ e successivo a $x_2$ e assumendo che la tensione $\tau$ sia costante, abbiamo che:\begin{itemize}
        \item La forza di $A$ su $B$ in $(x_1,t)$ è $-\tau T(x_1, t)$.
        \item Analogamente, la forza di $C$ su $B$ in $(x_2,t)$ è $\tau T(x_2, t)$
    \end{itemize} 
    E dunque la forza agente su $B$ è $F(t) := \tau\left(T(x_2,t) - T(x_1,t)\right)$
    \item Assumendo variazioni di piccola ampiezza, ovvero $u_x \ll 1$ abbiamo
    \[ F(t) \approx \tau\left(u_x(x_2,t) - u_x(x_1,t)\right) \hat{e}_u \]
    \item Assumiamo che qualsiasi forza esterna agisca sulla corda lo faccia in $B$ e sia della forma $f(\xi,t)\hat{e}_u$, dove $f$ indica la "densità lineare" di forza agente al tempo $t$. 
\end{enumerate}
Consideriamo l'intervallo di tempo $[t_1, t_2]$ e applichiamo la forma integrale della seconda legge di Newton\footnote{$\Delta p = \int F\de t $} per la variazione della quantità di moto $p$ del tratto $B$ nell'intervallo di tempo $[t_1,t_2]$:

\[ \intop_{x_1}^{x_2} \rho(\xi) u_t(\xi, t_2)\de \xi - \intop_{x_1}^{x_2} \rho(\xi) u_t(\xi, t_1)\de \xi = \intop_{t_1}^{t_2} \tau\left(u_x(x_2,t) - u_x(x_1,t)\right) \de t + \intop_{t_1}^{t_2}\intop_{x_1}^{x_2} f(\xi,t) \de \xi \de t \]

Definiamo le funzioni integrali

\[ \intop_{t_1}^{t_2} u_{tt}(x, t) \de t = u_t(x,t_2) - u_t(x, t_1) ,\qquad \intop_{x_1}^{x_2} u_{xx}(x,t)\de t = u_x(x_2,t) - u_x(x_1, t) \]

Per generalità di $[x_1,x_2]$ posso estendere le mie equazioni a tutto il dominio e sostituire ottenendo

\[ \intop_{x_1}^{x_2}\intop_{t_1}^{t_2} \rho(\xi) u_{tt}(\xi,t) - \tau u_{xx}(\xi,t) - f(\xi,t) \de t \de \xi \equiv 0 \Harr \rho(x) u_{tt}(x,t) - \tau u_{xx}(x,t)\equiv f(x,t) \]

Dividendo tutto per $\rho(x)$ e ponendo $a^2 := \frac{\tau}{\rho(x)}$, termine che corrisponde alla velocità di propagazione, otteniamo l'equazione

\[ u_{tt}(x,t) - a^2 u_{xx}(x,t) = \frac{f(x,t)}{\rho(x,t)} \]

Detta \bemph{equazione delle onde}, una PDE del secondo ordine quasilineare iperbolica.

\subsection{Soluzioni}

\subsubsection{Retta}

Assumiamo una corda di lunghezza infinita, densità costante e con forze esterne nulle, otteniamo il problema:

\[ \begin{cases}
    u_{tt} - a^2 u_{xx} = 0 & \forall x,t \in \R\\
    u(x,0) = \varphi(x) & \forall x \in \R\\
    u_t(x,0) = \psi(x) & \forall x \in \R
\end{cases} \]

Questa equazione ha come soluzione (ricavata usando le curve caratteristiche $x = \pm at$ come assi) la \bemph{soluzione di D'Alembert}:

\[ u(x,t) = \frac{\varphi(x+at) + \varphi(x-at) }{2} + \frac{1}{2a} \intop_{x-at}^{x+at} \psi(\xi)\de \xi\]

\subsubsection{Semiretta, condizioni di Dirichlet}

Consideriamo il problema definito sulla semiretta $\R_{\ge 0}$ con la condizione al bordo di Dirichlet

\[ \begin{cases}
    u_{tt} - a^2 u_{xx} = 0 & \forall t \in \R, \forall x \ge 0\\
    u(x,0) = \varphi(x) & \forall x \ge 0\\
    u_t(x,0) = \psi(x) & \forall x \ge 0\\
    u(0,t) = 0 & \forall t \in \R
\end{cases} \]

Possiamo trattarlo come il problema definito sulla retta definendo i prolungamenti dispari delle condizioni, ignorando momentaneamente la condizione di Dirichlet:

\[ \Phi(x) = \begin{cases} \varphi(x) & x\ge 0 \\ -\varphi(-x) & x<0\end{cases}, \quad \Psi(x) = \begin{cases}\psi(x) & x\ge 0 \\ -\psi(-x) & x<0\end{cases}\]

E procedendo come sopra con la formula di D'Alembert 

\[u(x,t) = \frac{\Phi(x+at) + \Phi(x-at) }{2} + \frac{1}{2a} \intop_{x-at}^{x+at} \Psi(\xi)\de \xi \]

Valutando la soluzione in $(0,t)$, dato che sia $\Phi$ che $\Psi$ sono dispari, otteniamo

\[ u(0,t) = \frac{\Phi(at) + \Phi(-at) }{2} + \frac{1}{2a} \intop_{-at}^{at} \Psi(\xi)\de \xi \equiv 0 \]

\subsubsection{Semiretta, condizioni di Neumann}

Consideriamo il problema definito sulla semiretta $\R_{\ge 0}$ con la condizione al bordo di Neumann

\[ \begin{cases}
    u_{tt} - a^2 u_{xx} = 0 & \forall t \in \R, \forall x \ge 0\\
    u(x,0) = \varphi(x) & \forall x \ge 0\\
    u_t(x,0) = \psi(x) & \forall x \ge 0\\
    u_x(0,t) = 0 & \forall t \in \R
\end{cases} \]

Possiamo trattarlo come il problema definito sulla retta definendo i prolungamenti pari delle condizioni, ignorando momentaneamente la condizione di Neumann:

\[ \Phi(x) = \begin{cases} \varphi(x) & x\ge 0 \\ \varphi(-x) & x<0\end{cases}, \quad \Psi(x) = \begin{cases}\psi(x) & x\ge 0 \\ \psi(-x) & x<0\end{cases}\]

E procedendo come sopra con la formula di D'Alembert

\[u(x,t) = \frac{\Phi(x+at) + \Phi(x-at) }{2} + \frac{1}{2a} \intop_{x-at}^{x+at} \Psi(\xi)\de \xi \]

Adesso, deriviamo la soluzione:

\[ u_x(x,t) = \frac{\Phi'(x+at) + \Phi'(x-at)}{2} + \frac{\Psi(x+at) - \Psi(x-at)}{2} \]

Dato che $\Phi$ e $\Psi$ sono pari, dunque $\Phi'$ è dispari, otteniamo che valutando in $(0,t)$:

\[ u_x(0,t) = \frac{\Phi'(at) + \Phi'(-at)}{2} + \frac{\Psi(at) - \Psi(-at)}{2} \equiv 0\]

\section{Calore}

\subsection{Soluzione dell'equazione del calore sul segmento $[0,L]$ con assegnati dati iniziali e condizioni al bordo omogenee di Dirichlet; giustificare rigorosamente la serie formale che rappresenta la
soluzione.}

Consideriamo il problema di Dirichlet con condizioni al bordo omogenee per l'equazione del calore sul segmento $[0,L]$ con $L>0$:

\[ \begin{cases} u_t - a^2 u_{xx} = 0 & \forall x \in (0,L), \forall t > 0 \\ u(x,0) = \varphi(x) & \forall x \in [0,L] \\ u(0,t) = u(L,t) = 0 & \forall t \ge 0 \end{cases}\]

Assumendo sufficiente regolarità per il dato $\varphi(x)$, calcoliamo i suoi coefficienti di Fourier per ogni $n \in \Z^+$:

\[ c_n := \frac{2}{L}\intop_0^L \varphi(\xi) \sin\left( \frac{\pi n}{L}\xi \right)\de\xi, \quad \varphi_n (x) := c_n \sin\left( \frac{\pi n}{L}x \right) \]

Dunque possiamo scrivere la nostra soluzione come 

\[ u(x,t) = \sum_{n\in \Z^{+}} u_n(x,t), \quad \text{dove}\quad u_n(x,t) := c_n e^{- a^2 \left( \frac{\pi n}{L}\right)^2 t }\sin \left( \frac{\pi n}{L}x \right)\]

A priori questa serie è puramente formale, dobbiamo verificare che abbia effettivo senso: osserviamo che possiamo derivare due volte, dato che assumendo $t \ge t_0$ con $t_0$ arbitrariamente piccolo e $x \in (0,L)$ valgono le seguenti stime sfruttando il termine esponenziale:

\[ \left| \frac{\partial u_n}{\partial t}(x,t) \right| < |\varphi_n(x)| \left(\frac{\pi n}{L}\right)^{2} a^{2} e^{- a^2 \left( \frac{\pi n}{L}\right)^2 t_0 },\quad \left| \frac{\partial^{2}u_n}{ \partial x^2}(x,t) \right| < |\varphi_n(x)| \left(\frac{\pi n}{L}\right)^{2} e^{- a^2 \left( \frac{\pi n}{L}\right)^2 t_0 } \]

Possiamo assumere ragionevolmente che la temperatura iniziale non sia infinita e che quindi ogni $\varphi_n$ sia limitata da un qualche $2M > 0$, dunque otteniamo convergenza uniforme per le serie 

\[ u_t(x,t) = \sum_{n\in \Z^+} \frac{\partial u_n}{\partial t}(x,t), \quad u_{xx}(x,t) = \sum_{n\in \Z^+} \frac{\partial^2 u_n}{\partial x^2}(x,t) \]

Notiamo che le ipotesi di regolarità minime che abbiamo fatto su $\varphi$ sono sufficienti a garantire la convergenza per tutte le derivate di $u$ in $[0,L]\times [t_0, +\infty)$, infatti vediamo che vale

\[ \left| \frac{\partial^{k+l}u_n}{\partial t^k \partial x^l}(x,t) \right| < 2M \left(\frac{\pi n}{L}\right)^{2k+l} a^{2k} e^{- a^2 \left( \frac{\pi n}{L}\right)^2 t_0 } \]

Per l'arbitrarietà di $t_0$ inoltre posso estendere questa regolarità a tutto $[0,L]\times (0,+\infty)$.\\
Infine dobbiamo controllare che $u$ sia continua: dato che $|u_n|$ è sempre maggiorata da $|\varphi_n|$, è sufficiente richiedere la continuità a tratti di $\varphi$ su $[0,L]$ e $\varphi(0)=\varphi(L) = 0$ per ottenere la continuità di $u$ su tutto $[0,L]\times [0,+\infty)$. Notiamo che definendo

\[ G(x,t,\xi) := \frac{2}{L}\sum_{n \in \Z^+} \sin\left( \frac{\pi n}{L}\xi \right)\sin\left( \frac{\pi n}{L}x \right) e^{-a^2 \left( \frac{\pi n}{L} \right)^2 t}\] 

Possiamo scrivere la nostra soluzione come 

\[ u(x,t)=  \intop_0^L  G(x,t,\xi)\varphi(\xi) \de\xi \]

\subsection{Determinare la soluzione dell'equazione del calore con termine non omogeneo e temperatura iniziale nulla sul segmento $[0,L]$.}

Consideriamo il problema non omogeneo (ovvero con una sorgente di calore) per l'equazione del calore sul segmento $[0,L]$ con $L>0$ con temperatura iniziale nulla:

\[ \begin{cases} u_t - a^2 u_{xx} = f(x,t) & \forall x \in (0,L), \forall t > 0 \\ u(x,0) = 0 & \forall x \in [0,L] \\ u(0,t) = u(L,t) = 0 & \forall t \ge 0 \end{cases}\]

La chiave è assumere che $u$ possa essere rappresentata in serie di Fourier come $f$ e sostanzialmente fare derivate usando l'equazione, notando che i coefficienti $u_n$ e $f_n$ sono funzioni in $t$, dunque otteniamo un'ODE per ciascun $u_n$ che ci da

\[ u_n(t) = \intop_0^t e^{-a^2\left( \frac{\pi n}{L} \right)^2 (t-\tau)} f_n(\tau) \de\tau \]

Riarrangiando e scambiando vari integrali e sommatorie otteniamo la soluzione

\[ u(x,t) = \intop_0^t\intop_0^L G(x,t-\tau,\xi)f(\xi,\tau)\de\xi\de\tau \]

\subsection{Soluzione dell'equazione del calore con termine non omogeneo e temperatura iniziale non nulla sul segmento $[0,L]$.}

Consideriamo il problema non omogeneo (ovvero con una sorgente di calore) per l'equazione del calore sul segmento $[0,L]$ con $L>0$ con temperatura iniziale assegnata:

\[ \begin{cases} u_t - a^2 u_{xx} = f(x,t) & \forall x \in (0,L), \forall t > 0 \\ u(x,0) = \varphi(x) & \forall x \in [0,L] \\ u(0,t) = u(L,t) = 0 & \forall t \ge 0 \end{cases}\]

Per risolverlo, consideriamo l'operatore differenziale:

\[ D(u)(x,t) := \frac{\partial u}{\partial t}(x,t) - a^2 \frac{\partial^2 u}{\partial x^2}(x,t) \]

Questo è lineare e ci permette di riscrivere il nostro problema come  

\[ \begin{cases} D(u)(x,t) = f(x,t) & \forall x \in (0,L), \forall t > 0 \\ u(x,0) = \varphi(x) & \forall x \in [0,L] \\ u(0,t) = u(L,t) = 0 & \forall t \ge 0 \end{cases}\]

Ora consideriamo $v,w$ soluzioni dei seguenti problemi:

\[ \begin{cases} v_t - a^2 v_{xx} = 0 & \forall x \in (0,L), \forall t > 0 \\ v(x,0) = \varphi(x) & \forall x \in [0,L] \\ v(0,t) = u(L,t) = 0 & \forall t \ge 0 \end{cases}, \qquad \begin{cases} w_t - a^2 w_{xx} = f(x,t) & \forall x \in (0,L), \forall t > 0 \\ w(x,0) = 0 & \forall x \in [0,L] \\ w(0,t) = w(L,t) = 0 & \forall t \ge 0 \end{cases}\]

Che abbiamo mostrato essere della forma (assumendo che $\varphi$ e $f$ ammettano serie di Fourier convergenti)

\[v(x,t) = \intop_0^L G(x, t, \xi)\varphi(\xi)\de\xi\quad\text{e} \quad w(x,t) = \intop_0^t\intop_0^L G(x,t-\tau,\xi) f(\xi,\tau)\de\xi\de\tau  \]  

Dunque mostriamo $u(x,t) = v(x,t)+w(x,t)$:

\[ \begin{cases} D(v+w)(x,t) = \underbrace{D(v)(x,t)}_{=0} + \underbrace{D(w)(x,t)}_{=f(x,t)} = f(x,t) & \forall x \in (0,L), \forall t > 0 \\ \underbrace{v(x,0)}_{= \varphi(x)} + \underbrace{w(x,0)}_{=0}  = \varphi(x) & \forall x \in [0,L] \\ u(0,t) = u(L,t) = 0 & \forall t \ge 0 \end{cases}\]

\end{document}