\documentclass{article}

\usepackage{cianatex}

\title{La metrica di Simpson}
\author{Filippo L. Troncana}
\date{Tanto tanto tempo fa, in una galassia lontana lontana}

\begin{document}
\maketitle

\section{Introduzione}

Nel decimo episodio della quinta stagione dei Simpson, Homer enuncia il seguente risultato:

\begin{theorem}
    La somma delle radici quadrate di due lati di un triangolo isoscele è uguale alla radice quadrata del lato rimanente.
\end{theorem}

Nella stessa scena, viene immediatamente corretto da una comparsa, producendo il seguente risultato:

\begin{theorem}[Pitagora-Simpson]
    La somma delle radici quadrate dei cateti di un triangolo rettangolo è uguale alla radice quadrata dell'ipotenusa
\end{theorem}

Ovviamente questo non è il nostro teorema di Pitagora, ma cerchiamo di dargli un senso.

\begin{definition}
    Per $n \in \Z^+$, definiamo la \textbf{Norma di Simpson}
    \[\definefunction{S}{\R^n}{[0,+\infty[}{(x_1,...,x_n)}{(\sqrt{|x_1|}+...+\sqrt{|x_n|})^2}\]
\end{definition}

\end{document}